\documentclass[b5paper, openany]{ctexbook}


\usepackage[margin=2.5cm]{geometry}


\usepackage{pifont}
\usepackage[perpage,symbol*]{footmisc}
\DefineFNsymbols{circled}{{\ding{192}}{\ding{193}}{\ding{194}}
{\ding{195}}{\ding{196}}{\ding{197}}{\ding{198}}{\ding{199}}{\ding{200}}{\ding{201}}}
\setfnsymbol{circled}



\usepackage{amsmath,amsfonts,mathrsfs,amssymb}
\usepackage{graphicx}

\usepackage[font=bf,labelfont=bf,labelsep=quad]{caption}

\usepackage{tikz}
%\usepackage{nicematrix}
\usepackage{bookmark}
\usepackage{ntheorem}
\theoremseparator{\;}



\usepackage{blkarray}
\usepackage{bm}
%\usepackage[colorlinks=true, linkcolor=black]{hyperref}

%\usepackage{enumerate}


\theoremstyle{plain}
\theoremheaderfont{\normalfont\bfseries} 
\theorembodyfont{\normalfont}


\usepackage[framemethod=tikz]{mdframed}



\newtheorem{example}{\bf 例}[chapter]
\newenvironment{solution}{\noindent {\bf 解:}}{}  %{\hfill $\clubsuit$\par}
\newenvironment{analyze}{\noindent {\bf 分析:}}{}
\newenvironment{rmk}{\noindent {\bf 注意:}}{}
\newenvironment{note}{\noindent {\bf 说明:}}{}



\renewcommand{\proofname}{\bf 证明:}
\newenvironment{proof}{{\noindent \bf 证明:}}{}%{\hfill $\square$\par}

\newcommand{\E}{\mathbb{E}}
\renewcommand{\Pr}{\mathbb{P}}
\newcommand{\EP}{\mathbb{E}^{\mathbb{P}}}
\newcommand{\EQ}{\mathbb{E}^{\mathbb{Q}}}
\newcommand{\dif}{\,{\rm d}}
\newcommand{\Var}{{\rm Var}}
\newcommand{\Cov}{{\rm Cov}}
\newcommand{\x}{\times}


 \usepackage{tcolorbox}
 \tcbuselibrary{breakable}
 \tcbuselibrary{most}

% \tcolorboxenvironment{tl}{colback = cyan!5!white, colframe = cyan!75!black,
%     colbacktitle = cyan!85!black, enhanced,  breakable}

% \tcolorboxenvironment{thm}{colback = magenta!5!white, colframe = magenta!75!black, 
%     colbacktitle = magenta!85!black, enhanced,     breakable}

% \tcolorboxenvironment{yl}{colback = red!5!white, colframe = red!75!black, 
%     colbacktitle = red!85!black, enhanced,     breakable}

% \tcolorboxenvironment{defn}{colback = blue!5!white, colframe = blue!75!black,
%     colbacktitle = blue!85!black, enhanced,    breakable
% }


% \tcolorboxenvironment{ex}{colback = cyan!5!white, colframe = cyan!75!black, fonttitle = \bfseries,
%    colbacktitle = cyan!85!black, enhanced,
%    attach boxed title to top center={yshift=-2mm}, breakable}

\newtcolorbox{ex}[1][]
  {colback = white, colframe = cyan!75!black, fonttitle = \bfseries,
    colbacktitle = cyan!85!black, enhanced,
    attach boxed title to top center={yshift=-2mm},breakable, 
    title=练习, #1}

\newtcolorbox{blk}[2][]
  {colback = white, colframe = magenta!75!black, fonttitle = \bfseries,
    colbacktitle = magenta!85!black, enhanced,
    attach boxed title to top left={xshift=5mm, yshift=-2mm},breakable, 
    title=#2, #1}


\setcounter{tocdepth}{2}

\setcounter{secnumdepth}{3}



\ctexset {
section = {
	name = {第,节},
 	number = \chinese{section}},
subsection = {
	name = {,、\hspace{-1em}},
	number = \chinese{subsection}
},
subsubsection = {
	name = {(,)\hspace{-1em}},
	number = \chinese{subsubsection},
}
}



\renewcommand{\contentsname}{目~~录}

\newcommand{\poly}{\polynomial[reciprocal]}



\usepackage{mathtools}

\setlength{\abovecaptionskip}{0.cm}
\setlength{\belowcaptionskip}{-0.cm}

\usetikzlibrary{decorations.pathmorphing, patterns}
\usetikzlibrary{calc, patterns, decorations.markings}
\usetikzlibrary{positioning, snakes}

\newcommand{\Lim}{\displaystyle\lim}

\usepackage{yhmath}
\usepackage{longdivision}
\usepackage{polynom}
\usepackage{polynomial}
\usepackage{multicol}

\renewcommand{\frac}{\dfrac}
\newcommand{\oc}{$^{\circ}{\rm C}$}
\usepackage{longtable}
\usepackage{tkz-euclide, tkz-base}



\usepackage{cases}
\begin{document}
%\fontsize{10.5}{11}\selectfont














\title{中学数学实验教材\\第六册}



\author{中学数学实验教材编写组编}
\date{1981年6月}

\maketitle




\frontmatter

%\chapter{前~~言}

这一套中学数学实验教材,内容的选取原则是精简实
用,教材的处理力求深入浅出,顺理成章,尽量作到使人人
能懂,到处有用.

    本教材适用于重点中学,侧重在满足学生将来从事理工
方面学习和工作的需要.

    本教材的教学目的是:使学生切实学好从事现代生产、
特别是学习现代科学技术所必需的数学基础知识;通过对数
学理论、应用、思想和方法的学习,培养学生运算能力,思
维能力,空间想象力,从而逐步培养运用数学的思想和方法
去分析和解决实际问题的能力;通过数学的教学和学习,培
养学生良好的学习习惯,严谨的治学态度和科学的思想方
法,逐步形成辩证唯物主义世界观.

   根据上述教学目的,本教材精选了传统数学那些普遍实
用的最基础的部分,这就是在理论上、应用上和思想方法上
都是基本的、长远起作用的通性、通法.比如,代数中的数
系运算律,式的运算,解代数方程,待定系数法;几何中的
图形的基本概念和主要性质,向量,解析几何;分析中的函
数,极限,连续,微分,积分;概率统计以及逻辑、推理论
证等知识.对于那些理论和应用上虽有一定作用,但发展余
地不大,或没有普遍意义和实用价值,或不必要的重复和过
于繁琐的内容,如立体几何中的空间作图,几何体的体积、
表面积计算,几何难题,因式分解,对数计算等作了较大的
精简或删减.

    全套教材共分六册.第一册是代数.在总结小学所学自
然数、小数、分数基础上,明确提出运算律,把数扩充到有
理数和实数系.灵活运用运算律解一元一次、二次方程,二
元、三元一次方程组,然后进一步系统化,引进多项式运
算,综合除法,辗转相除,余式定理及其推论,学到根式、
分式、部分分式.第二册是几何.由直观几何形象分析归纳
出几何基本概念和基本性质,通过集合术语、简易逻辑转入
欧氏推理几何,处理直线形,圆、基本轨迹与作图,三角比
与解三角形等基本内容.第三册是函数.数形结合引入坐
标,研究多项式函数,指数、对数、三角函数,不等式等.
第四册是代数.把数扩充到复数系,进一步加强多项式理论,
方程式论,讲线性方程组理论,概率(离散的)统计的初步
知识.第五册是几何.引进向量,用向量和初等几何方法综
合处理几何问题,坐标化处理直线、圆、锥线,坐标变换与
二次曲线讨论,然后讲立体几何,并引进空间向量研究空间
解析几何初步知识.第六册是微积分初步.突出逼近法,讲
实数完备性,函数,极限,连续,变率与微分,求和与积分.

本教材基本上采取代数、几何、分析分科,初中、高中
循环排列的安排体系.教学可按初一、初二代数、几何双科
并进,初三学分析,高一、高二代数(包括概率统计)、几
何双科并进,高三学微积分的程序来安排.

    本教材的处理力求符合历史发展和认识发展的规律,深
入浅出,顺理成章.突出由算术到代数,由实验几何到论证
几何,由综合几何到解析几何,由常量数学到变量数学等四
个重大转折,着力采取措施引导学生合乎规律地实现这些转
折,为此,强调数系运算律,集合逻辑,向量和逼近法分别
在实现这四个转折中的作用.这样既遵循历史发展的规律,
又突出了几个转折关头,缩短了认识过程,有利于学生掌握
数学思想发展的脉络,提高数学教学的思想性.

这一套中学数学实验教材是教育部委托北京师范大学、
中国科学院数学研究所、人民教育出版社、北京师范学院、
北京景山学校等单位组成的领导小组组织“中学数学实验教
材编写组”,根据美国加州大学伯克利分校数学系项武义教
授的《关于中学实验数学教材的设想》编写的.第一版印出
后,由教育部实验研究组和有关省市实验研究组指导在北
京景山学校、北京师院附中、上海大同中学、天津南开中
学、天津十六中学、广东省实验中学、华南师院附中、长春
市实验中学等校试教过两遍,在这个基础上编写组吸收了实
验学校老师们的经验和意见,修改成这一版《中学数学实验
教材》,正式出版,内部发行,供中学选作实验教材,教师
参考书或学生课外读物.在编写和修订过程中,项武义教授
曾数次详细修改过原稿,提出过许多宝贵意见.

    本教材虽然试用过两遍,但是实验基础仍然很不够,这
次修改出版,目的是通过更大范围的实验研究,逐步形成另
一套现代化而又适合我国国情的中学数学教科书.在实验过
程中,我们热忱希望大家多提意见,以便进一步把它修改好.

\begin{flushright}
    中学数学实验教材编写组\\
    一九八一年三月
\end{flushright}








\tableofcontents


\mainmatter

\chapter{函数的极限和连续函数的性质}

\section{函数的极限}
\subsection{函数极限的概念}

在第四册下,我们研究了数列的极限,数列是一种特殊的函数,这里的自变数$n$取自然数列$1, 2, 3,\ldots,n,\ldots$的值,现在我们来研究更一般的情形,即函数$f(x)$随$x$连续变化而变化的情形,下面转到一般函数的极限。

\begin{blk}{定义1}
    如果$x$通过任何一个无限增大的数列$\{x_n\}$, 对应的函数值数列
$f (x_1 ) ,f (x_2) , \ldots,f (x_n ) ,\ldots$
都以定数$\ell$为它的极限,就说函数$f(x)$, 当$x\to +\infty$时,以$\ell$为极限,记作
\begin{equation}
    \lim_{x\to+\infty}f(x)=\ell\qquad \text{或}\qquad f(x)\to \ell\quad (x\to +\infty)
\end{equation}
\end{blk}
 
从几何上看,极限式(1.1)表示随着$x$无限增大,曲线$y=f(x)$以直线$\ell$为渐近线(图1.1)。
\begin{figure}[htp]
    \centering
    
    \caption{}
\end{figure}

类似地,可以定义函数极限$\Lim_{x\to -\infty} f(x)=\ell$, 这时,变量
$x$通过代数值无限地变小,而绝对值无限地增大的任何一个数列$\{x_n\}$。

如果函数$f(x)$当$x\to+\infty$和$x\to-\infty$时,都以定值为极限,就说$f(x)$当$x\to \infty$时,以定值$\ell$为极限,记作$\Lim_{x\to\infty}f(x)=\ell$, 或者$f(x)\to \ell\; (x\to\infty)$。

\begin{example}
    证明$\Lim_{x\to\infty}\frac{1}{x}=0$
\end{example}

\begin{proof}
任何数列$\{x_n\}$的值$x_1,x_2,\ldots,x_n,\ldots$趋向于$+\infty$或$-\infty$时,对应的函数数列
\[\frac{1}{x_1},\frac{1}{x_2},\ldots, \frac{1}{x_n},\ldots\]
的绝对值$\left|\frac{1}{x_n}\right|$便趋向于零,即
    \[\lim_{n\to\infty}\left|\frac{1}{x_n}\right|=0\]
    从而$\Lim_{x\to\infty}\frac{1}{x}=0$
\end{proof}

\begin{example}
证明函数$\sin x$, 当$x\to\infty$时,没有极限.
\end{example}

\begin{proof}
令自变量$x$取数列$x_n=-\frac{\pi}{2}+2n\pi\quad (n=1,2,3,\ldots)$的值趋向$+\infty$,则对应的函数值数列
\[\begin{split}
    \sin x_n&=\sin\left(-\frac{\pi}{2}+2n\pi\right)\\
&=\sin\left(-\frac{\pi}{2}\right) =-1 \qquad (n=1, 2, 3, \ldots) .
\end{split}\]
恒取定值$-1$, 于是
\[\lim_{n\to\infty} \sin x_n=\lim_{n\to\infty} \sin \left(-\frac{\pi}{2}+2n\pi\right)=-1\]

再令自变量$x$取数列
\[x_n=\frac{\pi}{2}+2n\pi\qquad  (n=1, 2, 3, \ldots )\]
的值趋向于$+\infty$, 则对应的函数值数列
\[\sin x_n =\sin\left(\frac{\pi}{2}+2n\pi\right) =1\qquad  (n=1, 2, 3, \ldots)\]
恒取定值1,于是
\[\lim_{n\to\infty} \sin x_n=\lim_{n\to\infty} \sin\left(\frac{\pi}{2}+2n\pi\right) =1\]
由于$x$取趋向$+\infty$的两个不同数列时,$y=\sin x$可以有不同的极限,因此
$\Lim_{x\to\infty} \sin x$不存在。
\end{proof}

\begin{blk}{定义2}
 设函数$f(x)$在点$a$附近有定义(但在$x=a$时,
可以没有定义),如果当自变量$x$不论通过怎样一个以$a$为极限但始终不等于$a$的数列$\{x_n\}$, 对应的函数值数列$f (x_1) ,f (x_2) , \ldots,f(x_n ) ,\ldots$
总有极限$\ell$, 就说:

当$x$趋近于$a$时,$f(x)$以$\ell$为极限,记作
\begin{equation}
   \lim_{x\to a}f(x)=\ell,\qquad \text{或}\qquad f(x)\to \ell\quad (x\to a) 
\end{equation}
\end{blk}

极限式(1.2)的几何意义如图1.2所示:当$x$无限地靠近$a$,但总不能等于$a$时,曲线$y=f(x)$上的点$(x,f(x))$无限地近$(a,\ell)$点.

\begin{figure}[htp]
    \centering
    
    \caption{}
\end{figure}

初学的人常常要问:为什么在定义中谈到$x$趋近于$a$时,要限制$x$始终不等于$a$呢?这是因为我们关心的是函数$f(x)$在$a$附近的变化趋势,它和函数$f(x)$在
$x=a$这一点的值没有什么必然关系,这也就是说,无论$f(x)$在点$a$取什么值甚至没有定义,都不影响在这一点的极限的存在和极限值。

\begin{example}
设$f(x)=\frac{3}{4}\cdot \frac{x^2-1}{x-1}$,$x\in(-\infty,1)\cup(1,+\infty)$,求$\Lim_{x\to 1}f(x)$
\end{example}

\begin{solution}
    $f(x)=\frac{3}{4}\cdot \frac{x^2-1}{x-1}$在$x=1$时无意义,因为那时
    分母就变成零,因此,这里没有函数值$f(1)$, 曲线$y=f(x)$也没有相应于横坐标为1的那个点,但是让$x$任意地趋近于1是完全可以的,若$x\ne 1$, 则有
  \[  f (x) =\frac{3}{4}\cdot  \frac{(x-1)(x+1)}{x-1}=\frac{3}{4}(x+1)\]
  因此,不论$x$通过怎样一个以1为极限的数列$\{x_n\}$, 对于相应的数列$\{f(x_n)\}$, 我们都有
  \[\lim_{x_n\to 1} f (x_n) =\frac{3}{4} (1+1) =\frac{3}{2}\]

    从几何上看,曲线$y=f(x)$除去点$\left(1,1\frac{1}{2}\right)$外是与直线$y=\frac{3}{4}(x+1)$一致的,唯独在那一点,曲线有个空隙,而在$x=1$的邻近的点只要充分接近于点1, 所对应的函数值与$\frac{3}{2}$的差的绝对值可以任意小,如图1.3。
\end{solution}
    
\begin{figure}[htp]
    \centering
\begin{tikzpicture}[>=latex]
    \draw[->](-3,0)--(5,0)node[right]{$x$};
    \draw[->](0,-1.5)--(0,4.5)node[right]{$y$};
\foreach \x in {1,2,3}
{
    \draw(0,\x)node[left]{$\x$}--(.1,\x);
    \draw(\x,0)node[below]{$\x$}--(\x,.1);
}
\node at (.25,-.25){$O$};
\draw[dashed](0,1.5)--(1,1.5)--(1,0);
\draw[domain=-2:4, samples=10, very thick]plot(\x, {0.75*(\x+1)});
\draw(1,1.5)[fill=white]  circle(1.5pt) node[right]{$\left(1,1\tfrac{1}{2}\right)$};
\node at (3.25,3)[right]{$y=\frac{3}{4}\cdot \frac{x^2-1}{x-1}$};
\end{tikzpicture}
    \caption{}
\end{figure}

\begin{example}
证明当$x\to 0$时,函数$f(x)=\sin\frac{1}{x}$没有极限。
\end{example}

\begin{proof}
函数$f(x)=\sin\frac{1}{x}$对于一切$x\ne 0$的值有定义,因此这个函数在点$x=0$的领域内有定义。

当$x$取数列$\{x_n\}=\left\{\frac{2}{(2n+1)\pi}\Big| n=1,2,3,\ldots\right\}$的值而趋于零时,数列$\left\{\frac{1}{x_n}\right\}$相应的值是
\[\frac{3\pi}{2},\frac{5\pi}{2},\frac{7\pi}{2},\ldots,(2n+1)\frac{\pi}{2},\ldots\]
此时数列$\left\{\sin\frac{1}{x_n}\right\}$便交替地取$-1$和$+1$这两个数值,换言之
\[\sin\frac{1}{x_n}=(-1)^n,\qquad n=1,2,3,\ldots\]
因此,当$n\to\infty$时,数列$\left\{\sin\frac{1}{x_n}\right\}$不趋于任何极限值。这就证明了当$x\to 0$时,函数$f(x)=\sin\frac{1}{x}$的极限不存在.
\end{proof}

函数的图象大致如图1.4所示。曲线关于原点对称,在包含原点的每一个对称邻域$(-\delta, \delta)$内,曲线$y=\sin\frac{1}{x}$
在原点的邻近作无数次振动,且曲线的振幅恒为1, 虽将原点的邻域的长缩小,也不能减少振动的次数。

\begin{figure}[htp]
    \centering
\begin{tikzpicture}[>=latex]
    \draw[->](-4,0)--(4,0)node[right]{$x$};
    \draw[->](0,-2)--(0,2)node[right]{$y$};
    \foreach \x in {1,-1}
    {
        \draw[dashed](-4,\x)--(4,\x)node[right]{$y=\x$};
    }
\draw[domain=-4:-.05, samples=1000, thick]plot(\x, {sin(180/pi/\x)});
\draw[domain=.05:4, samples=1000, thick]plot(\x, {sin(180/pi/\x)});
\foreach \x/\xtext in {1/1, .637/\frac{2}{\pi}}
{
    \draw(\x, 0)node[below]{$\xtext$}--(\x,.1);
}
\end{tikzpicture}    
    \caption{}
\end{figure}

上面是用数列的极限来说明函数的极限,其实我们也可以直接定义函数极限。

\begin{blk}{定义3}
  如果函数$f$在点$a$邻域上有定义(可能去掉点$a$本身),使得当$0<|x-a|<\delta$时,就有$|f(x)-\ell|<\varepsilon$, 那么就说$\ell$为当$x$趋近于$a$时,函数$f$在点$a$的极限值。
\end{blk}

我们对这个定义需要说明以下几点:
\begin{enumerate}
    \item 用定义3验证某数$\ell$是函数$f$在点$a$的极限的办法就是对于任给的$\varepsilon>0$, 要找到这样的正数$\delta$使得能够由不等式$|x-a|<\delta$推出不等式$|f(x)-\ell|<\varepsilon$,虽然$\varepsilon$是任意的正数,但是在找$\delta$的过程中,$\varepsilon$是固定不变的,$\delta$依赖于$\varepsilon$。
    \item 对于已给的$\varepsilon$, 只要证明有一个$\delta>0$存在就行.因为如果有一个$\delta$存在,把$\delta$再缩小一些,显然仍满足我们的要求。
    \item 不等式$|x-a|>0$只是说明$x\ne a$, 即把$x$等于$a$的情况去掉,这是因为我们关心的是函数$f$在点$a$附近的变化趋势,而和函数在$x=a$这点的值无关。
    \item 我们指出定义2和定义3是等价的.
\end{enumerate}

\begin{example}
用定义3证明$\Lim_{x\to 1}\frac{x^3-1}{x-1}=3$
\end{example}

\begin{proof}
    任给$\varepsilon>0$, 要找$\delta>0$, 使由$0<|x-1|<\delta$推出
    $\left|\frac{x^3-1}{x-1}-3\right|<\varepsilon$成立。

    当$x\ne 1$时,
\[\begin{split}
    \left|\frac{x^3-1}{x-1}-3\right|&=|(x^2+x+1)-3|\\
    &=|(x-1)(x+2)|=|x-1|\cdot |x-2|
\end{split}\]

要由$|x-1|\cdot |x+2|<\varepsilon$找$\delta$,显然,这里因子$|x+2|$引起了麻烦。为方便起见,先假定$0<|x-1|<1$,即取$\delta_1=1$,这样
\[0<|x-1|<1\quad\Rightarrow\quad |x+2|=|(x-1)+3|\le |x-1|+3<4\]因此,要使
\[\begin{cases}
    0<|x-1|<1\\
    |x-1|\cdot |x+2|<\varepsilon
\end{cases}\]
只须
\[\begin{cases}
    0<|x-1|<1\\
    4|x-1|<\varepsilon
\end{cases}\Rightarrow\quad \begin{cases}
    0<|x-1|<1\\
    |x-1|<\frac{\varepsilon}{4}
\end{cases}\]
由此可见,只须取$\delta=\min\left(1,\frac{\varepsilon}{4}\right)$,即取$\delta$为1与$\frac{\varepsilon}{4}$中的较小者。

$\therefore\quad $对于任意$\varepsilon>0$,取$\delta=\min\left(1,\frac{\varepsilon}{4}\right)$,则当$0<|x-1|<\delta$时,即有
\[\left|\frac{x^3-1}{x-1}-3\right|<\varepsilon\]
这也就证明了
\[\lim_{x\to 1}\frac{x^3-1}{x-1}=3\]
\end{proof}

\begin{example}
    证明$\Lim_{x\to a}\sqrt{x}=\sqrt{a}\quad (a>0)$
\end{example}

\begin{proof}
对于任意的$\varepsilon>0$, 我们必须找到一个$\delta>0$, 使得当$|x-a|<\delta$时,$|\sqrt{x}-\sqrt{a}|<\varepsilon$成立,因为
\[|\sqrt{x}-\sqrt{a}|=\frac{|x-a|}{\sqrt{x}+\sqrt{a}}<\frac{|x-a|}{\sqrt{a}}\]
所以要使$\frac{|x-a|}{\sqrt{a}}<\varepsilon$,只须$|x-a|<\sqrt{a}\varepsilon$

如果取$\delta=\sqrt{a}\varepsilon$,则$\frac{|x-a|}{\sqrt{a}}<\frac{\sqrt{a}\varepsilon}{\sqrt{a}}=\varepsilon$
因此,对于任意$\varepsilon>0$,取$\delta=\sqrt{a}\varepsilon$,则当$|x-a|<\delta$时,就有$|\sqrt{x}-\sqrt{a}|<\varepsilon$成立。这也就证明了
\[\lim_{x\to a}\sqrt{x}=\sqrt{a}\quad (a>0)\]
\end{proof}

有时虽然$f(x)$在某点的左边(或右边)没有定义,如图1.5中的$a,b$两点,我们也可以谈论$f$在点$a$或点$b$两点的极限,譬如对于所有小于$a$的数,$f$虽然没有定义,但是我们可以考察,当$x$从$a$的右侧趋近于$a$时,函数$f$的变化趋势,也就是考察$f$的单边极限是否存在。

\begin{figure}[htp]
    \centering
\begin{tikzpicture}[>=latex]
    \draw[->](-1,0)--(3,0)node[right]{$x$};
    \draw[->](0,-1)--(0,2.5)node[right]{$y$};
\draw[very thick](.5,1)to[bend right=-15]node[above]{$y=f(x)$} (2.5,1.5);
\draw(.5,1)--(.5,0)node[below]{$a$};
\draw(2.5,1.5)--(2.5,0)node[below]{$b$};
\node at (-.25,-.25){$O$};
\end{tikzpicture}
    \caption{}
\end{figure}

\begin{blk}{定义}
     设$f(x)$在区间$(a,b)$上有意义,如果任给$\varepsilon>0$, 总存在某个$\delta>0$使得当$x\in (a,a+\delta)$时,总有$|f(x)-\ell|<\varepsilon$,我们就说函数$f$在点$a$以$\ell$为\textbf{右极限},记作:
\[\lim_{x\to a^+}f(x)=\ell\]
\end{blk}

类似地,可以定义\textbf{左极限},只需把开区间$(a,a+\delta)$换成$(b-\delta,b)$就行了,并记作
\[\lim_{x\to b^-} f(x)=\ell\]

\begin{example}
    设函数$f(x)=\begin{cases}
        x-1,&x\le 1\\
x+1,&x>1
    \end{cases}\quad $    
    求$\Lim_{x\to 1^-}f(x)$和$\Lim_{x\to 1^+}f(x)$
\end{example}

\begin{figure}[htp]
    \centering
    \begin{minipage}[t]{0.48\textwidth}
    \centering
    \begin{tikzpicture}[>=stealth, scale=.6]
        \draw[->](-3,0)--(4,0)node[right]{$x$};
        \draw[->](0,-3)--(0,4.5)node[right]{$y$};
        \draw(0,1)node[left]{1}--(.1,1);
        \draw(0,2)node[left]{2}--(.1,2);
        \draw(1,0)node[below]{1}--(1,.1);
        \node at (-.25,-.25){$O$};
        \draw[domain=-2:1, samples=10, very thick]plot(\x,{\x-1});
        \draw[domain=1:3, samples=10, very thick]plot(\x,{\x+1});
    \node at (-2,-3)[left]{$y=x-1$};
    \node at (3,4)[right]{$y=x+1$};
    \draw[dashed](1,0)--(1,2)--(0,2);
    \draw (1,2)[fill=white] circle(2.5pt);
    
    \end{tikzpicture}
    \caption{}
    \end{minipage}
    \begin{minipage}[t]{0.48\textwidth}
    \centering
    \begin{tikzpicture}[>=stealth, scale=.8]
        \draw[->](-3,0)--(4,0)node[right]{$x$};
    \draw[->](0,-2.5)--(0,3)node[right]{$y$};
    \foreach \x in {-2,-1,1,2,3}
    {
        \draw(\x,0)node[below]{$\x$}--(\x,.1);
    }
\foreach \x in {-2,-1,1,2}
{
    \draw(0,\x)--(.1,\x)node[right]{$\x$};
}
\foreach \x in {-2,-1,...,2}
{
    \draw[very thick](\x,\x)--(\x+1,\x);
    \draw (\x+1,\x)[fill=white]circle(2pt);
}
    \node at (.25,-.25){$O$};
    \end{tikzpicture}
    \caption{}
    \end{minipage}
    \end{figure}


\begin{solution}
\[\begin{split}
    \lim_{x\to 1^-}f(x)&=\lim_{x\to 1^-}(x-1)=0\\
    \lim_{x\to 1^+}f(x)&=\lim_{x\to 1^+}(x+1)=2
\end{split}\]
图1.6显示了上面的结果。
\end{solution}

\begin{example}
    函数$y=[x]$代表不超过$x$的最大整数,即若$n\le x<n+1$,$n\in\mathbb{Z}$,则$y=[x]=n$。(图1.7)

求$\Lim_{x\to 2^+}\frac{[x]}{x},\quad \Lim_{x\to 2^-}\frac{[x]}{x}$
\end{example}

\begin{solution}
    显然,当$x=2$时,$\frac{[x]}{x}=\frac{[2]}{2}=\frac{2}{2}=1$, 又
\[\frac{[x]}{x}=\begin{cases}
    \frac{1}{x},& x\in(1,2)\\
    \frac{2}{x},& x\in(2,3)
\end{cases}\]
所以
\[\begin{split}
    \lim_{x\to 2^+}\frac{[x]}{x}&=\lim_{x\to 2^+}\frac{2}{x}=1\\
     \lim_{x\to 2^-}\frac{[x]}{x}&=\lim_{x\to 2^-}\frac{1}{x}=\frac{1}{2}
\end{split}\]    
\end{solution}


下面的命题说明函数的极限与函数的单边极限的关系:

\begin{blk}{定理}
极限$\Lim_{x\to a} f(x)$
存在的必要和充分条件是左极限$\Lim_{x\to a^-} f(x)$和右极限$\Lim_{x\to a^+} f(x)$都存在,并且二者相等。
\end{blk}

\begin{proof}
    必要性
    
    如果$\Lim_{x\to a} f(x)=\ell$, 就是说任给$\varepsilon>0$, 总存在$\delta>0$,
使得当$0<|x-a|<\delta$, 即当$x\in (a-\delta,a)\cup(a,a+\delta)$时,有$|f(x)-\ell|<\varepsilon$。换言之,当$x\in (a-\delta,a)$和$x\in (a,a+\delta)$时,都有$|f (x) -\ell|<\varepsilon$,因此
\[\Lim_{x\to a^-} f(x)=\ell,\qquad \Lim_{x\to a^+} f(x)=\ell\]

充分性

如果$\Lim_{x\to a^-} f(x)=\ell$且$\Lim_{x\to a^+} f(x)=\ell$, 那么总存在$\delta_1>0$, 使得当$x\in(a-\delta_1,a)$时,有
$|f (x) -\ell|<\varepsilon$。

又存在$\delta_2>0$, 使得当$x\in (a,a+\delta_2)$时,有$|f (x) -\ell|<\varepsilon$。

取$\delta=\min(\delta_1,\delta_2)$,于是当$x\in(a-\delta,a+\delta)$时,有$|f (x) -\ell|<\varepsilon$。
这就是说:
\[\Lim_{x\to a} f(x)=\ell\]
\end{proof}

\begin{example}
说明$\Lim_{x\to 3}\frac{|x-3|}{x-3}$是否存在?
\end{example}

\begin{solution}
\[\begin{split}|x-3|&=\begin{cases}
    x-3, & x>3\\
    3-x, &x<3
\end{cases}\\
    \Lim_{x\to 3^-}\frac{|x-3|}{x-3}&=\Lim_{x\to 3^-}\frac{3-x}{x-3}=\Lim_{x\to 3^-}(-1)=-1\\
    \Lim_{x\to 3^+}\frac{|x-3|}{x-3}&=\Lim_{x\to 3^+}\frac{x-3}{x-3}=\Lim_{x\to 3^+}(1)=1
\end{split}\]

$\because\quad \Lim_{x\to 3^-}\frac{|x-3|}{x-3}\ne \Lim_{x\to 3^+}\frac{|x-3|}{x-3}$
    
$\therefore\quad \Lim_{x\to 3}\frac{|x-3|}{x-3}$
不存在。
\end{solution}

\subsection{函数值趋于无穷大}

如果函数$f$在点$a$的邻域上有定义(可能去掉点$a$本身)对于无论多么大的正数$G$, 总存在一个够小的正数$\delta$, 使得当$0<|x-a|<\delta$时,就有$|f(x)|>G$, 那么就说当$x$趋于$a$时,函数$f(x)$趋于\textbf{无穷大},记作
\[\lim_{x\to a} f (x) =\infty\]

\begin{example}
    求证$\Lim_{x\to 0}\frac{1}{x}=\infty$
\end{example}
   
\begin{proof}
设$G$是任意给定的正数,我们要求出一个$\delta>0$, 使得当$|x|<\delta$时,$|f(x)|=\left|\frac{1}{x}\right|>G$。

事实上,要使$\frac{1}{|x|}>G$, 
只须$0<|x|<\frac{1}{G} $。取$\delta=\frac{1}{G}$, 于是当$|x|<\delta$时,就有$\frac{1}{|x|}>G$
因此,
$$\Lim_{x\to 0}\frac{1}{x}=\infty$$
\end{proof}

\begin{example}
    证明当$x\to 0$时,函数$f(x)=\frac{1}{x}\sin\frac{1}{x}\quad (x\ne 0)$不趋于无穷大.
\end{example}

\begin{proof}
如果自变量$x$取数列$\{x_n\}=\left\{\frac{2}{(2n+1)\pi}\Big|n=1, 2, 3,\ldots\right\}$的值趋于0时,$\sin\frac{1}{x}$在原点的任意邻域内无限次交替地取$-1, 1$这两个值,对于这些值,$|f(x)|=\frac{1}{x}=\frac{\pi}{2}(2n+1)$趋于无穷大.

但是当$x$取数列$\{x\}=\left\{\frac{1}{n\pi}\Big| n=1, 2, 3,\ldots\right\}$的
值趋于0时,由于
$\sin\frac{1}{x}=\sin (n\pi) =0$, 
故对于这些值,
$\Lim_{x'_n\to 0} f (x) =0$.

可见在原点的邻近不存在这样的$\delta>0$, 使得当$|x|<\delta$时,$|f(x)|>G$, 因此,当$x\to 0$时,$f(x)=\frac{1}{x}\sin\frac{1}{x}\quad (x\ne 0)$不趋于无穷大.

$y=f(x)$的图象位于两条双曲线$xy=\pm 1$之间,且在原点的邻近作无限多次振动,越靠近原点,曲线的振幅越大(图1.8).

\begin{figure}[htp]
    \centering
\begin{tikzpicture}[>=latex, scale=.8]
\draw[->](-5,0)--(5,0)node[right]{$x$};
\draw[->](0,-4)--(0,4)node[right]{$y$};
\draw[domain=-4.5:-.3, samples=100]plot(\x,{-1/\x}); 
\draw[domain=-4.5:-.3, samples=100]plot(\x,{1/\x}); 
\draw[domain=.3:4.5, samples=100]plot(\x,{-1/\x}); 
\draw[domain=.3:4.5, samples=100]plot(\x,{1/\x}); 
\draw [domain=.5:4.5, samples=200,  thick]plot(\x, {sin(180/\x*pi)/\x});
\draw [domain=.5:4.5, samples=200,  thick]plot(-\x, {sin(180/\x*pi)/\x});
\end{tikzpicture}
    \caption{}
\end{figure}

\end{proof}

如果对于任何$G>0$, 存在$\delta>0$, 当$0<|x-a|<\delta$时,有$f(x)>G$, 就说当$x\to a$时,函数$f(x)$趋于正无无穷大,记作
\[\lim_{x\to a}f(x)=+\infty\]

如果对于任何$G>0$, 存在$\delta>0$, 当$0<|x-a|<\delta$时,有$f(x)<-G$, 就说当$x\to a$时,函数$f(x)$趋于负无穷大,记作
\[\lim_{x\to a}f(x)=-\infty\]

例如,我们有:
\[\lim_{x\to 0}\frac{1}{x^2}=+\infty,\qquad \lim_{x\to 0}\frac{(-1)}{x^2}=-\infty\]

类似地,我们可以定义:
\[\lim_{x\to a^-}f(x)=+\infty,\quad \lim_{x\to a^+}f(x)=+\infty,\quad \lim_{x\to a^-}f(x)=-\infty,\quad \lim_{x\to a^+}f(x)=-\infty\]
的含义,这里不再写出。建议读者将这些定义严格地写出来。

例如,我们有:
\[\lim_{\theta\to \tfrac{\pi^+}{2}}\tan\theta =+\infty,\qquad \lim_{\theta\to \tfrac{\pi^-}{2}}\tan\theta =-\infty\]
\[\lim_{x\to 0^+}\log_a x =-\infty\quad (a>1),\qquad \lim_{x\to 0^+}\log_a x =+\infty\quad (0<a<1)\]

\begin{ex}
\begin{enumerate}
    \item 用函数极限定义证明:
    \begin{multicols}{2}
        \begin{enumerate}
            \item $\Lim_{x\to \infty}\frac{1}{2x+1}=0$
            \item $\Lim_{x\to 2}x^2=4$
            \item $\Lim_{x\to -1}\frac{x-3}{x^2-9}=\frac{1}{2}$
            \item $\Lim_{x\to 1}\frac{1}{x^2}=1$
            \item $\Lim_{x\to 1}\frac{x^3-x}{x-1}=2$
        \end{enumerate}
    \end{multicols}
    \item 说明:$\Lim_{x\to 3}\frac{x}{x^2-9}=\infty$
    \item 下面极限是否存在?
\begin{multicols}{2}
    \begin{enumerate}
        \item $\Lim_{x\to 1}\frac{2x|x-1|}{x-1}$
        \item $\Lim_{x\to 3}\frac{[x]^2-9}{x^2-9}$
    \end{enumerate}
\end{multicols}
\end{enumerate}
\end{ex}

\subsection{函数极限算法定理}
函数的极限算法定理与数列的极限算法定理类似,因为所谓$\Lim_{x\to a}u(x)=A$, $\Lim_{x\to a}v(x)=B$的意思就是对于任何一个各项都不同于$a$并且以$a$为极限的数列$x_n\to a$, 便有函数值数列$\{u(x_n)\}$, $\{v(x_n)\}$, 并且$\Lim_{n\to \infty}u(x_n)=A$, 
$\Lim_{n\to \infty} v(x_n)=B$, 因此,根据第四册下第三章的定理就可以直接得到相应的结果。现在给出函数的极限运算定理如下:

\begin{blk}{定理}
设$\Lim_{x\to a}u(x)=A$, $\Lim_{x\to a}v(x)=B$,那么
\begin{enumerate}
    \item $\Lim_{x\to a}[u(x)+v(x)]=\Lim_{x\to a}u(x)+\Lim_{x\to a}v(x)$
    \item $\Lim_{x\to a}[u(x)\cdot v(x)]=\Lim_{x\to a} u(x)\cdot \Lim_{x\to a}v(x)$
    \item $\Lim_{x\to a}[c\cdot u(x)]=c\cdot \Lim_{x\to a}u(x)$
    \item $\Lim_{x\to a}\frac{u(x)}{v(x)}=\frac{\Lim_{x\to a}u(x)}{\Lim_{x\to a}v(x)}$ 
    
    只要$v(x)$恒不为0,而且$\Lim_{x\to a}v(x)\ne 0$
    \item $\Lim_{x\to a}\sqrt[n]{u(x)}=\sqrt[n]{\Lim_{x\to a}u(x)}$
    \item 如$\Lim_{x\to a}u(x)=A=\Lim_{x\to a} v(x)$,而$|x-a|<\delta$时,$u(x)<f(x)<v(x)$,则:
    \[\Lim_{x\to a} f(x)=A\]
    即:如果$u(x)$与$v(x)$趋向同一极限$A$,且$f(x)$在$u(x)$与$v(x)$之间,那么,$f(x)$便也趋向那个极限$A$。
\end{enumerate}
\end{blk}

现在只有5需要补证.

设$\Lim_{x\to a}u(x)=A>0$, 则根据函数极限定义:对于$\varepsilon=\frac{A}{2}>0$, 存在$\delta>0$,使得$0<|x-a|<\delta$时,有
\begin{equation}
    u(x)>\frac{A}{2}>0
\end{equation}

于是,
\[\sqrt[n]{u(x)}-\sqrt[n]{A}=\frac{u(x)-A}{\sum^n_{k=1}[u(x)]^{\tfrac{n-k}{n}}A^{\tfrac{k-1}{n}}}\]
由(1.3)式
\[[u(x)]^{\tfrac{n-k}{n}}>\left(\frac{A}{2}\right)^{\tfrac{n-k}{n}}\]
故
\[\begin{split}
    \sum^n_{k=1}[u(x)]^{\tfrac{n-k}{n}}A^{\tfrac{k-1}{n}}&>\sum^n_{k=1}\left(\frac{A}{2}\right)^{\tfrac{n-k}{n}}A^{\tfrac{k-1}{n}}\\
    &=A^{\tfrac{n-1}{n}}\sum^n_{k=1}\left(\frac{1}{2}\right)^{\tfrac{n-k}{n}}\\
    &=A^{\tfrac{n-1}{n}}\left\{\left(\frac{1}{2}\right)^{\tfrac{n-1}{n}}+\left(\frac{1}{2}\right)^{\tfrac{n-2}{n}}+\cdots+\left(\frac{1}{2}\right)^{\tfrac{1}{n}}+1\right\}\\
    &=A^{\tfrac{n-1}{n}}\left\{\frac{1-\left(\frac{1}{2}\right)^{\tfrac{n-1}{n}}\cdot \left(\frac{1}{2}\right)^{\tfrac{1}{n}}}{1-\left(\frac{1}{2}\right)^{\tfrac{1}{n}}}\right\}\\
    &=A^{\tfrac{n-1}{n}}\cdot\frac{\frac{1}{2}}{1-\left(\frac{1}{2}\right)^{\tfrac{1}{n}}}=\frac{A^{\tfrac{n-1}{n}}}{2-2^{\tfrac{n-1}{n}}}\\
\end{split}\]
$\therefore\quad \left|\sqrt[n]{u(x)}-\sqrt[n]{A}\right|<|u(x)-A|\cdot \frac{2-2^{\tfrac{n-1}{n}}}{A^{\tfrac{n-1}{n}}}$

由题设$\Lim_{x\to a}u(x)=A$,而$\frac{2-2^{\tfrac{n-1}{n}}}{A^{\tfrac{n-1}{n}}}$是一个与$x$无关的常数,所以
\[\lim_{x\to a} \sqrt[n]{u(x)}=\sqrt[n]{A}=\sqrt[n]{\lim_{x\to a}u(x)}\]

下面我们来证明两个重要极限公式,为此先介绍一个引理。

\begin{blk}{引理}
     对于任意实数$\theta$, 都有$|\sin\theta|\le |\theta|$
\end{blk}

\begin{proof}
  由于对于任意实数。有$|\sin\theta|\le 1$, 那么仅考虑$\theta\in (0,\pi/2)$即可。
  
  在单位圆$O$上(图1.9),
 截取弧$\wideparen{P_0M}$使弧长$|\wideparen{P_0M}|$等于$\theta$, 其中$P_0$和$M$分别有坐标$(1, 0)$和$(x,y)$, 于是
 \[\sin\theta =y<\sqrt{y^2+ (1-x)^2} =|P_0M|<|\wideparen{P_0M}|=\theta\]
 
 显然,当$\theta=0$时,有$\sin\theta=\theta$. 因此
\[  |\sin\theta | \le |\theta|,\qquad  \theta\in  (0,\pi/2)\]

  如果$-\pi/2<\theta<0$, 则仍有$|MN|<\wideparen{P_0M}$,于是
\[|\sin\theta|<|\theta|,\qquad \theta\in (-\pi/2, 0) \]

  如果$|\theta|\ge \pi/2$, 则因为$\pi/2>1$与$|\sin\theta|\le 1$, 而同样地也得到$|\sin\theta|<|\theta|$. 
  
  因此,对于一切$\theta\ne 0$的值,有$|\sin\theta|<|\theta|$; 当$\theta=0$时,有$|\sin\theta|=|\theta|$。
\end{proof}



\begin{figure}[htp]\centering
    \begin{minipage}[t]{0.48\textwidth}
    \centering
\begin{tikzpicture}[>=latex]
\draw[->](-2.5,0)--(2.5,0)node[right]{$x$};
\draw[->](0,-2)--(0,2.5)node[right]{$y$};
\draw(0,0) circle(1.5);
\draw[very thick](0,0)--(45:1.5)node[above right]{$M(x,y)$}--(1.5,0)node[below right]{$P_0(1,0)$};
\draw[dashed](1.5/1.414,0)node[below]{$N$}--(1.5/1.414,1.5/1.414);
\node at (0,0)[below left]{$O$};
\end{tikzpicture}    
    \caption{}
    \end{minipage}
    \begin{minipage}[t]{0.48\textwidth}
    \centering
    \begin{tikzpicture}[>=latex, scale=1.3]

\draw[->](-1.5,0)--(2.5,0)node[right]{$x$};
\draw[->](0,-1.5)--(0,2)node[right]{$y$};
\draw(0,0) circle(1);
\draw[very thick](0,0)--(60:2)node[above right]{$C$}--(1,0)node[below right]{$A(1,0)$};
\draw[thick](.5,0)node[below]{$B$}--(60:1)node[above]{$D$};
\node at (0,0)[below left]{$O$};   
\draw(0.2,0) arc (0:60:.2)node[right]{$\theta$};   
    \end{tikzpicture}
    \caption{}
    \end{minipage}
    \end{figure}

\begin{blk}{定理}
\[\lim_{\theta\to 0}\frac{\sin\theta }{\theta}=1,\qquad \lim_{\theta\to 0}\frac{\cos\theta-1}{\theta}=0\]
\end{blk}

\begin{proof}
    让我们先证明$\Lim_{\theta\to 0}\frac{\sin\theta }{\theta}=1$。由图1.10容易看出:

假定$\theta$的单位是弧度,于是:
\[\begin{split}
    \triangle OBD\text{的面积}&=\frac{1}{2}OB\cdot BD=\frac{1}{2}\cos\theta\cdot \sin\theta\\
    \triangle OAC\text{的面积}&=\frac{1}{2}OA\cdot AC=\frac{1}{2}\cdot 1\cdot \tan\theta\\
    \text{扇形$OAD$的面积}&=\frac{1}{2}OA\cdot \wideparen{AD}=\frac{1}{2}\cdot 1^2\cdot \theta
\end{split}\]



\end{proof}








 
\chapter{变率和微商}
从本章起,我们开始学习单变量的微积分学的基本概念和基础理论。

微积分学是研究变量的数学,变量之间的关系就是函数,因此,函数是微积分学研究的主要对象。

在函数的基本性质中,有两个最基本、最重要的概念-变率与求和,为了解决求函数的变率与求函数$f$在$[a,b]$上的和的问题就相应地产生微分与积分运算,而这两种运算之间,也存在着一种自然的互逆关系,在本章中,我们由函数的变率问题引出函数的微商(导数)概念,并给出初等函数的一套求导法则,在下一章中,我们揭示微分运算与积分运算的互逆关系,这就是微积分学的基本定理。

总起来说,微分反映了函数的局部性质,或在某个点附近的性质;积分则反映了函数的整体性质,或某个区间的性质;函数的局部性质与整体性质之间的有机联系,恰恰反映了微分运算与积分运算之间的互逆关系。

\section{微商(导数)的定义}
函数关系$y=f(x)$就是确定变量$y$如何随着变量$x$的变动而变动的关系,对于给定的函数$y=f(x)$, 变量$y$在变量$x$的不同点附近的变动情况是不尽相同的,这就是说,在变量$x$的
某个值$x_1$外,当$x_1$略加变动时,相应的$y$的变动可能相当剧烈(急增,或急减);而在变量$x$的另一个值$x_2$处,$y$的变动就可能较为迟缓,但是,用这样的语言来表达函数在某一点处的变率是不精确的,我们需要用“数量”来确切地表达这个意思,这就是函数在某点(或菜时刻)的变率的问题,简称变率。

\subsection{直线函数的变率}
一次函数$f (x) =kx+b$
是一种最简单的函数,它的函数图象是一条斜率等于$k$的直线,如图2.1所示。
\begin{figure}[htp]
    \centering
\begin{tikzpicture}[>=latex]
    \draw[->](-2,0)--(4,0)node[right]{$x$};
    \draw[->] (0,-1)--(0,5)node[right]{$y$};
    \draw[very thick] (-1.5,-.5)--(4,5);
\draw[->](-.2,0) arc (0:45:.8)node[below]{$\theta$};
\draw(2,3)node[left]{$P_1$}--(2,0)node[below]{$x_1$};
\draw(3,4)node[left]{$P$}--(3,0)node[below]{$x$};
\draw[|<->|](3.2,4)--node[right]{$f(x)-f(x_1)$}(3.2,3);
\draw[|<->|](2,2.8)--node[below]{$x-x_1$}(3,2.8);
\draw(2,3)--(3,3);\node at (.25,-.25){$O$};
\end{tikzpicture}
    \caption{}
\end{figure}

设$x_1$, $f(x_1)$和$x$, $f(x)$分别是直线上点$P_1$和$P$的坐标,为了反映函数变化快慢的问题,无论$x<x_1$还是$x>x_1$,自然地考虑在点$x_1$邻近,函数与自变量的相应的改变量的比:
\[\frac{f(x)-f(x_1)}{x-x_1}=\frac{(kx+b)-(kx_1+b)}{x-x_1}=k=\tan\theta\]
上面的表达式称为函数的\textbf{差商},它表示函数在区间$[x_1,x]$或$[x,x_1]$上对于自变量$x$的\textbf{平均变化率}。由于$k$是不随变量$x$变动而变动的常数,因此,一次函数在自变量的任何一个区间内的平均变化率都是常数。

如果让自变量的变化区间的长度无限地缩短,也就是让$x$无限地接近于$x_1$时,平均变化率所趋向的极限
\[\lim_{x\to x_1} \frac{f (x) -f (x_1)}{x-x_1} =k\]

\subsection{平滑曲线的切线与变率}
一般的函数$y=f(x)$的图象通常不是直线,由于函数
和它的图象的多样性,为了讨论的方便起见,我们先把讨论的范围限制在“平滑”的曲线上,常用的函数$y=f(x)$的图象往往是“平滑的”,平滑性的直观内涵是:用愈高倍的显微镜去观察曲线的微段,就愈象直线段,比较明确的几何说法是:一条曲线在$P$点的平滑性就是存在唯一的一条过$P$点的切线,它无限地逼近曲线在$P$点邻近的微段,于是,当$y=f(x)$的图象$C$在$P$点存在唯一的一条切线时,我们就说曲线$C$在$P$点平滑,而$P$点叫做曲线的平滑点,一条在每点都平滑的曲线叫做平滑的曲线,一个函数的图象平滑曲线时,我们就称这种函数为平滑函数。

同学可能会问这样一个问题:在曲线的点$P$存在唯一的一条切线的含义是什么?因为迄今我们对于一般的曲线的切
线还未下过定义呢!

我们从图2.2和2.3注意到不能把切线定义为与曲线只有一个交点的直线。这样的定义限制得既太紧同时又太松。因
为,照此定义,图2.2所示的直线就不是过曲线上$P$点的切线了,实际上,尽管图2.2的直线与曲线还有一个交点$Q$, 但它在曲
线$P$点邻近却与曲线密合,故仍应该是过曲线$P$点的切线;又图2.3表明过抛物线上任何一点$P$与$y$轴平行的直线虽然与抛物线只有一个交点,但它的其余部分却远离$P$点邻近的弧,故它不应该是抛物线的切线,定义切线的可行途径是从割线开始,并应用极限的概念。


如图2.4所示,取曲线$y=f(x)$上点$P$附近的另一点$Q$, 通过这两点画一条直线,这直线叫做过曲线上$P$点的割线,让$Q$点沿曲线向点$P$移动,这条割线将达到极限位置,此极限位置与$Q$点从哪一侧趋向于$P$是无关的,我们称这个割线的极限位置为过曲线上$P$点的切线。

割线的这种极限位置的存在性这一假设,与曲线在点$P$具有唯一的一条切线或确定的方向的假设是等价的。

现在我们要对曲线$y=f(x)$用解析式子把割线的这种极限位置存在的过程表示出来。
设$\alpha$是割线$PQ$同正$x$轴构成的夹角,$\alpha_1$是过点$P$点的切线同正$x$轴构成的夹角,于是
\[\lim_{Q\to P}\alpha=\alpha_1\]
设$x_1,y_1$和$x,y$分别是点$P$和$Q$的坐标,这时,我们立即得到
\[\tan\alpha=\frac{y-y_1}{x-x_1}=\frac{f(x)-f(x_1)}{x-x_1}\]
因此,上述求极限的过程(不考虑垂直切线$\alpha_1=\frac{\pi}{2}$的情况)可由下式来表示:
\[\lim_{x\to x_1}\frac{f(x)-f(x_1)}{x-x_1}=\lim_{\alpha\to \alpha_1}\tan\alpha=\tan\alpha_1\]
这就是说过曲线$y=f(x)$上$P(x_1,y_1)$点的切线的斜率等于$y=f(x)$的差商当$x\to x_1$时的极限.

\begin{example}
    求抛物线$y=ax^2+bx+c$在$x_0$处的切线的斜率。
\end{example}

\begin{solution}
    解依题意$(x_0,f(x_0)=ax_0^2+bx_0+c)$在抛物线上,并设$(x_0+h,f(x_0+h))$是抛物线上点$(x_0,f(x_0))$的附近的一点,我们有
\[\begin{split}
&\qquad \lim_{h\to 0}\frac{f(x_0+h)-f(x_0)}{(x_0+h)-x_0}\\
&=\lim_{h\to 0}\frac{[a(x_0+h)^2+b(x_0+h)+c]-[ax^2_0+bx_0+c]}{h}    \\
&=\lim_{h\to 0}\frac{(2ax_0+b)h+h^2}{h}\\
&=\lim_{h\to 0}[(2ax_0+b)+h]=2ax_0+b
\end{split}\]
所以抛物线$y=ax^2+bx+c$在$x_0$处的切线的斜率是$2ax_0+b$.
\end{solution}

\begin{example}
    一质点沿一直线在$t$秒内移动的距离是$s=s(t)=t^2+4t$。
    
    求:质点的初速度;在两秒末的速度;前两秒内的平均速度。
\end{example}

\begin{analyze}
    如果质点从起点开始所走的距离$s$是时间$t$的线性函数,则由2.1知道该质点在每一时刻的速度都是常数,它的大小由平均速度来确定,即等于一次函数的斜率,此时我们说该质点作匀速运动,但是,如果运动不再是匀速的,即质点的速度每时每刻都是变的,那么我们将时刻$t$的速度(也叫做瞬时速度)理解成什么呢?为了回答这个问题,我们考察差商
\[\frac{\Delta s}{\Delta t}=\frac{s(t)-s(t_0)}{t-t_0}\]
或者写成\[\frac{s(t_0+\Delta t)-s(t_0)}{\Delta t}\]
这个差商称为在$t_0$和$t_0+\Delta t$之间的这段时间间隔上的质点的平均速度,对照着$s=s(t)=t^2+4t$的图象来看,这个平均速度也就是过曲线上的$P(t_0,s(t_0))$
点及它邻近一点$Q(t_0+\Delta t,s(t_0+\Delta t))$的割线的斜率(图2.5)。当$\Delta t$ 很小时,可以认为,从时刻$t_0$到$t_0+\Delta t$这段时间内,速度来不及有很大变化,可以近似地看成匀速运动,因而这段时间内的平均速度就可以看成时刻$t_0$的瞬时速度的近似值。

\begin{figure}[htp]
    \centering
    \begin{tikzpicture}[>=latex, scale=.5]
\draw[->](-5.5,0)--(3,0)node[right]{$t$};
\draw[->](0,-5)--(0,7)node[right]{$s$};

\draw[domain=-5.25:0, samples=300, dashed]plot(\x, {\x*\x+4*\x});
\draw[domain=0:1.25, samples=100, thick]plot(\x, {\x*\x+4*\x});
\foreach \x in {-4,-3,-2,-1,1,2}
{
    \draw(\x,0)--(\x,.2);
}
\node at (-2,0)[below]{$-2$};
\node at (1,0)[below]{$1$};
\node at (0,0)[below left]{$O$};
    \end{tikzpicture}

    \caption{}
\end{figure}



显然,从时刻$t_0$到时刻$t_0+\Delta t$, 质点走过的路程为
\[\begin{split}
    \Delta s&=s(t_0+\Delta t)-s(t_0)\\
    &=(t_0+\Delta t)^2+4(t_0+\Delta t)+(t^2_0+4t_0)\\
    &=(2t_0+4)\Delta t+(\Delta t)^2
\end{split}\]
所以这段时间内的平均速度为
\[\begin{split}
    \frac{\Delta s}{\Delta t}&=\frac{s(t_0+\Delta t)-s(t_0)}{\Delta t}\\
    &=\frac{(2t_0+4)\Delta t+(\Delta t)^2}{\Delta t}\\
    &=(2t_0+4)+\Delta t
\end{split}\]
$\Delta t$越小,这个平均速度就越接近时刻$t_0$的瞬时速度$v_0$, 我们自然令$\Delta t\to 0$, 求差商的极限值,得到
\[\begin{split}
    \lim_{\Delta t\to 0}\frac{s(t_0+\Delta t)-s(t_0)}{\Delta t}&= \lim_{\Delta t\to 0}[(2t_0+4)+\Delta t]\\
    &=2t_0+4
\end{split}\]
这样平均速度$\frac{s(t_0+\Delta t)-s(t_0)}{\Delta t}$,当$\Delta t\to 0$时的极限值
就表达了质点在时刻$t_0$的瞬时速度,把它记作
\[s'(t_0)=   \lim_{\Delta t\to 0}\frac{s(t_0+\Delta t)-s(t_0)}{\Delta t}\]
\end{analyze}

\begin{solution}
\begin{enumerate}
\item 质点的初速度
\[s'(0)=2\x0+4=4(\ms)\]
\item 质点在两秒末的速度
\[s'(2)=2\x2+4=8(\ms)\]
\item \[\text{质点在前两秒内的平均速度}=\frac{\text{在前两秒内所走距离}}{\text{时间}}=\frac{2^2+4\x2}{2}
=6(\ms)\]
\end{enumerate}
\end{solution}

从这个问题可以看出质点在$t_0$时刻的瞬时速度$s'(t_0)$的几何意义就是曲线$s=s(t)$在$P(t_0,s(t_0))$点的切线的斜率,所以函数在某点的变率有确定值与函数的图象在该点有唯一的一条切线是两个密切相关的概念。

\subsection{微商(导数)的定义}
从上面所举的两个例子来看,问题来自不同的领域:
\begin{enumerate}
    \item 求过曲线上一点的切线,
    \item 求函数在某点的变率.
\end{enumerate}
但解决的方法却完全一样,就是计算函数的差商的极限,这种极限反映了自然界中很多不同现象在量方面的共性,因此有必要从这些具体问题中把它抽象出来加以研究,再反过来去解决这类具体问题。

\begin{blk}{定义}
    设$y=f(x)$是定义在闭区间$[a,b]$上的一个函数,$x_0\in (a,b)$, 如果极限
\[\lim_{\Delta x\to 0}\frac{f (x_0+\Delta x) -f (x_0)}{\Delta x}\]
存在,我们就说$f(x)$在点$x_0$处\textbf{可微},并称这极限为函数$f(x)$在$x_0$点的\textbf{微商}(或导数),记为$f'(x_0)$或
$\frac{\dd y}{\dd x}\Big|_{x=x_0}$.
\end{blk}

显然,$f'(x_0)$的值与点$x_0$有关,当点$x_0$在开区间$(a,b)$内变化时,$f'(x_0)$也将跟着变化,因此,如果函数$f(x)$在开区间$(a,b)$内每点都可微(即存在有限导数),那么$f'(x)$便是一个新的函数,称为$f(x)$的\textbf{导函数}。

求已知函数的导函数$f'(x)$的运算,称为微商 运算,计算过程如下:
\begin{enumerate}
\item 设$\Delta x$为自变量某个值$x$的改变量:
\item 计算$f(x)$在点$x$的相应改变量
\[\Delta y=f (x+\Delta x) -f (x) \]
\item 计算$f(x)$在点$x$的差商
\[\frac{\Delta y}{\Delta x}=\frac{f (x+\Delta x) -f (x)}{\Delta x}\]
\item 计算
\[\lim_{\Delta x\to 0}\frac{f (x+\Delta x) -f (x)}{\Delta x}=f'(x)=\frac{\dd y}{\dd x}\]
\end{enumerate}

应当注意,这里$\frac{\dd y}{\dd x}$是一个独立的记号,它表示函数$f(x)$
在点$x$的导数,不能把它当成一个分数来看待,必须把它看成一个整体。

从微商的定义可以看出:
\begin{enumerate}
\item 曲线在一点的切线的斜率,就是函数在这一点的变率(微商或导数)。    
\item 微商所涉及的是函数的“局部”性质,也就是说,函数$y=f(x)$在一点$x_0$处是否可微只与函数$y=f(x)$在$x=x_0$处及其近旁的性质有关,而与其它地方无关。    
\item 如果$f(x)$在点$x$可微,按照极限存在的条件,必须且只须
\[\lim_{\Delta x\to 0^+}\frac{f (x+\Delta x) -f (x)}{\Delta x},\qquad \lim_{\Delta x\to 0^-}\frac{f (x+\Delta x) -f (x)}{\Delta x}\]
同时存在而且相等。

上面两式分别称为$f(x)$在点$x$的\textbf{右导数}和\textbf{左导数},记为$f'_+(x)$和$f'_-(x)$.
\end{enumerate}

\begin{example}
今有一个正在膨胀的肥皂泡,
\begin{enumerate}
\item 求肥皂泡的体积对于半径的增大率,
\item 如果肥皂泡的半径每秒增大0.1cm,问当半径为2cm时,体积的增大率是多少?
\end{enumerate}
\end{example}

\begin{solution}
    肥皂泡的体积$V$与半径$r$的函数关系是。
\[ V(r) =\frac{4}{3}\pi r^3\]
体积$V(r)$对于半径$r$的增大率,依导函数定义,就是$V(r)$对于$r$的导函数,故
\[\begin{split}
    V'(r)&=\lim_{\Delta r\to 0}\frac{V(r+\Delta r)-V(r)}{\Delta r}\\
    &=\lim_{\Delta r\to 0}\frac{\frac{4\pi }{3}[(r+\Delta r)^3-r^3]}{\Delta r}\\
    &=\frac{4\pi }{3}\lim_{\Delta r\to 0}[3r^2+3r(\Delta r)+(\Delta r)^2]\\
    &=4\pi r^2
\end{split}\]

因此,肥皂泡的体积$V$对于半径$r$的增大率是$4\pi r^2$. 当$r=2$时,\[V'(2)=4\pi \cdot 2^2=16\pi \]
上式表示在$r=2$时,肥皂泡体积对于半径的增大率是$16\pi$, 它
的含义也可以这样理解,体积增大的快慢在$r=2$时,是$16\pi$倍于半径增大的快慢。

由于半径的增大率是每秒0.1cm,故体积在半径等于2时的增大率是
\[16\pi \x0.1=1.6\pi ({\rm cm^3/s})\]
\end{solution}

\begin{ex}
\begin{enumerate}
    \item 设$y=\frac{1}{x},\quad (x\ne 0)$,求
\begin{enumerate}
    \item 当$x$取改变量$\Delta x$后,函数改变量$\Delta y$的表达式;
    \item 当$x=3$, $\Delta x=-1$时,$\Delta y$的值;
    \item 当$x=3$时,$\frac{\Delta y}{\Delta x}$的表达式和$\frac{\dd y}{\dd x}\Big|_{x=3}$的值.
\end{enumerate}

    \item 设$\phi(x)=\frac{2}{x^2},\quad (x\ne0)$, 求
\begin{enumerate}
\item 当$x=1$, $\Delta x=-0.5$时,$\Delta \phi(x)$的值;
\item 对于任何$x\; (x\ne 0)$, 在$\Delta x$的间隔内,$\phi(x)$的平均变率$\frac{\Delta \phi}{\Delta x}$的表达式;
\item $\phi(x)$在任何点$x\; (x\ne 0)$处的瞬时变率$\phi'(x)$.
\end{enumerate}

    \item 平均变化率$\frac{\Delta y}{\Delta x}=\frac{f (x+\Delta x) -f (x)}{\Delta x}$依赖于哪两个变量?在平均变化率取极限求瞬时变化率的过程中,$x$是变量还是常量?$\Delta x$是变量还是常量?
    \item 点$P(2, 8)$在抛物线$y=x^2+2x$上,点$Q$为抛物线上任何一点。
\begin{enumerate}
    \item 求割线$PQ$的斜率的表达式;
  \item 当$Q$点横坐标为$2.1, 1.9, 2.002, 1.998, 2+h$时,求割线斜率的值;
\item 求在$P(2, 8)$点处,抛物线$y=x^2+2x$的切线方程和法线方程。
\end{enumerate}    

\item 求圆面积对于它的半径的变率,对于它的直径的变率.
\item 圆的半径的变率为2cm/s, 求圆面积在半径等于4cm
时的变率。
\end{enumerate}
\end{ex}

\subsection{函数的可微性与连续性的关系}

\begin{blk}
    {定理} 如果函数$f(x)$在点$x_0$可微,那么$f(x)$在点$x_0$处连续。
\end{blk}

\begin{proof}
设$f(x)$在点$x_0$处是可微的,也就是$f(x)$在$x_0$处有导数,则
\[f'(x_0)=\lim_{x\to x_0}\frac{f(x)-f(x_0)}{x-x_0}\]
所以
\[\begin{split}
    \lim_{x\to x_0}[f(x)-f(x_0)]&=\lim_{x\to x_0}\frac{f(x)-f(x_0)}{x-x_0}(x-x_0)\\
    &=\lim_{x\to x_0}\frac{f(x)-f(x_0)}{x-x_0}\cdot \lim_{x\to x_0}(x-x_0)\\
&=f'(x_0)\cdot 0=0
\end{split}\]
\end{proof}

上面的定理说明函数在其有导数的点一定连续,在其不
连续的点一定没有导数。但是它的逆命题不一定成立,即连续的函数不一定有导数。前面我们曾指出一个函数可微的充分必要条件是它的左导数和右导数都存在而且相等,下面给出在某些点不可微的函数的例子。













\begin{example}
    
\end{example}

    
\begin{solution}
    
\end{solution}

\begin{example}
    
\end{example}




\begin{solution}
    
\end{solution}









































































































\begin{example}
    
\end{example}

    
\begin{solution}
    
\end{solution}

\begin{example}
    
\end{example}




\begin{solution}
    
\end{solution}
































































\begin{example}
    
\end{example}

    
\begin{solution}
    
\end{solution}

\begin{example}
    
\end{example}




\begin{solution}
    
\end{solution}























































































































\chapter{求函数从$a$到$b$的和与积分}
函数有两个基本性质——变率与和,在前一章,我们研究了函数的瞬时变率的概念,即
\[f'(x_0) =\lim_{h\to 0} \frac{f (x_0+h) -f (x_0)}{h}\]
以及它的应用,在本章我们将研究对于一个给定函数$f(x)$“求从$a$到$b$的和”这个概念。在给出定义之前,我们先举几个实例来看一看。

\paragraph{速率与距离} 
一列行驶中的火车,它的行进速率$v$是时间$t$的函数,即
$v=f(t)$, 如图3.1, 我们可以从速率表读得当时的速率,很自然地,我们想知道火车在从$t=a$到$t=b$这一段时间间隔内一共走了多少路程。从函数的观点看,所谓求在时间$[a,b]$内所走过的距离就是求速率函数$v=f(t)$由$t=a$到$t=b$的和。

\begin{figure}[htp]
    \centering
\begin{tikzpicture}[>=latex]
\draw[->](-.5,0)--(6,0)node[right]{$t$};
\draw[->](0,-.5)--(0,3)node[right]{$v=f(t)$};
\draw[very thick](0,0)--(1,1.5)--(1.6,1.5);
\draw[dashed](1,1.5)--(1,0)node[below]{$a$};
\node at (0,0){
\pscurve[linewidth=1.5pt] (1.6,1.5)(2.5,2.5)(3.5,2.2)(4,1.2)(4.3,.8)(4.8,.6)(5,0)
};
\draw[dashed](3.5,2.2)--(3.5,0)node[below]{$b$};

\end{tikzpicture}
    \caption{}
\end{figure}

让我们用$D(f,[a,b])$
表示所走过的距离,这个记号强调$D$依赖于$f$和区间$[a,b]$。

\paragraph{变力所作的功}

假定某物体在一个平行于$OX$轴的力$P$的作 用下沿直线
$OX$运动,力$P$的方向与物体运动的方向一致,并且力的大小随离开$O$点的距离而改变,即变力$P$是所在点的横坐标$x$的函数$P=P(x)$, 如图3.2。假定物体在这个变力$P$作用之下,从直线$OX$的一点$a$移到另一点$P$, 那么力$P$所作的功就是变力函数$P(x)$由$x=a$到$x=b$的和,我们用
$W (P, [a,b])$
表示力$P$所作的功,它表明$W$依赖于$P(x)$和$[a,b]$。

\begin{figure}[htp]
    \centering
    \begin{minipage}[t]{0.48\textwidth}
    \centering
    \begin{tikzpicture}[>=latex, scale=1]
\draw[->](-.5,0)--(4,0)node[right]{$x$};
\draw[->](0,-.5)--(0,3)node[right]{$P(x)$};
\draw[domain=0:3.5, samples=100, very thick]plot(\x, {3/(\x+1)-.5});
\draw(1,0)node[below]{$a$}--(1,1);
\draw(3,0)node[below]{$b$}--(3,.25);
\node at (0,0)[below left]{$O$};
    \end{tikzpicture}
    \caption{}
    \end{minipage}
    \begin{minipage}[t]{0.48\textwidth}
    \centering
    \begin{tikzpicture}[>=latex, scale=1]
\draw[->](-.5,0)--(4,0)node[right]{$x$};
\draw[->](0,-.5)--(0,3)node[right]{$y$};
\draw[domain=0:3.5, samples=100, very thick]plot(\x, {.3*(\x-2.2)^2+.5})node[above right]{$y=f(x)$};
\node at (0,0)[below left]{$O$};
\draw(.5,0)node[below]{$a$}--(.5,1.367)node[above right]{$M_0$};
\draw(3,0)node[below]{$b$}--(3,.692)node[above]{$M$};
\fill[pattern=north east lines, domain=.5:3, samples=100, very thick]plot(\x, {.3*(\x-2.2)^2+.5})--(3,.692)--(3,0)--(.5,0)--(.5,1.367);


    \end{tikzpicture}
    \caption{}
    \end{minipage}
\end{figure}

\paragraph{曲边梯形的面积}

令$y=f(x)$是函数$f$的图象,表示一条曲线,如图3.3. 我们要求曲线上的一段弧$M_0M$ 与其两端的纵坐标线及$x$轴上的线段$[a,b]$所围成的图形的面积。这样的图形(它有三条边是直线,其中两条互相平行,第三条与前两条
互相垂直,而第四条边是曲线)叫做曲边梯形。显然曲边梯形的面积$A$依赖于$y=f(x)$和$x$轴上的线段$[a,b]$, 记这一面积为$A (f, [a,b])$。

在上一章,我们利用函数$f(x)$和它的图象之间的对应关系,就可以把函数的变率和它的图象的切线的斜率相对应地一并分析讨论,这样,一方面可以把函数的变率这种“数量”的概念用切线来形象化,便于想象;而另一方面又可以把“切线”这种几何概念数量化,便于计算,在这一章,我们也要利用函数$f(x)$和它的图象之间的对应关系,把函数$f(x)$由$x=a$到$x=b$的“和”与曲线$y=f(x)$的曲边梯形的“面积”相对应地一并分析讨论,并且说明函数的“求从$a$到$b$的和”恰好对应于求曲边梯形的面积。这也就是为什么把函数“求从$a$到$b$的和”这种基本运算叫做积分的道理。

\section{“和”与“面积”}
对于任意曲线围成的图形,我们还没有规定它的“面积”的意义,和它密切相关的“函数从$a$到$b$的和”的概念至今也没有明确的解析的定义。在这一节我们要把这两个概念由“直观的定性理解”推进到“数量化的定量定义”。唯有确立了它们的“解析的定义”,它们才真正地成为能算好用的量。

\subsection{“和”与“面积”的基本性质}
现在让我们先从“函数的和”与“曲线形的面积”的直观内涵来分析一下它们分别所应有的基本性质。

\subsubsection{“曲线形的面积”的基本性质}

从面积的直观内涵容易看出下列两点:
\begin{enumerate}
\item (单调性)设区域$R_1$包含在$R_2$之内,即$R_1\subseteq R_2$, 则:
$R_1$的面积$\le R_2$的面积.(图3.4)
\item (可加性)设区域$R$可以用一条曲线分割成两块区
域$R_1+R_2$, 则有:
$R$的面积$=R_1$的面积$+R_2$的面积.(图3.5)
\end{enumerate}

\begin{figure}[htp]
    \centering
    \begin{minipage}[t]{0.48\textwidth}
    \centering
    \begin{tikzpicture}[>=latex, scale=1]
\draw[very thick](0,0) ellipse [x radius=2, y radius =1];
\draw[very thick](-.5,0) ellipse [x radius=1, y radius =.6];
\node at (0,1.3){$R$};
\node at (-.5,0){$R_1$};
\node at (1.3,0){$R_2$};
    \end{tikzpicture}
    \caption{}
    \end{minipage}
    \begin{minipage}[t]{0.48\textwidth}
    \centering
    \begin{tikzpicture}[>=latex, scale=1]
\draw[very thick](0,0) ellipse [x radius=2, y radius =1];
\node at (0,1.3){$R$};
\node at (-1,0){$R_1$};
\node at (1,0){$R_2$};
\draw[very thick](0,1) to [bend left](.3,.6)to [bend left](0,-.3)to [bend right](0,-1);
    \end{tikzpicture}
    \caption{}
    \end{minipage}
\end{figure}

\subsubsection{“函数的和”的基本性质}

同样地,从“函数从$a$到$b$的和”的直观内涵容易看出下列两个基本性质,即

\begin{blk}{性质1:单调性}
设函数$f(t)$和$g(t)$在$a\le t\le b$上,$f(t)\le g(t)$恒成立,那么:

“$f(t)$由$t=a$到$t=b$的和”$\le $“$g(t)$由$t=a$到$t=b$
的和”。
\end{blk}

例如,两部车子在$t=a$到$t=b$的时间内,甲车的速率$f(t)\le $乙车的速率$g(t)$, 则甲车在上述时间内所经过的里程$\le $乙车在上述时间内所经过的里程。

\begin{blk}{性质2:可加性}
设$a<b<c$,那么,有:

“$f(t)$由$t=a$到$t=b$的和”$+$“$f(t)$由$t=b$到$t=c$的和”$=$“$f(t)$由$t=a$到$t=c$的和”。
\end{blk}

上述基本性质是一目了然的,让我们先用来说明一些简单的基本事实。

\begin{example}
    设函数$y=f(t)=k$(常数),则由和的直观内涵,显然应有:
\[\text{常数函数从$a$到$b$的和}=k(b-a)\]

例如,以等速率每小时$k$公里行驶的火车从$t=a$小时到$t=b$小时内所走的总里程应该是$k(b-a)$公里。

现在让我们利用函数的图象来观察上述数值$k(b-a)$的几何意义。

\begin{figure}[htp]
    \centering
    \begin{minipage}[t]{0.48\textwidth}
    \centering
    \begin{tikzpicture}[>=latex, scale=.9]
\draw[->](-.5,0)--(5,0)node[right]{$t$};
\draw[->](0,-.5)--(0,3)node[right]{$y$};
\fill[pattern=north east lines](1,0) rectangle (4,1.5);
\draw[dashed](0,1.5)--(1,1.5)--(1,0)node[below]{$t=a$};
\draw[dashed](4,1.5)--(4,0)node[below]{$t=b$};
\node at (0,0)[below left]{$O$};
\draw[very thick](1,1.5)--(4,1.5);
\node at (0,.75)[left]{$k>0$};
\node at (2.5,.75){\Large $+$};

    \end{tikzpicture}
    \caption{}
    \end{minipage}
    \begin{minipage}[t]{0.48\textwidth}
    \centering
    \begin{tikzpicture}[>=latex, scale=.9]
\draw[->](-.5,0)--(5,0)node[right]{$t$};
\draw[->](0,-2)--(0,1.5)node[right]{$y$};
\fill[pattern=north east lines](1,0) rectangle (4,-1.5);
\draw[dashed](0,-1.5)--(1,-1.5)--(1,0)node[above]{$t=a$};
\draw[dashed](4,-1.5)--(4,0)node[above]{$t=b$};
\node at (0,0)[below left]{$O$};
\draw[very thick](1,-1.5)--(4,-1.5);
\node at (0,-.9)[left]{$k<0$};
\node at (2.5,-.75){\Large $-$};
    \end{tikzpicture}
    \caption{}
    \end{minipage}
\end{figure}

\begin{enumerate}
    \item 当$k>0$时,函数$f(t)=k$的图象在$t$轴上方,$k(b-a)$对应于一个矩形的面积,如图3.6中阴影部分的面积.
    \item 当$k<0$时,函数$f(t)=k$的图象在$t$轴下方,$k(b-a)$对应于一个矩形的面积的负值,如图3.7中阴影部分的面积的负值。
\end{enumerate}

假如我们把$t$轴之上的面积定义为正的,而把$t$轴之下的面积定义为负的,则当$y=k$(常数)时,常数函数从$a$到$b$的和$=k(b-a)=$上述的有号面积。
\end{example}

\begin{example}
设$y=f(t)$是一个阶梯函数,即分段地是常数函数:
\[0=t_0<t_1<t_2<\cdots<t_{i-1}<t_i<\cdots <t_n=b\]
当$t_{i-1}\le t<t_i$时,$f(t)=k_i,\; i=1,2,\ldots,n$。由性质2,我们可以分段地用常数函数求和,就可以得到阶梯函数$f(t)$从$a$到$b$的和:
\[k_1 (t_1-t_0) +k_2 (t_2-t_1) +\cdots +k_n (t_n-t_{n-1})\]

作出阶梯函数的图象(图3.8),上述函数和就等于下列逐段由矩形并起来的区域的“有号面积”。
\begin{figure}[htp]
    \centering
\begin{tikzpicture}[>=latex, yscale=.7]
\draw[->](-5,0)--(7,0)node[right]{$t$};
\draw[->](0,-2)--(0,4)node[right]{$y=f(t)$};
\node at (0,0)[above right]{$O$};

\foreach \x/\xtext in {-3/t_1, -1/t_2, 1.5/t_3, 2.5/t_4, 3.5/t_5, 4.3/t_6, 5.5/t_{n-1},-4.5/{t_0=a},6.5/{t_n=b}}
{
    \node at (\x,0)[below]{$\xtext$};
}

\draw[pattern=north east lines, thick](-4.5,0) rectangle(-3,1.5);
\draw[pattern=north east lines, thick](-3,1) rectangle(-1,0);
\draw[pattern=north east lines, thick](-1,0) rectangle(1.5,-1.5);
\draw[pattern=north east lines, thick](1.5,0) rectangle(2.5,1);
\draw[pattern=north east lines, thick](2.5,0) rectangle(3.5,2);
\draw[pattern=north east lines, thick](3.5,0) rectangle(4.3,-1);
\draw[pattern=north east lines, thick](5.5,0) rectangle(6.5,1);
\foreach \x in {-3.75, -2, 2, 3, 6}
{
    \node at (\x,.5){$+$};
}
\foreach \x in {.75, 3.9}
{
    \node at (\x,-.5){$-$};
}
\node at (4.8,.5){…};\node at (4.8,-.5){…};

\end{tikzpicture}
    \caption{}
\end{figure}
\end{example}

\subsection{逼近法求和}

对于一个比较一般的函数,例如$y=kt+c$, $y=t^2$, $y=\sin t$等等,我们必须给函数的和下一个适当的定义,同时要提供计算这个和的方法,这里我们将要再一次地用逼近法的观点去解答上述问题,回顾上一章讨论变率时,我们的基本想法是用折线函数去逼近一般的平滑函数,所以折线函数是讨论函数变率的简单好用的基本函数,因为它们的变率十分简单,分段地是个常数而且无限逼近函数在某一点的变率,由上一段的讨论;我们知道阶梯函数的“从$a$到$b$的和”是十分简明的,它们在“函数求从$a$到$b$的和”这方面是不是也扮演着
这种既简单又基本的角色呢?这就得看一看能否用阶梯函数的从$a$到$b$的和去无限逼近一般性的“函数从$a$到$b$的和”了。

现在让我们使用几个简单的实例来试试看。

\begin{example}
假设物体以初速度$u$, 加速度$a>0$在直线上运动,于是物体在任何时刻$t$的速度是$v=f (t) =u+at$, 求物体从$t=0$到$t=T$时所经过的距离,即求速度函数$f(t)$在时间间隔$[0,T]$上的和.
\end{example}

\begin{analyze}
    一个自然的作法是用两个阶梯函数$g_n(t)$和$G_n(t)$把上述函数$f(t)=u+at$夹逼在中间,即
    \[g_n(t)\le f(t)\le G_n(t)\]
对于任何$0\le t\le T$都成立,那么由性质1,$g_n(t)$从0到$t$的和$\le f(t)$从$0$到$T$的和$\le G_n(t)$从0到$T$的和。

如果能使$g_n(t)$从0到$T$的和与$G_n(t)$从0到$T$的和,当$n\to\infty$时的极限相同,则由上述夹逼关系就可以看出$f(t)$从0到$T$的和必须等于这个共同极限.
\end{analyze}

\begin{figure}[htp]
    \centering
\begin{tikzpicture}[>=latex]
\draw[->](-.5,1)--(8,1)node[right]{$t$};
\draw[|<->|](-.2,1)--node[fill=white]{$u$}(-.2,2);
\draw[->](0,0)--(0,6)node[right]{$v=f(t)=u+at,\; (0\le t\le T)$};
\node at (0,1)[below left]{$O$};
\draw[ultra thick](0,2)node[above left]{$A$}--(7,5.5)node[above]{$B$};
\draw[<->](7.3,1)node[below]{$C$}--node[fill=white]{$u$}(7.3,2);
\draw[|<->|](7.3,5.5)--node[fill=white]{$aT$}(7.3,2);
\foreach \x in {1,2,3,...,6,7}
{
    \draw[dashed](\x,1)node[below]{$t_{\x}$}--(\x,2+\x*.5);
    \draw[very thick](\x-1, 1.5+\x*0.5)--(\x, 1.5+ \x*0.5)--(\x, 2+\x*0.5);
    \draw[thick](\x-1, 1.5+\x*0.5)--(\x-1, 2+\x*0.5)--(\x, 2+ \x*0.5);
}
\draw[dashed](0,2)--(7,2);
\draw(7,0)--(7,1);
\draw[<->](0,.2)--node[fill=white]{$T$}(7,.2);
\end{tikzpicture}
    \caption{}
\end{figure}

下面就是把上述想法付诸实践的具体做法之一。

\begin{solution}
\begin{enumerate}
    \item 把时间间隔$n$等分,取分点$t_i=\frac{i}{n}T,\; i=0,1, 2,\ldots,n$, 于是在每个分点$t_i=\frac{i}{n}T$处的速度分别是
\[f (t_0) =f (0) =u,\;  f(t_1) =u+at_1,\; f (t_3) =u+at_2,\ldots,f (t_n) =f (T) =u+aT\]
\item 用阶梯函数近似代替$v=f(t)=u+at$. 

假设物体在时间的每个小区间$[t_i,t_{i+1}],\; i=0,1, 2,\ldots,(n-1)$内,以小区间起点处的速度作匀速运动,我们得到一个速度的阶梯函数
\[g_n (t)=f(t_i)=u+at_i,\quad t_i\le t<t_{i+1},\quad i=0,1,2,\ldots,(n-1)\]
因为$f(t)=u+at$是严格递增的,所以$g_n(t)$有以下性质:
\[g_n (t)=f(t_i)\le f(t),\quad t_i\le t<t_{i+1},\quad i=0,1,2,\ldots,(n-1)\]
从而,对于任何$0\le t\le T$,有$g_n(t)\le f(t)$.

假设物体在时间的每个小区间$[t,t+1],\; i=0,1, 2,\ldots,(n-1)$内,以小区间终点处的速度作匀速运动,我们得到另一个速度的阶梯函数
\[G_n(t)=f(t_{i+1})=u+at_{i+1},\quad t_i< t\le t_{i+1},\quad i=0,1,2,\ldots,(n-1)\]

由于$f(t)=u+at$是递增的,所以$G(t)$有以下性质
\[G_n (t)=f(t_{i+1})\ge f(t),\quad t_i< t\le t_{i+1},\quad i=0,1,2,\ldots,(n-1)\]
从而对于任何$0\le t\le T$,有
\[G_n(t)\ge f(t)\]
这样,我们就得到了$v=f(t)$的夹逼阶梯函数:
\[ g_n(t) \le f (t) \le G_n (t),\quad 0\le t\le T\]

(在图3.8中,绘出了$f(t)=u+at$的图象和当$n=7$时阶梯函数$g_7(t)$和$G_7(t)$的图象.)

\item 求阶梯函数的总和

应用基本性质2, 我们得到$g_n(t)$从0到$T$的和
\[f(t_0)(t_1-t_0)+f(t_1)(t_2-t_1)+\cdots+f (t_{n-1}) (t_n-t_{n-1})\]

$\because\quad t_1-t_0=t_2-t_1=\cdots=t_n-t_{n-1}=\frac{T}{n}$

因此:
\[\begin{split}
\text{$g_n(t)$从0到$T$的和}&=\left[f(t_0)+f(t_1)+\cdots+f(t_{n-1})\right]\cdot \frac{T}{n}\\
&=\left[u+(u+at_1)+(u+at_2)+\cdots +(u+at_{n-1})\right]\cdot \frac{T}{n}\\
&=\left[nu+a(t_1+t_2+\cdots+t_{n-1})\right]\cdot \frac{T}{n}\\
&=\left[nu+a\left(\frac{T}{n}+\frac{2T}{n}+\cdots+\frac{(n-1)T}{n}\right)\right]\cdot \frac{T}{n}\\
&=uT+a\cdot [1+2+\cdots+(n-1)]\left(\frac{T}{n}\right)^2\\
&=uT+a\cdot \frac{n(n-1)}{2}\cdot \frac{T^2}{n^2}\\
&=uT+\frac{1}{2}aT^2\left(1-\frac{1}{n}\right)
\end{split}\]

\[\begin{split}
    \text{$G_n(t)$从0到$T$的和}&=f(t_1)(t_1-t_0)+f(t_2)(t_2-t_1)+\cdots+f(t_n)(t_n-t_{n-1})  \\
   &= \left[f(t_1)+f(t_2)+\cdots+f(t_{n})\right]\cdot \frac{T}{n}\\
    &=\left[(u+at_1)+(u+at_2)+\cdots +(u+at_{n})\right]\cdot \frac{T}{n}\\
    &=\left[nu+a\left(\frac{T}{n}+\frac{2T}{n}+\cdots+\frac{nT}{n}\right)\right]\cdot \frac{T}{n}\\
    &=uT+a\cdot (1+2+\cdots+n)\cdot \left(\frac{T}{n}\right)^2\\
    &=uT+\frac{1}{2}aT^2\left(1+\frac{1}{n}\right)
    \end{split}\]
综合上述计算和基本性质1, 即得:$g_n(t)$从0到$T$的和$\le f(t)$从0到$T$的和$\le G_n(t)$从0到$T$的
和,即
\[uT+\frac{1}{2}aT^2\left(1-\frac{1}{n}\right)\le f(t)
\text{从0到$T$的和}\le uT+\frac{1}{2}aT^2\left(1+\frac{1}{n}\right)\]

\item 求阶梯函数和式的极限.

因为
\[\begin{split}
    \lim_{n\to\infty}\left\{uT+\frac{1}{2}aT^2\left(1-\frac{1}{n}\right)\right\}&=\lim_{n\to\infty}\left\{uT+\frac{1}{2}aT^2\left(1+\frac{1}{n}\right)\right\}\\
    &=uT+\frac{1}{2}aT^2
\end{split}\]
所以,这个共同的极限值$uT+\frac{1}{2}aT^2$就是物体在变速$v=f(t)=u+at$运动下,从$t=0$到$t=T$所走的距离,也就是函数$v=f(t)=u+at$从$t=0$到$t=T$的和.
\end{enumerate}

从几何上看,上述极限值$uT+\frac{1}{2}aT^2=\frac{u+(u+aT)T}{2}$就是在图3.8中的梯形$OABC$的面积.
\end{solution}    

\begin{example}
    设函数$f(x)=x^2$, $a=0$, $b>0$, 求$f(x)$由0到$b$的和,相应地,求抛物线$y=x^2$在线段$[0,b]$上所盖的曲边三角形$OBC$的面积(图3.9).
\end{example}

\begin{figure}[htp]
    \centering
\begin{tikzpicture}[>=latex, scale=.5]
\draw[->](-5,0)--(5,0)node[right]{$x$};
\draw[->](0,-1.5)--(0,11)node[right]{$y$};
\draw[domain=-3.25:3.25, samples=100, very thick]plot(\x, {\x*\x});
\node at (0,0)[below left]{$O$};
\foreach \x in {.5,1,...,3}
{
    \draw[thick](\x-.5,0) rectangle (\x, \x*\x);
}
\foreach \x in {.5,1,...,2.5}
{
    \draw(\x,\x*\x) -- (\x+.5,\x*\x);
}
\node at (3,9)[right]{$C$};
\node at (3,0)[below]{$B$};
\node at (-2.5,6)[left]{$y=x^2$};
\end{tikzpicture}
    \caption{}
\end{figure}

\begin{solution}
\begin{enumerate}
    \item 把线段$[0,b]$ $n$等分,等分点的坐标是
   $ x_i=\frac{b}{n}i\;  (i=0, 1, 2,\ldots,n)$ 
   \item 作阶梯函数$g_n(x)$和
    $G_n (x)$.

    根据$f(x)=x^2$在$x>0$时递增,定义
\[\begin{split}
g_n(x)&=f(x_i)=x^2_i=\left(\frac{b}{n}\right)^2i^2,\quad x_i\le x<x_{i+1}\\
G_n(x)&=f(x_{i+1})=x^2_{i+1}=\left(\frac{b}{n}\right)^2(i+1)^2,\quad x_i<x\le x_{i+1}\\
\end{split}\]
其中,$i=0,1,2,\ldots,(n-1)$。于是,$g_n(x)\le f(x)\le G_n(x),\quad 0\le x\le b$

\item 求阶梯函数的和
\[\begin{split}
s_n&=g_n(x)\text{从0到$b$的和}\\
&=f\left(x_{0}\right) \cdot \frac{b}{n}+f\left(x_{1}\right) \cdot \frac{b}{n}+\cdots+f\left(x_{n-1}\right) \cdot \frac{b}{n}\\
&=\left\{f\left(x_{0}\right)+f\left(x_{1}\right)+\cdots+f\left(x_{n-1}\right)\right\} \cdot \frac{b}{n} \\
&=\left\{0^{2}+1^{2}+2^{2}+\cdots+(n-1)^{2}\right\} \cdot\left(\frac{b}{n}\right)^3\\
&=\frac{1}{6}n(n-1)(2 n-1) \cdot\left(\frac{b}{n}\right)^{3} =\frac{b^3}{6}\left(1-\frac{1}{n}\right)\left(2-\frac{1}{n}\right) 
\end{split}\]
\[\begin{split}
   S_{n}= G_{n}(x) \text { 从0到$b$的和}
   &= \left\{f\left(x_{1}\right)+f\left(x_{2}\right)+\cdots+f\left(x_{n}\right)\right\} \cdot \frac{b}{n} \\
   &=\left\{1^2+2^{2}+\cdots+n^{2}\right\} \cdot\left(\frac{b}{n}\right)^{3} \\
   &=\frac{1}{6}n(n+1)(2 n+1) \cdot\left(\frac{b}{n}\right)^{3} \\
   &=\frac{b^3}{6}\left(1+\frac{1}{n}\right)\left(2+\frac{1}{n}\right) 
\end{split}\]

综合上述计算和性质1,有:
\[\begin{split}
\frac{b^3}{6}\left(1-\frac{1}{n}\right)\left(2-\frac{1}{n}\right)&=s_n\le \begin{bmatrix}
\text{函数$f(x)=x^2$从0到$b$的和,}\\
\text{相应的曲边三角形$OBC$的面积}
\end{bmatrix}\\
&\le S_n=\frac{b^3}{6}\left(1+\frac{1}{n}\right)\left(2+\frac{1}{n}\right)
\end{split}\]

\item 令$n\to\infty$,求阶梯函数和式的极限。因为
\[\begin{split}
    \frac{b^3}{6} \left(1-\frac{1}{n}\right)\left(2-\frac{1}{n}\right)\to \frac{b^3}{3}\qquad (n\to\infty)\\
    \frac{b^3}{6} \left(1+\frac{1}{n}\right)\left(2+\frac{1}{n}\right)\to \frac{b^3}{3}\qquad (n\to\infty)\\
\end{split}\]
所以$\frac{1}{3}b^3$是能被$s_n$、$S_n$左、右夹逼的唯一实数,而所求的“$f(x)=x^2$从0到$b$的和”以及“曲边三角形$OBC$的面积,”这两个值都是夹逼在$s_n$和$S_n$中间的实数,故它们都等于$\frac{1}{3}b^3$.
\end{enumerate}
\end{solution}

\begin{blk}
    {推论} $f(x)=x^2$从$a$到$b$的和$=\frac{1}{3}b^3-\frac{1}{3}a^3\quad  (b> a> 0)$.
\end{blk}


\begin{example}
设$y=f(x)$是一个定义在$[a,b]$上的递增连续函数,试说明它从$a$到$b$的和是可以确定的。
\end{example}

\begin{figure}[htp]
    \centering
\begin{tikzpicture}[>=latex]
\draw[->](-.5,0)--(5,0)node[right]{$x$};
\draw[->](0,-.5)--(0,4)node[right]{$y$};
\draw[domain=.5:4, samples=50, very thick]plot(\x, {.2*\x*\x+.5});
\draw[dashed](1,.7)--(1,0)node[below]{$a$};
\draw[dashed](4,3.7)--(4,0)node[below]{$b$};
\draw[dashed](2,1.3)--(2,0);
\draw[dashed](3,2.3)node[left]{$y=f(x)$}--(3,0);
\node at (0,0)[below left]{$O$};
\end{tikzpicture}
    \caption{}
\end{figure}

\begin{solution}
\begin{enumerate}
    \item     取$[a,b]$之间的$n$等分点,将它分成$n$段,即有
\[x_0=a<x_1<x_2<\cdots<x_n=b\]
它的第$i$段是$[x_{i-1},x_i]$,且$x_i-x_{i-1}=\frac{b-a}{n}$,$x_i=a+\frac{i}{n}(b-a)$

\item 定义$f(x)$的上、下夹逼阶梯函数如下:
\[\begin{split}
    g_n(x)&=f(x_{i-1}),\qquad x_{i-1}\le x\le x_i \\
    G_n(x)&=f(x_i),\qquad x_{i-1}\le x\le x_i 
\end{split}(i=1,2,3,\ldots,n)\]
由$f(x)$的递增性,对于任何一个$x\in[a,b]$,有
\[g_1(x)\le g_2(x)\le\cdots \le g_n(x)\le\cdots \le f(x)\le \cdots\le G_n(x)\le\cdots\le G_2(x)\le G_1(x)\]
简写成
\[g_n(x)\le f(x)\le G_n(x),\qquad a\le x\le b\]
\item 求相应的阶梯函数从$a$到$b$的和
\begin{equation}
\begin{split}
    s_n=g_n(x)\text{从$a$到$b$的和}
    &=\frac{b-a}{n}\left[f(x_0)+f(x_1)+\cdots+f(x_{n-1})\right]\\
    &=\frac{b-a}{n}\sum^n_{i=1}f(x_{i-1})
\end{split}
\end{equation}
\begin{equation}
    \begin{split}
    S_n=G_n(x)\text{从$a$到$b$的和}
    &=\frac{b-a}{n}\left[f(x_1)+f(x_2)+\cdots+f(x_{n})\right]\\
    &=\frac{b-a}{n}\sum^n_{i=1}f(x_{i})   
    \end{split}
    \end{equation}
\end{enumerate}
由性质1, 有:
\[s_1<s_2<\cdots<s_n<\cdots<f(x)\text{从$a$到$b$的和}<\cdots<S_n<\cdots<S_2<S_1\]
简写成
\begin{equation}
    s_n<f(x)\text{从$a$到$b$的和}<S_n
\end{equation}
在例3.3, 例3.4中,$f(x)$有明确的解析式,我们可以用求和公式直接求出$s_n$和$S_n$的表达式,从而可以得出它们的共同极限值,在这里,我们只知道$f(x)$的递增性,当然无法将和式$s_n$和$S_n$进一步化简,但是我们可以说明,当$n\to\infty$时,$S_n-s_n\to 0$.

事实上,
\[\begin{split}
    S_n-s_n&=\frac{b-a}{n}\sum^n_{i=1}f(x_{i})   -\frac{b-a}{n}\sum^n_{i=1}f(x_{i-1}) \\
    &=\frac{b-a}{n}\sum^n_{i=1}\left[f(x_{i})-f(x_{i-1})\right]   \\
    &=\frac{b-a}{n}\left[f(x_{n})-f(x_{0})\right]  =\frac{b-a}{n}\left[f(b)-f(a)\right]   
\end{split}\]
因此,当$n\to\infty$时,$S_n-s_n\to 0$.

又$f(x)$从$a$到$b$的和是介于$s_n$和$S_n$中间的唯一实数,
此
\[\lim_{n\to\infty} s_n=\lim_{n\to\infty} S_n=f(x)\text{ 从$a$到$b$的和}\]

以上简单明了的分析,说明了下面两点互相关联的事实:
\begin{enumerate}
    \item 当函数$f(x)$在$a\le x\le b$上递增时,则它从$a$到$b$的和可以用上述两个夹逼阶梯函数从$a$到$b$的和去无限逼近.
    \item 由于$\Lim_{n\to\infty}s_n$与$\Lim_{n\to\infty}S_n$存在,且$\Lim_{n\to\infty}s_n=\Lim_{n\to\infty}S_n$, 我们可以用$f(x)$的上、下夹逼阶梯函数序列,即:$g_1 (x)\le g_2(x)\le \cdots\le g_n(x)\le \cdots\le f(x)\le \cdots\le G_1(x)\le \cdots\le G_2(x)\le G_1(x),\; (a\le x\le b)$的从$a$到$b$的和的极限$\Lim_{n\to\infty}s_n=\Lim_{n\to\infty}S_n$作为上述$f(x)$从$a$到$b$的和的数量化定义。
\end{enumerate}
\end{solution}

\begin{blk}
    {定义1}若$f(x)$是$[a,b]$上的递增连续函数,将$[a,b]$等分成$n$段,分点坐标$x_i=a+\frac{i}{n}(b-a),\; i=0,1, 2,\ldots,n$,则存在两个阶梯函数:
\[    g_n (x) =f (x_{i-1}) ,\quad x_{i-1}\le  x< x_i\]
    和
\[    G_n (x)=f(x_i),\quad x_{i-1}<x\le x_i,\quad (i=1,2,\ldots,n)\]
满足下面的性质:
\begin{enumerate}
    \item 对于任何$x\in[a,b]$,$g_n(x)\le f(x)\le G_n(x)$
    \item 相应的阶梯函数从$a$到$b$的和
\[s_n=\frac{b-a}{n}\sum^n_{i=1}f(x_{i-1}),\qquad S_n=\frac{b-a}{n}\sum^n_{i=1}f(x_{i})  \]
\end{enumerate}
当$n\to\infty$时,$S_n-s_n\to 0$,那么$f(x)$从$a$到$b$的和是$s_n$与$S_n$的共同极限,即:
\[\begin{split}
f(x)\text{从$a$到$b$的和}&=\frac{b-a}{n}\lim_{n\to\infty}\sum^n_{i=1}f(x_{i-1})\\
&=\frac{b-a}{n}\lim_{n\to\infty}\sum^n_{i=1}f(x_{i})
\end{split}\]

同样地,如果$f(x)$是$[a,b]$上的递减连续函数。那么将$[a,b]$ $n$等分,分点坐标$x_i=a+\frac{i}{n}(b-a)$, 这时有两个阶梯函数
\[    g_n (x) =f (x_{i}) ,\quad x_{i-1}<  x\le x_i\]
    和
\[    G_n (x)=f(x_{i-1}),\quad x_{i-1}\le x< x_i,\quad (i=1,2,\ldots,n)\]
满足下面的性质:
\begin{enumerate}
    \item 对于任何$x\in[a,b]$,$g_n(x)\le f(x)\le G_n(x)$
    \item 相应的阶梯函数$g_n(x)$与$G_n(x)$相应的和
\[\begin{split}
    s_n&=g_n\text{从$a$到$b$的和}=\frac{b-a}{n}\sum^n_{i=1}f(x_{i})\\
    S_n&=G_n\text{从$a$到$b$的和}=\frac{b-a}{n}\sum^n_{i=1}f(x_{i-1})  
\end{split}\]
\end{enumerate}
当$n\to\infty$时,$S_n-s_n=\frac{b-a}{n}\left[f(b)-f(a)\right]\to 0$,即:
\[\Lim_{n\to\infty}s_n=\Lim_{n\to\infty}S_n\]
\end{blk}

\begin{blk}
    {定义2} 递减连续函数从$a$到$b$的和为上述夹逼阶梯函数$g_n(x)$和$G_n(x)$从$a$到$b$的和的共同极限,即  
\[\begin{split}
    f(x)\text{从$a$到$b$的和}&=\lim_{n\to\infty} \frac{b-a}{n}\sum^n_{i=1}f(x_i)\\
    &=\lim_{n\to\infty} \frac{b-a}{n}\sum^n_{i=1}f(x_{i-1})\\
\end{split}\]
\end{blk}

通常我们把递增或递减的函数合称为单调函数,常见的
函数$f(x)$都是分段单调连续的,例如,$y=\sin x$本身虽然不是单调的,但是它在$\left[-\frac{\pi}{2}+2k\pi,\frac{\pi}{2}+2k\pi\right]\; (k\in\mathbb{Z})$这些区间的每一段是递增的;而在$\left[\frac{\pi}{2}+2k\pi,\frac{3\pi}{2}+2k\pi\right]\; (k\in\mathbb{Z})$这些区间的每一段是递减的。因此,对于一般在$[a,b]$上连续的函数,如果存在有限个分点使得$f(x)$在每个分段上都是单调的,我们可以逐段取上、下夹逼阶梯函数,合起来作为定义在$[a,b]$上的函数$f(x)$的阶梯函数,于是得出
\[g_n (x)\le f(x)\le G_n(x),\qquad x\in [a,b]\]
又因为有限个在分段上趋于0的量的和仍趋于0, 所以
\[\text{“$G_n(x)$从$a$到$b$的和”}-\text{“$g_n(x)$从$a$到$b$的和”}\to 0\]

总结以上讨论,我们叙述为存在定理和定义如下:

\begin{blk}
    {定理} 设$y=f(x)$是一个定义在$[a,b]$上分段单调函数,则存在满足下列性质的两系列阶梯函数:
    \[g_n(x)\le f(x)\le G_n(x),\qquad x\in[a,b]\]
而且当$n\to 0$时,“$G_n(x)$从$a$到$b$的和,”与“$g_n(x)$从$a$到$b$的和”趋于共同极限。
\end{blk}


\begin{blk}
    {定义3} 设$f(x)$是$[a,b]$上的连续函数,如果 存在有限个分点:
\[a=a_0<a_1<\cdots<a_k<\cdots<a_1=b\]
使得$f(x)$在每个分段$[a_{k-1},a_k]$都是单调的,再将每个分段
    $[a_{k-1},a_k]$都$n$等分,则当$n\to\infty$时,有
\[\begin{split}
f(x)\text{从$a$到$b$的和}&=\lim_{n\to\infty}\frac{a_1-a_0}{n}\sum^n_{i=1}f(x_i)+\lim_{n\to\infty}\frac{a_2-a_1}{n}\sum^{2n}_{i=n+1}f(x_i)+\\
&\cdots +\lim_{n\to\infty}\frac{a_\ell-a_{\ell-1}}{n}\sum^{\ell n}_{i=\ell(n-1)+1}f(x_i)
\end{split}\]
\end{blk}

为了说明上述定义是合理的,我们就得证明上述$f(x)$从$a$到$b$的和与夹逼的阶梯函数列$\{G_n(x)\}$和$\{g_n(x)\}$的选取无关,其证明如下:

\begin{proof}
    设$\{\bar G_m(x)\}$和$\{\bar g_m(x)\}$是另外一组满足存在定理的上、下夹逼函数列,则由下述不等式
\[g_n(x)\le f(x)\le G_n(x),\quad \bar g_m(x)\le f(x)\le \bar G_m(x),\qquad a\le x\le b\]
即有
\[g_n(x)\le \bar G_m(x),\qquad \bar g_m(x)\le G_n(x)\]
所以由基本性质1,有
\[\begin{split}
    s_n&=g_n(x)\text{从$a$到$b$的和}\le \bar G_m(x)\text{从$a$到$b$的和}=\bar S_m\\
    \bar s_m&=\bar g_m(x)\text{从$a$到$b$的和}\le G_n(x)\text{从$a$到$b$的和}=S_n\\
\end{split}\]
所以
\[\lim_{n\to\infty}s_n=\lim_{m\to\infty}\bar S_m,\qquad \lim_{m\to\infty}\bar s_m\le \lim_{n\to\infty}S_n\]
但是
\[\lim_{n\to\infty}s_n\le \lim_{m\to\infty}\bar S_m=\lim_{m\to\infty}\bar s_m\le \lim_{n\to\infty} S_n=\lim_{n\to\infty}s_n\]
所以上述极限必须相等,即:
\[\lim_{n\to\infty}s_n= \lim_{m\to\infty}\bar S_m=\lim_{m\to\infty}\bar s_m=\lim_{n\to\infty} S_n\]
\end{proof}

\begin{example}
求$f(x)=2x-x^2$从0到2的和.
\end{example}

\begin{solution}
    由$f'(x)=2-2x$, $f''(x)=2$知$f(1)=2-1=1$是
$f(x)$的极大值,并且$y=2x-x^2$在$[0, 1]$上递增,在$[1,2]$上递减,于是,我们把区间$[0, 1]$和$[1, 2]$都$n$等分,设分点坐标$x_i=\frac{i}{n},\; i=0, 1, 2,\ldots,2n$, 即有
\[0=x_0<x_1<x_2<\cdots <x_n=1<x_{n+1}<x_{n+2}<\cdots <x_{2n}=2\]
且$\Delta x_i=x_i-x_{i-1}=\frac{1}{n}$,由定义3得到:
\[\begin{split}
    f(x)&=2x-x^2\text{从0到2的和}\\
    &=\lim_{n\to\infty}\frac{1}{n}\sum^n_{i=1}\left[2\left(\frac{i}{n}\right)-\left(\frac{i}{n}\right)^2\right]+\lim_{n\to\infty}\frac{1}{n}\sum^{2n}_{i=n+1}\left[2\left(\frac{i}{n}\right)-\left(\frac{i}{n}\right)^2\right] \\
&=\lim_{n\to\infty}\frac{1}{n}\sum^{2n}_{i=1}2\left(\frac{i}{n}\right)-\lim_{n\to\infty}\sum^{2n}_{i=1}\left(\frac{i}{n}\right)^2\\
&=2\lim_{n\to\infty}\frac{1}{n^2}\sum^{2n}_{i=1}i-\lim_{n\to\infty}\frac{1}{n^3}\sum^{2n}_{i=1}i^2\\
&=2\lim_{n\to\infty}\frac{1}{n^2}\cdot \frac{(2n+1)2n}{2}-\lim_{n\to\infty}\frac{1}{n^3}\cdot \frac{2n(2n+1)(4n+1)}{6}\\
&=2\lim_{n\to\infty}\left(2+\frac{1}{n}\right)-\lim_{n\to\infty}\frac{1}{3}\left(2+\frac{1}{n}\right)\left(4+\frac{1}{n}\right)\\
&=4-\frac{8}{3}=\frac{4}{3}
\end{split}\]
\end{solution}

\begin{ex}
\begin{enumerate}
    \item 已知质点的运动速度$v=t+4$, 试求质点在前10秒内所走的路程。
    \item 求$f(x)=x^3$在$1\le x\le 2$上的和。
\end{enumerate}
\end{ex}


\section{定积分的定义和基本性质}
\subsection{定积分定义}

设函数$f(x)$在$[a,b]$上连续,如果存在有限个分点
\[a=a_0<a_1<a_2<\cdots<a_{\ell-1}<a_{\ell}=b\]
使得$f(x)$在每个分段$[a_{k-1},a_k]\; (k=1, 2,\ldots,\ell)$上都是单调的,我们把$f(x)$从$a$到$b$的和叫做$f(x)$从$a$到$b$的\textbf{定积分},并记作
\[\int^b_a f(x) \dd x\]

这里,积分符号所使用的是长$S$形的求和号的变形,而从部分区间长$\Delta x_i=\frac{a_k-a_{k-1}}{n}$过渡到极限,则通过字母$\dd$来表示。

我们把数$a$与$b$称为\textbf{积分限}($a$称为\textbf{下限},$b$称为\textbf{上限})。区间$[a,b]$称为\textbf{积分区间},函数$f(x)$称为\textbf{被积函数},乘积$f(x)\dd x$称为\textbf{被积表达式},定积分符号下出现的字母$x$叫做\textbf{积分变量}。

上述定积分定义用定积分符号表示就是:
\[\int^b_a f(x) \dd x=\sum^t_{k=1}\int^{a_k}_{a_{k-1}}f(x)\dd x\]
其中$\Int^{a_k}_{a_{k-1}}f(x)\dd x$是$f(x)$为单调的第$k$个分段$[a_{k-1},a_k]$上的夹逼阶梯函数$g_n(x)$和$G_n(x)$从$a_{k-1}$到$a_k$的和的共同极限,即:
\[\begin{split}
    \int^{a_k}_{a_{k-1}}f(x)\dd x&=\lim_{n\to \infty}\frac{a_k-a_{k-1}}{n}\sum^n_{i=1}f(x_{i-1})\\
    &=\lim_{n\to \infty}\frac{a_k-a_{k-1}}{n}\sum^n_{i=1}f(x_{i})
\end{split}\]

定积分定义的另一种表述形式是:设函数$f(x)$在$a\le x\le b$上分段单调连续,如果存在两系列上、下夹逼阶梯函数$\{g_n(x)\}$, $\{G_n(x)\}$, 使得
\[g_n (x)\le f(x)\le G_n(x),\qquad a\le x\le b\]
并且$g_n(x)$从$a$到$b$的和$s_n$与$G_n(x)$从$a$到$b$的和$S_n$具有相同的
极限,这个极限叫做$f(x)$从$a$到$b$的定积分$\Int^b_a f(x) \dd x$.

综上所述,在积分符号中,我们只对$a<b$, 即积分下限小于积分
上限的情形给出了定义,若$a>b$,我们定义
\[\int^a_b f(x) \dd x=-\int^b_a f(x) \dd x\]
此外,由于定积分可以解释为曲边梯形的面积,自然可以定
\[\int^a_a f(x) \dd x=0\]
作了这样的规定之后,不论$a<b$, $a>b$或$a=b$, 定积分都有意义了。


\begin{ex}
\begin{enumerate}
    \item 求证 $\Int_{a}^{b} k x\dd x=\frac{k b^{2}}{2}-\frac{k a^{2}}{2} \quad(a<b)$
    \item  求 $\Int_{8}^{0}\left(x^{2}-4 x\right) \dd x$
    \item  求 $\Int_{-1}^{2}\left(x^{3}-3 x\right) \dd x$
\end{enumerate}
\end{ex}

\subsection{逼近法求曲线形的面积}

任意一条曲线围成的图形(图3.11)常常可以用两组互相垂直的直线把它分成若干部分,每一部分都是一个曲边梯形(图3.12), 在这里并不排除下述情形(图3.13):两条平行的边中有一条缩成了一点,因而曲边梯形变成了曲边三角形,这样一来,我们的问题就化成了求曲边梯形面积的问题。

\begin{figure}[htp]
    \centering
    \begin{minipage}[t]{0.45\textwidth}
        \centering
        \begin{tikzpicture}[>=latex, scale=1]
    \draw(-2,3)--(2,3);
    \draw(-2,2)--(2,2); 
    \draw(-.5,0)--(-.5,4);
    \draw(.5,0)--(.5,4); 
    \node at (0,0){
\pscurve(-0.32,3.56)(-0.82,3.36)(-1.12,2.9)(-1.42,1.72)(-1.26,0.9)(-1.08,0.78)(-0.64,0.64)(-0.56,0.64)(-0.5,0.62)(-0.42,0.62)(-0.36,0.6)(-0.3,0.58)(0.1,0.5)(0.18,0.5)(0.24,0.52)(0.48,0.7)(0.64,0.86)(0.7,0.9)(1.36,1.62)(1.42,1.7)(1.36,3.16)(0.96,3.56)(0.84,3.66)(0.78,3.7)(0.6,3.8)(0.54,3.82)(0.34,3.82)(-0.32,3.56)
};

        \end{tikzpicture}
        \caption{}
        \end{minipage}
    \begin{minipage}[t]{0.25\textwidth}
    \centering
    \begin{tikzpicture}[>=latex, scale=1]
    \draw(0,1.3)--(0,0)--(2.5,0)--(2.5,1.8);
    \node at (0,0){
\pscurve(0,1.3)(1.4,2)(2.5,1.8)
};
    \end{tikzpicture}
    \caption{}
    \end{minipage}
    \begin{minipage}[t]{0.25\textwidth}
    \centering
    \begin{tikzpicture}[>=latex, scale=1]
    \draw(0,2)--(0,0)--(1.5,0);
    \node at (0,0){
\pscurve(0,2)(0.5,1.8)(1,1.3)(1.5,0)
};
    \end{tikzpicture}
    \caption{}
    \end{minipage}
  \end{figure}

\begin{blk}
  {命题} 设$y=f(x)$是定义在$a\le x\le b$的分段单调连续函数,而且$f(x)\ge 0$, 则区域$R=\{a\le x\le b,\; 0\le y\le f(x)\}$的
面积等于
\[\int^b_a f (x) \dd x \]  
\end{blk}

\begin{figure}[htp]
    \centering
\begin{tikzpicture}[>=latex]
    \draw[->, thick](-.5,0)--(7,0)node[right]{$x$};
    \draw[->, thick](0,-.5)--(0,5)node[right]{$y$};
\draw[domain=.5:6, samples=100, very thick]plot(\x, {1.5*sin(1.2*(\x-1) r)+3});
\draw(.5,0)node[below]{$a$} rectangle  (1.1, .18+3);
\draw(1.1,0) rectangle  (1.8, 1.23+3);
\draw(1.8,0) rectangle  (2.5, 1.46+3);
\draw[pattern=north east lines](2.5,0) rectangle  (3.5, .212+3);
\draw[pattern=north east lines](3.5,0) rectangle  (4.2, -.965+3);
\draw[pattern=north east lines](4.2,0) rectangle  (5, -1.49+3);
\draw(5,0) rectangle  (5.5, -1.16+3);
\draw(5.5,0) rectangle (6, -.419+3);

\draw[pattern=north east lines](.5,0) rectangle  (1.1, -.847+3);
\draw[pattern=north east lines](1.1,0) rectangle  (1.8, .18+3);
\draw[pattern=north east lines](1.8,0) rectangle  (2.5, 1.23+3);
\draw(2.5,0) rectangle  (3.5, 1.46+3);
\draw(3.5,0) rectangle  (4.2, .212+3);
\draw(4.2,0) rectangle  (5, -.965+3);
\draw[pattern=north east lines](5,0) rectangle  (5.5,-1.49+3);
\draw[pattern=north east lines](5.5,0) rectangle (6, -1.16+3);

\node at (6,0)[below]{$b$};
\node at (0,0)[below left]{$O$};
\node at (4.5,3.2)[above]{$y=f(x)$};

\end{tikzpicture}
    \caption{}
\end{figure}


\begin{proof}
    由$\Int^b_a f (x) \dd x$的定义得知,存在两系列阶梯函数
$\{g_n(x)\}$和$\{G_n(x)\}$, 满足下面的性质:
\[g_n(x)\le f (x) \le G_n(x)\]
而且$G_n(x)$从$a$到$b$的和与$g_n(x)$从$a$到$b$的和趋于共同的极限$\Int^b_a f (x) \dd x$. 令
\[\begin{split}
    R_n&=\{a\le x\le b,\; 0\le y\le g_n(x)\}\\
    R'_n&=\{a\le x\le b,\; 0\le y\le G_n(x)\}\\
\end{split}\]
它们都是由高高低低的狭长方形所组成的区域(图3.14),则:
\[R_n\subset \text{曲边梯形区域 }R \subset R'_n\]
而且
\[R_n\text{的面积}=g_n(x)\text{从$a$到$b$的和}s_n,\qquad R'_n\text{的面积}=G_n(x)\text{从$a$到$b$的和}S_n\]
即有
\[s_n<\text{曲边梯形区域$R$的面积}A<S_n\]
由定积分定义有:
\[\lim_{n\to\infty}s_n=\lim_{n\to\infty}S_n=\Int^b_a f (x) \dd x\]
又曲边梯形面积$A=(f,\; [a,b])$也是夹逼在$s_n$和$S_n$中间的实数,所以
\[A=(f,\; [a,b])=\Int^b_a f (x) \dd x\]
\end{proof}

\begin{ex}
\begin{enumerate}
    \item 求$\Int^1_{-2}e^x\dd x$. 

提示:$\frac{e^{\Delta x}-1}{\Delta x}=1$.    

\item 求在$x=0$与$x=\pi$间正弦曲线$y=\sin x$与$Ox$轴所包的面积。
\end{enumerate}
\end{ex}    

\subsection{定积分的基本性质}

为简单起见,我们约定以下被积函数在积分区间上连续并且可以分段单调,即$f(x)$是在$[a,b]$上连续并且只有有限个极大值和极小值的函数。

\begin{blk}{性质1}
若$a<b<c$,那么,
\[\int^b_a f(x)\dd x+\int^c_b f(x)\dd x=\int^c_a f(x)\dd x\]
\end{blk}

\begin{proof}
因为$\Int^b_a f(x)\dd x$存在,故对于在$[a,b]$上的
$f(x)$, 存在一组上、下夹逼函数列$\{g_n(x)\}$和$\{G_n(x)\}$, 使得
\[g_n (x) \le f (x) \le G_n (x),\qquad x\in [a,b]\]
且相应的阶梯函数在$[a,b]$上的和$s_n$与$S_n$适合
\begin{equation}
    s_n<\int^b_a f(x)\dd x<S_n,\qquad S_n-s_n\to 0
\end{equation}
对于在$[b,c]$上的$f(x)$, 同样得到
\[\bar g_n (x) \le f (x) \le \bar G_n (x),\qquad x\in [b,c]\]
且
\begin{equation}
    \bar  s_n<\int^c_b f(x)\dd x<\bar S_n,\qquad \bar S_n-\bar s_n\to 0
\end{equation}
$(3.4)+(3.5)$得到
\begin{equation}
    s_n+ \bar  s_n<\int^b_a f(x)\dd x+\int^c_b f(x)\dd x<S_n+\bar S_n
\end{equation}

另一方面,对于函数$f(x)\; (a\le x\le c)$存在下面一组阶梯函数列:
\[\tilde g_n(x)=\begin{cases}
    g_n(x),  &  a\le x\le b\\
    \max\left(g_n(b), \bar g_n(b)\right),  &  x=b\\
    \bar g_n(x), & b<x\le c
\end{cases}\]
\[\tilde G_n(x)=\begin{cases}
    G_n(x),  &  a\le x\le b\\
    \max\left(G_n(b), \bar G_n(b)\right),  &  x=b\\
    \bar G_n(x), & b<x\le c
\end{cases}\]

由性质2,阶梯函数列$\tilde g_n(x)$与$\tilde G_n(x)$的从$a$到$c$的和分别是$s_n+\bar s_n$和$S_n+\bar S_n$。因为$\tilde g_n(x)$与$\tilde G(x)$满足条件:
\begin{enumerate}
    \item $\tilde g_n(x)\le f(x)\le \tilde G(x),\quad a\le x\le c$
    \item \[\begin{split}
    \lim_{n\to\infty}\left[(S_n+\bar S_n)-(s_n+\bar s_n)\right]&=\lim_{n\to\infty}\left[(S_n-s_n)+(\bar S_n-\bar s_n)\right]\\
    &=\lim_{n\to\infty}(S_n-s_n)+\lim_{n\to\infty}(\bar S_n-\bar s_n)=0
    \end{split}\]
\end{enumerate}
所以根据定积分定义得到
\[\lim_{n\to\infty}(S_n+\bar S_n)=\lim_{n\to\infty}(s_n+\bar s_n)=\int^c_a f(x)\dd x \]

因为$\Int^c_a f(x)\dd x $与$\Int^b_a f(x)\dd x +\Int^c_b f(x)\dd x $都是被$s_n+\bar s_n$与$S_n+\bar S_n$所夹逼的唯一实数,所以
\[\Int^c_a f(x)\dd x=\Int^b_a f(x)\dd x +\Int^c_b f(x)\dd x\]
\end{proof}

\begin{blk}{性质2}
若$f(x)=f_1(x)+f_2(x)$,则:
\[\Int^b_a f(x)\dd x=\Int^b_a f_1(x)\dd x +\Int^b_a f_2(x)\dd x\]
\end{blk}

\begin{proof}
设$\{g_n(x)\}$与$\{G_n(x)\}$是$f_1(x)$在$[a,b]$上的上、下夹逼阶梯函数列,又$\{\bar g_n(x)\}$与$\{\bar G_n(x)\}$是$f_2(x)$在$[a,b]$上的上、下夹逼阶梯函数列。$s_n, S_n, \bar s_n, \bar S_n$是相应的阶梯函数从$a$到$b$的和。于是由
$\Int^b_a f_1(x)\dd x$和$\Int^b_a f_2(x)\dd x$的存在,得:
\begin{align}
g_n(x)\le f_1(x)\le G_n(x),\qquad a\le x\le b\\
\bar g_n(x)\le f_2(x)\le \bar G_n(x),\qquad a\le x\le b
\end{align}
并且
\begin{align}
s_n<\Int^b_a f_1(x)\dd x<S_n,\quad \text{且 } S_n-s_n\to 0\\
\bar s_n<\Int^b_a f_2(x)\dd x<\bar S_n,\quad \text{且 } \bar S_n-\bar s_n\to 0
\end{align}
由$(3.9)+(3.10)$,得到
\begin{equation}
   s_n+ \bar s_n <\Int^b_a f_1(x)\dd x+\Int^b_a f_2(x)\dd x<S_n+\bar S_n
\end{equation}
另一方面,由$(3.7)+(3.8)$,得到
\[g_n(x)+\bar g_n(x)\le f_1(x)+ f_2(x)\le G_n(x)+\bar G_n(x),\qquad a\le x\le b\]
而且,$g_n(x)+\bar g_n(x)$与$G_n(x)+\bar G_n(x)$也是在$[a,b]$上的阶梯函数,我们要说明它们是$f_1(x)+ f_2(x)$在$[a,b]$上的一组夹逼阶梯函数列。

因为:
\[\begin{split}
    g_n(x)+\bar g_n(x)\text{从$a$到$b$的和}&=\left[ g_n(x)\text{从$a$到$b$的和}\right]+\left[\bar g_n(x)\text{从$a$到$b$的和}\right]\\
    &=s_n+\bar s_n\\
    G_n(x)+\bar G_n(x)\text{从$a$到$b$的和}&=\left[ G_n(x)\text{从$a$到$b$的和}\right]+\left[\bar G_n(x)\text{从$a$到$b$的和}\right]\\
    &=S_n+\bar S_n
\end{split}\]
而且
\begin{equation}
    \left(S_n+\bar S_n\right)-\left(s_n+\bar s_n\right)=\left(S_n-s_n\right)+\left(\bar S_n-\bar s_n\right)\to 0
\end{equation}
所以,由$\Int^b_a \bigl(f_1(x)+ f_2(x)\bigr)\dd x$的定义,得到
\begin{equation}
    \lim_{n\to\infty}\left(S_n+\bar S_n\right)=\lim_{n\to\infty}\left(s_n+\bar s_n\right)=\Int^b_a \bigl(f_1(x)+ f_2(x)\bigr)\dd x
\end{equation}
由(3.10)和(3.13),$\Int^b_a \bigl(f_1(x)+ f_2(x)\bigr)\dd x$与$\Int^b_a f_1(x)\dd x + \Int^b_a  f_2(x)\dd x$
是被$s_n+\bar s_n$与$S_n+\bar S_n$所夹逼的唯一实数,所以
\[\Int^b_a \bigl(f_1(x)+ f_2(x)\bigr)\dd x=\Int^b_a f_1(x)\dd x + \Int^b_a  f_2(x)\dd x\]
\end{proof}

\begin{blk}{性质3}
\[\Int^b_a kf(x)\dd x =k\Int^b_a f(x)\dd x \]
\end{blk}

此法则的证明大致和性质2的证明相同,即当$\{g_n(x)\}$和$\{G_n(x)\}$上、下夹逼$f(x)$时,那么在$k>0$的情形,$\{kg_n(x)\}$和$\{kG_n(x)\}$上、下夹逼$kf(x)$; 在$k<0$的情形,$\{kg_n(x)\}$和$\{kG_n(x)\}$上、下夹逼$kf(x)$. 我们把证明的过程留给读者去写。

\begin{blk}{性质4}
    若$m\le f(x)\le M$, $a\le x\le b$,那么
    \[m(b-a)\le \Int^b_a f(x)\dd x \le M(b-a) \]
    \end{blk}

\begin{proof}
    设$\{g_n(x)\}$与$\{G_n(x)\}$是$f(x)$在$[a,b]$上的夹逼阶梯函数列,$s_n$与$S_n$是相应的阶梯函数在$[a,b]$上的和,根据积分的定义,有
\[\lim_{n\to\infty}s_n=\lim_{n\to\infty}S_n=\int^b_a f(x)\dd x\]
换言之,任给一个正数$\varepsilon$,存在自然数$N$,使得当$n<N$时,有
\begin{equation}
    S_n<\int^b_a f(x)\dd x+\varepsilon
\end{equation}
成立。

又由于$G_n(x)\ge f(x)\ge m,\quad x\in[a,b]$,根据性质1,得到:
\begin{equation}
    S_n\ge m(b-a)
\end{equation}
所以,当$n>N$时,由(3.14)和(3.15),有
\[m(b-a)\le S_n<\int^b_a f(x)\dd x+\varepsilon\]
即
\begin{equation}
    m(b-a)<\int^b_a f(x)\dd x+\varepsilon
\end{equation}
成立。

因为这个不等式(3.16)对于每个正数$\varepsilon$都成立,所以
\[m(b-a)\le \int^b_a f(x)\dd x\]
同样证明,得到
\[M(b-a)\ge \int^b_a f(x)\dd x\]
因此
\[m(b-a)\le \int^b_a f(x)\dd x\le M(b-a)\]
\end{proof}

\begin{figure}[htp]
    \centering
\begin{tikzpicture}[>=latex]
\draw[->, thick](-1.5,0)--(3,0)node[right]{$x$};
\draw[->, thick](-1,-.5)--(-1,4)node[right]{$y$};
\node at (-1,0)[below left]{$O$};
\draw[domain=-.1:2.3, samples=50, very thick]plot(\x, {\x*(\x-1)*(\x-2)+2});
\draw(-.1,0)node[below]{$a$} rectangle (2.3,2.9);
\node at (2.3,0)[below]{$b$};
\draw(1.577,0)node[below]{$x_0$}--(1.577,1.615);
\draw(-.1,1.615)--(2.3,1.615);
\draw[dashed](-1,1.615)node[left]{$m$}--(-.1,1.615);
\draw[dashed](-1,2.9)node[left]{$M$}--(-.1,2.9);

\node at (1.1,2.5){$y=f(x)$};
\end{tikzpicture}
    \caption{}
\end{figure}

从几何图形来看,以曲线$y=f(x)$为一边而以线段$[a,b]$为底边的曲边梯形界于以$[a,b]$为底边,高分别等于$m$和$M$的两个矩形之内,
故曲边梯形的面积$\Int^b_a f(x)\dd x$
在两个矩形面积$m(b-a)$和$M(b-a)$之间(图3.14)。

这个性质是说分段单调的连续函数$f(x)$在区间$[a,b]$上的定积分是有界的。

\begin{blk}{性质5}
如果$f(x)\le \varphi(x),\; a\le x\le b$,那么:
\[\int^b_a f(x)\dd x\le \int^b_a \varphi(x)\dd x\]
\end{blk}

\begin{proof}
    设$\{g_n(x)\}$与$\{G_n(x)\}$是$f(x)$在$[a,b]$上的夹逼阶梯函数列,又$\{\bar g_n(x)\}$与$\{\bar G_n(x)\}$是$\varphi(x)$在$[a,b]$上的夹逼阶梯函数列,
又$s_n, S_n, \bar s_n, \bar S_n$是相应的阶梯函数从$a$到$b$的和。于是根据定积分的定义,有:
\[s_n<\int^b_a f(x)\dd x<S_n,\qquad \bar s_n<\int^b_a \varphi(x)\dd x<\bar S_n\]
当$n\to\infty$时,得到:
\[\begin{split}
\lim_{n\to\infty}S_n =\lim_{n\to\infty}s_n =\int^b_a f(x)\dd x\\
\lim_{n\to\infty}\bar S_n =\lim_{n\to\infty}\bar s_n =\int^b_a \varphi(x)\dd x\\
\end{split}\]
换言之,任给正数$\varepsilon$,存在$N_1$,使得当$n>N_1$时,有
\[s_n>\int^b_a f(x)\dd x-\frac{\varepsilon}{2}\]

当$n>N_2$时,有
\[\bar s_n<\int^b_a \varphi(x)\dd x+\frac{\varepsilon}{2}\]

因为$\bar G_n(x)>\varphi(x)>f(x)>g_n(x),\quad x\in[a,b]$,根据性质1,得到:$\bar S_n=s_n$.

所以,当$n>\max(N_1,N_2)$时,有
\[\int^b_a f(x)\dd x<s_n+\frac{\varepsilon}{2}<\bar S_n+\frac{\varepsilon}{2}<\int^b_a \varphi(x)\dd x+\varepsilon\]
因为这个不等式对于每个正数$\varepsilon$都成立,所以必然有
\[\int^b_a f(x)\dd x\le \int^b_a \varphi(x)\dd x\]
\end{proof}

\begin{blk}{性质6:定积分中值定理}
   设函数$f(x)$在闭区间$[a,b]$上分段单调连续,又
$m=f(c)$, $M=f(d)$分别是$f(x)$在$[a,b]$上的最小值和最大值,则在$[a,b]$上至少存在一点$\xi$, 使得下面的等式成立:
\[\int^b_a f (x) \dd x=f(\xi) (b-a)\] 
\end{blk}

\begin{proof}
    因为$m\le f (x) \le M,\quad x\in [a,b]$, 
由性质4, 即得
\[m (b-a)\le \int^b_a f (x) \dd x\le M(b-a)\]
上面不等式的两端各除以$(b-a)$, 得
\[f(c)=m\le \frac{\int^b_a f (x) \dd x}{b-a}\le M=f(d)\]
因为$f(x)$在$[a,b]$上连续,再由连续函数的中间值定理,必存在一个$\xi\in(c,d)$, 使得
\[f(\xi)=\frac{1}{b-a}\int^b_a f (x) \dd x\]
两边再乘以$(b-a)$,得
\[\int^b_a f (x) \dd x=f(\xi)(b-a)\]
这就是我们所要证明的。
\end{proof}

这个公式的几何意义是:以线段$[a,b]$为底边,以曲线$y=f(x)$为曲边的曲边梯形,它的面积等于同一底边
而高为$f(\xi)$的一个矩形的面积(图3.15)。因此,$f(\xi)$称为曲边梯形的平均高度。

我们也称
\[f(\xi)=\frac{1}{b-a}\int^b_a f (x) \dd x\]
为$f(x)$在$[a,b]$上的平均值。

\begin{figure}[htp]
    \centering
    \begin{minipage}[t]{0.48\textwidth}
    \centering
    \begin{tikzpicture}[>=latex, xscale=.8]
\draw[->, thick](-.5,0)--(5,0)node[right]{$x$};
\draw[->, thick](0,-.5)--(0,4)node[right]{$y$};
\node at (0,0)[below left]{$O$};
\draw[domain=.7:4, very thick, samples=100]plot(\x, {ln(\x)+1.5})node[above]{$y=f(x)$};
\draw[dashed](.7,0)node[below]{$a$} rectangle (4,2.3);
\draw[dashed](2.23,0)node[below]{$\xi$}--(2.23,2.3)node[below right]{$f(\xi)$};
\draw[dashed](4,2.89)--(4,2.3);
\node at (4,0)[below]{$b$};

    \end{tikzpicture}
    \caption{}
    \end{minipage}
    \begin{minipage}[t]{0.48\textwidth}
    \centering
    \begin{tikzpicture}[>=latex, scale=.5]
  \draw[->, thick](-2,0)--(6,0)node[right]{$x$};
\draw[->, thick](0,-8.5)--(0,2)node[right]{$y$};
\node at (0,0)[below left]{$O$};
\draw[domain=-1:4, very thick, samples=100]plot(\x, {2*\x-\x*\x});
\fill[pattern=north east lines, domain=2:4, very thick, samples=100]plot(\x, {2*\x-\x*\x})--(4,0)node[above]{4}
--(2,0)node[above]{2};
\draw(1,0)node[below]{1}--(1,.2);
    \end{tikzpicture}
    \caption{}
    \end{minipage}
  \end{figure}

\begin{example}
求$\Int^4_2 (2x-x^2)\dd x$.
\end{example}


\begin{solution}
    根据本节的定积分计算法则得:
\[\begin{split}
    \Int^4_2 (2x-x^2)\dd x&=\Int^4_2 2x\dd x+\Int^4_2 (-x^2)\dd x\\
    &=2\Int^4_2 x\dd x-\Int^4_2 x^2\dd x
\end{split}\]
利用例3.4的计算结果,得
\[\Int^4_2 x\dd x=\frac{4^2-2^2}{2}=6,\qquad \Int^4_2 x^2\dd x=\frac{4^3-2^3}{3}=18\frac{2}{3}\]
因此
\[\Int^4_2 (2x-x^2)\dd x=2\x 6-18\frac{2}{3}=-6\frac{2}{3}\]
它的几何意义是图3.16中阴影区域的有号面积。
\end{solution}

\begin{example}
  设$f(x)=\begin{cases}
      x^2,& 0\le x\le 1\\
      1,& 1<x\le 2
  \end{cases}$,  求$f(x)$在$[0,2]$上的平均值.
\end{example}

\begin{solution}
由平均值定义,再利用性质1,有    
\[\begin{split}
    f(\xi)&=\frac{1}{b-a}\int^b_a f(x)\dd x=\frac{1}{2}\int^2_0 f(x)\dd x\\
    &=\frac{1}{2}\left[\int^1_0 f(x)\dd x+\int^2_1 f(x)\dd x\right]\\
    &=\frac{1}{2}\left[\int^1_0 x^2\dd x+\int^2_1 1\dd x\right]\\
    &=\frac{1}{2}\left[\frac{1}{3}+1\right]=\frac{2}{3}
\end{split}\]
\end{solution}

\begin{ex}
\begin{enumerate}
    \item 利用以前定积分的结果和定积分的性质,计算下列定积分:
\begin{enumerate}
    \item $\Int^3_{-1}(2x^2-4x)\dd x$
    \item 若$f(x)=\begin{cases}
        x,& 0\le x<1\\
        x-2,& 1\le x\le 2
    \end{cases}$,则$\Int^2_0 f(x)\dd x=?$
    \item 若$f(x)=\begin{cases}
        x,& 0\le x\le 1\\
        x-2,& 1< x\le 2
    \end{cases}$,则$\Int^2_0 f(x)\dd x=?$
\item 若$g(t)=2t^2+|t|-1,\; -1\le t\le 1$,则$\Int^1_{-1} g(t)\dd t=?$
\end{enumerate}

\item 将图中阴影部分的面积用定积分表示。
\item 设$\ell(t)=mt+b$, ($m,t$是常数),$t\in [c,d]$,

求证:$\Int^d_c (mt+b)\dd t=\frac{\ell(c)+\ell(d)}{2}(d-c)$
\item 证明:
\begin{enumerate}
    \item 若$0<x<10$,则$\frac{1}{1016}\le\frac{1}{x^3+16}\le \frac{1}{16}$
    \item $\frac{5}{508}\le \Int^{10}_0\frac{1}{x^3+16}\dd x\le \frac{5}{8}$
\end{enumerate}
\end{enumerate}
\end{ex}

\begin{figure}[htp]
    \centering
\begin{tikzpicture}[>=latex]
    
\begin{scope}[scale=.6]
\draw[->](-2,0)--(4,0)node[right]{$x$};
\draw[->](0,-3)--(0,7)node[right]{$y$};
\fill[domain=-1:3, samples=50, pattern=north east lines]plot(\x,{2*(\x-1)^2-2})--(3,0)--(-1,0)--(-1,6);
\draw[domain=-1:3, samples=50, very thick]plot(\x,{2*(\x-1)^2-2})node[above]{$y=2x^2-4x$};
\node at (0,0)[above right]{$O$};
\node at (-1,0)[below]{$-1$};
\node at (3,0)[below]{$3$};
\node at (-1,3)[left]{$x=-1$};
\node at (3,3)[right]{$x=3$};
\node at (1,-2)[below]{$(1,-2)$};
\node at (1,-3.8){(a)};
\end{scope}
\begin{scope}[xshift=5cm, yshift=-.8cm, scale=.9]
\draw[->](-1,0)--(4,0)node[right]{$x$};
\draw[->](0,-1)--(0,5.5)node[right]{$y$};
\node at (0,0)[below left]{$O$};
\draw[domain=0:2.2, samples=50, very thick]plot(\x,{\x^2})node[above]{$y=x^2$};
\draw[thick](0,0)--(3.5,3.5)node[right]{$y=x$};
\node at (1,0)[below]{$1$};
\node at (2,0)[below]{$2$};
\draw[dashed](0,1)node[left]{1}--(1,1)--(1,0);
\fill[domain=0:1, samples=50, pattern=north east lines]
plot(\x,{\x^2})--(2,2)--(2,0)--(0,0);
\node at (2,-1.5){(b)};
\end{scope}

\end{tikzpicture}
    \caption*{第2题}
\end{figure}



\subsection*{习题3.2}
\begin{enumerate}
    \item 计算下列定积分:
\begin{multicols}{3}
\begin{enumerate}
    \item $\Int^1_4 |x|\dd x$
    \item $\Int^2_4 |x|\dd x$
    \item $\Int^4_1 |x|\dd x$
\end{enumerate}
\end{multicols}
\item 求$\Int^b_a |x|\dd x$.

提示:分$a<b<0$, $a<0<b$, $0<a<b$三种情况讨论。
\item 证明:
\begin{enumerate}
    \item 若$\frac{\pi}{4}\le x\le\frac{\pi}{2}$,则$\frac{2}{\pi}\le \frac{\sin x}{x}\le \frac{2\sqrt{2}}{\pi}$
    \item $\frac{1}{2}\le \Int^{\tfrac{\pi}{2}}_{\tfrac{\pi}{4}}\frac{\sin x}{x}\dd x\le \frac{\sqrt{2}}{2}$
\end{enumerate}
\item 已知作用在作直线运动的质点上的力是$F=s^2+1$, 试求从距离1到10之间所作的功.
\item 求$y=\begin{cases}
    A\sin\frac{2\pi t}{T}, & 0\le t\le \frac{T}{2}\\
    0,& \frac{T}{2}\le t\le T
\end{cases}$ 在$[0,T]$上的平均值。
\end{enumerate}
\chapter{微积分学基本定理}
在前两章中,我们分别引入了函数的变率(导数),函数的和(定积分)的基本概念,本章将研究函数的导函数与函数的求和函数这两者之间的互逆关系,并说明我们可以用求导函数的逆运算方法来计算定积分.

\section{微积分学基本定理}
\subsection{导函数与求和函数}
函数$f(x)$在点$x_0$处的变化率(导数)的定义是
\[f' (x_0) =\lim_{h\to 0}\frac{f (x_0+h) -f (x_0)}{h}\]
显然.$f(x_0)$的值与$f(x)$在点$x_0$的值以及在点$x_0$的邻近的函数值有关,当点$x_0$在$(a,b)$内变化时,$f(x_0)$也跟着变化,那么$f(x_0)$便是一个新函数称为$f(x)$的导函数.计算一个函数的导函数是一件比较简便的事情.

定积分$\int^b_a f(x)\dif x$的定义是把区间$[a,b]$无限细分
而得到上下夹逼阶梯函数的和的共同极限,其几何意义是曲线$y=f(x)$和直线$x=a$, $x=b$, 及$y=0$所围成的区域的有号面积.

假如我们考虑$f(x)$在一个变动的区间$[a,x]$上的和,
即让区间的左端点固定,右端点变动,则
\[S_f(x)=\text{函数$f$从$a$到$x$的和}=\int^x_a f (x) \dif x\]
可以看作上限变量$x$的函数,在这里,积分符号中的$x$既表示积分变量,又表示积分上限,容易混淆,因此,为了区别起见,我们用字母$t$来代表积分变量,这样上式就写成
\[S_f (x)=\int^x_a f(t)\dif t\]

和函数$S_f(x)$在$x=x_0$处的值$S_f(x_0)$的几何意义就是曲线$y=f(t)$, 直线$t=a$, $t=x_0$, $y=0$所围成的区域的有号面积,它是随区域的变动界线$t=x_0$的变动而变动的(图4.1).

\begin{figure}[htp]
    \centering
    \begin{tikzpicture}[>=latex]
        \draw[->](-3,0)--(5,0)node[right]{$t$};
        \draw[->](0,-1.5)--(0,2)node[right]{$y$};
    \node at (0,0) [below left]{$O$};
    \node at (0,0){
        \pscurve[linewidth=1.5pt] (-2.75,-.5)(-2,-1)(-.8,0)(1,1.25)(2.5,0)(3.5,-1.1)(4,-1.3) 
    };
    \draw(-2,-1)node[below]{$M'$}--node[left]{$t=x_0'$}(-2,0)node[above]{$P'$};
    \draw(1,1.25)node[above]{$B$}--node[right]{$t=a$}(1,0)node[below]{$A$};
    \draw(3.5,-1.1)node[below]{$M$}--node[right]{$t=x_0$}(3.5,0)node[above]{$P$};
    \node at (-.8,0)[above]{$D$};
    \node at (2.5,0)[above]{$C$};
    \node at (2,1)[right]{$y=f(t)$};
    
    \end{tikzpicture}
    \caption{}
\end{figure}

例如,当变动界限(积分上限)在图中的$PM$位置时,则
\[S_f(x_0)=\int^{x_0}_a f(t)\dif t=ACB\text{的面积}-CPM\text{的面积}\]

当变动界限在图中的$P'M'$位置时,则
\[\begin{split}
    S_f(x'_0)&=\int^{x_0'}_a f(t)\dif t=-\int^a_{x_0'} f(t)\dif t\\
    &=-\left\{-P'DM'\text{的面积}+DAB\text{的面积}  \right\}\\
    &=P'DM'\text{的面积}-DAB\text{的面积}
\end{split}\]

例如,折线函数
\[g(x)=\begin{cases}
    \frac{1}{2}(x-1) & x\in[1,2]\\
    \frac{3}{2}(x-2)+\frac{1}{2}(2-1) & x\in [2,3]\\
    (x-3)+\frac{1}{2}(2-1)+\frac{3}{2}(3-2) & x\in [3,4]
\end{cases}\]
的导函数(除去在折线段的那些交接点处不作定义外)是阶梯函数
\[g'(x)=\begin{cases}
    \frac{1}{2} & x\in [1,2)\\
    \frac{3}{2} & x\in (2,3)\\
    1& x\in (3,4]
\end{cases}\]
反过来,该阶梯函数的和函数,是上述折线函数$g(x)$, 我们有如下的图解关系:
\begin{center}
    \begin{tikzpicture}[>=latex]
        \node (A) at (0,0) {\{折线函数\}};
        \node (B) at (5,0) {\{阶梯函数\}};
\draw[->](A) to [bend left=30]node[above]{求导} (B);
\draw[->](B) to [bend left=30]node[above]{求和} (A);

    \end{tikzpicture}
\end{center}

上述简明的例子表明“微分”与“积分”(或求函数由$a$到$b$的和)之间的运算关系应该是互逆的.

\subsection{微积分学基本定理}

\begin{blk}
  {定理1} 设$f(t)$是在$[a,b]$上的分段单调连续函数,又它的和函数是
\[S_f(x)=\int^x_a f (t) \dif t\]
那么
\[\frac{\dif }{\dif x}S_f(x)=\frac{\dif }{\dif x}\int^x_a f (t) \dif t=f(x),\qquad a\le x\le b\]
\end{blk}

\begin{figure}[htp]
    \centering
    \begin{tikzpicture}[>=latex, xscale=3]
        \draw[->](.4,0)--(3,0)node[right]{$t$};
        \draw[->](.6,-.5)--(.6,4.5)node[right]{$y$};
    \draw[domain=.9:2.5, very thick, samples=100]plot(\x, {2.5*(\x-1)*(\x-2)*(\x-3)+2})node[right]{$y=f(t)$};
    \draw(.9,0)node[below]{$a$}--(.9,1.4225);
    \draw[pattern=north east lines, domain=1.4:1.7, samples=50]plot(\x, {2.5*(\x-1)*(\x-2)*(\x-3)+2})--(1.7,2.6825)--(1.7,0)node[below]{$x$}--(1.4,0)node[below]{$x+\bar h$}--(1.4,2.96);
    \draw[pattern=north west lines, domain=1.7:2, samples=50]plot(\x, {2.5*(\x-1)*(\x-2)*(\x-3)+2})--(2,2)--(2,0)node[below]{$x+h$}--(1.7,0)--(1.7,2.6825);
    \draw[<-](1.55,2)--(1.55,3.5)node[right]{$S_f(x)-S_f(x+\bar h),\; (\bar h<0)$};
    \draw[<-](1.85,.5)--(2.1,.5)node[right]{$S_f(x+h)-S_f(x),\; (h>0)$};
    \node at (.6,0)[below left]{$O$};
    \end{tikzpicture}
    \caption{}
\end{figure}

\begin{proof}
  如图4.2,设$h>2$,则
\[\begin{split}
    \frac{S_f(x+h)-S_f(x)}{h}&=\frac{\Int^{x+h}_a f(t)\dd t-\Int^{x}_a f(t)\dd t}{h}\\
    &=\frac{\Int^{x+h}_x f(t)\dd t}{h}\qquad \text{(定积分性质1)}\\
    &=\frac{hf(\xi)}{h}\quad (x<\xi<x+h)\\
    &=f(\xi)\qquad \text{(定积分中值定理)}
\end{split}\]
于是
\begin{equation}
    \lim_{h\to 0^+}\frac{S_f(x+h)-S_f(x)}{h}=\lim_{\xi\to x}f(\xi)=f(x)
\end{equation}

设$\bar h<0$,则
\[\begin{split}
    \frac{S_f(x)-S_f(x+\bar h)}{\bar h}&=\frac{\Int^{x}_a f(t)\dd t-\Int^{x+\bar h}_a f(t)\dd t}{\bar h}\\
    &=\frac{\Int^{x}_{x+\bar h} f(t)\dd t}{\bar h}\qquad \text{(定积分性质1)}\\
    &=\frac{-\bar h f(\bar\xi)}{\bar h}=-f(\bar\xi)\quad (x+\bar h<\bar \xi<x)\\
\end{split}\]
即:
\[\frac{S_f(x+\bar h)-S_f(x)}{\bar h}=f(\bar\xi),\qquad (x+\bar h<\bar \xi<x)\]
于是
\begin{equation}
    \lim_{h\to 0^-}\frac{S_f(x+\bar h)-S_f(x)}{\bar h}=\lim_{\bar\xi\to x}f(\bar\xi)=f(x)
\end{equation}
由(4.1)和(4.2),得
\[\frac{\dd }{\dd x}S_f(x)=\frac{\dd }{\dd x}\int^x_a f(t)\dd t=f(x)\]
\end{proof}

\begin{blk}
    {引理} 如果两个函数$F(x)$和$G(x)$满足条件
\[F' (x) =G' (x) ,\qquad a\le x\le b\]
那么
\[G(x)=F(x)+C,\qquad (C\text{是常数})\]
\end{blk}

\begin{proof}
    由$G'(x)-F'(x)=0$, 得
\[[G (x) -F (x) ]'=0\]
根据中值定理(参看第三章)知,对于每一个$x\in (a,b)$, 恒有
\[G(x)-F(x)=C,\qquad (C\text{是常数})\]
所以
\[G (x) =F (x) +C\]
\end{proof}

\begin{blk}
 {定义} 设$f(x)$是在区间$[a,b]$上给定的函数,如果可微函数$G(x)$满足条件
\[G'(x)=f(x)\qquad (\text{或}\quad \dd G(x)=f(x)\dd x)\]
那么就称$G(x)$为$f(x)$在$[a,b]$上的一个\textbf{原函数}.
\end{blk}

例如,设$G'(x)=\cos x$, 那么$G(x)=\sin x$就称为$\cos x$的一个原函数.

\begin{blk}
    {定理2} 设$f(x)$是给定的分段单调连续函数,如果一个可微函数$G(x)$满足条件
\begin{equation}
    G'(x)=f(x)\qquad \text{(或$\dd G(x)=f(x)\dd x$)},\quad a\le x\le b
\end{equation}    
那么
\[S_f (x)=\int^x_a  f (t) \dd t=G (x) -G (a) \]
\end{blk}

\begin{proof}
    由定理1知,$f(x)$的一个原函数存在,就是它的和函数,即
\[\frac{\dd }{\dd x}\int^x_a f(t)\dd t=f(x)\]
题设$G(x)$是$f(x)$的另一个原函数,故由引理得
\begin{equation}
    \int^x_a    f (t) dt=G (x) +C 
\end{equation}
这里的常数要由条件$\Int^a_a f(t)\dd t=0$来确定.

在(4.4)中,令$x=a$,得
\[0=G(a)+C\]
所以:$C=-G(a)$,代入(4.4),得
\[\int^x_a f(t)\dd t=G(x)-G(a)\]
\end{proof}

在上面等式中,令$x=b$, 便得到下面著名的莱布尼兹-牛顿公式:
\[\int^b_a f (t) \dd t= G (b) -G (a) \]
这个公式告诉我们,求分段单调连续函数$f(x)$在$[a,b]$上的定积分可以化简为去求$f(x)$的一个原函数$G(x)$, 而函数值的差$G(b)-G(a)$就是定积分的值.

\begin{example}
    设$f(x)=k$(常数函数),则$G(x)=kx$是一个满足$G'(x)=k=f(x)$的原函数,所以
\[\int^b_a k\dd x=kb-ka=k(b-a)\]
以后我们引用一个符号:
\[G(x)\Bigg|^b_a=G(b)-G(a)\]
\end{example}

\begin{example}
    设$f(x)=kx^n,\; (n\in\mathbb{N})$,则
\[G(x)=k\frac{x^{n+1}}{n+1}\]
是满足$G'(x)=kx^n$的一个原函数,所以
\[\int^b_akx^n\dd x=k\frac{x^{n+1}}{n+1}\Bigg|^b_a=\frac{k}{n+1}(b^{n+1}-a^{n+1})\]
\end{example}

\begin{example}
设$f(x)=x^{-n}=\frac{1}{x^n}\; (x\ne 0,\; n\in\mathbb{N})$,则
\[G(x)=\frac{x^{-n+1}}{-n+1}\]
是满足$G'(x)=x^{-n}$的一个原函数.要特别注意定理2中$f(x)$是连续的条件,在任何包含0的区间上,函数$f(x)=\frac{1}{x^n}$无界,即有第二类间断点,如果$a,b$异号,则定积分$\Int^b_a x^{-n}\dd x$无意义,但若$a,b$同时为正或同时为负,则
\[\int^b_ax^{-n}\dd x=\frac{x^{-n+1}}{-n+1}\Bigg|^b_a=\frac{b^{-n+1}}{-n+1}-\frac{a^{-n+1}}{-n+1}\]
当然,又有$n\ne -1$时,上式才能成立.

如果$n=-1$,依复合函数求导法则,得
\[(\ln|x|)'=\frac{1}{|x|}\cdot |x|'\]
若$x>0$,则
\[(\ln|x|)'=\frac{1}{x}\cdot 1=\frac{1}{x}\]
若$x<0$,则
\[(\ln|x|)'=\frac{1}{-x}\cdot (-1)=\frac{1}{x}\]
总之
\[(\ln|x|)'=\frac{1}{x},\quad (x\ne 0)\]
因此
\[\begin{split}
\int^b_a \frac{1}{x}\dd x=\ln |x|\Bigg|^b_a &=\ln|b|-\ln |a|\\
&=\ln\left|\frac{b}{a}\right|=\ln\frac{b}{a}\quad (x\ne 0,\; ab>0)
\end{split}\]
\end{example}

\begin{example}
    求$\Int^3_1 |x-2|\dd x$
\end{example}

\begin{solution}
\[\begin{split}
    \Int^3_1 |x-2|\dd x&=\Int^2_1 -(x-2)\dd x+\Int^3_2 (x-2)\dd x\\
&=\left(-\frac{x^2}{2}+2x\right)\Bigg|^2_1+\left(\frac{x^2}{2}-2x\right)\Bigg|^3_2\\
&=\frac{1}{2}+\frac{1}{2}=1
\end{split}\]
\end{solution}

\begin{example}
求$\Lim_{n\to\infty}\frac{\sum^n_{i=1}i^p}{n^{p+1}}\quad (p\in \mathbb{N})$
\end{example}


\begin{solution}
\[\begin{split}
    \Lim_{n\to\infty}\frac{\sum^n_{i=1}i^p}{n^{p+1}}&=\Lim_{n\to\infty}\sum^n_{i=1}\left(\frac{i}{n}\right)^p\cdot \frac{1}{n}\\
    &=\int^1_0 x^p\dd x=\left.\frac{x^{p+1}}{p+1}\right|^1_0=\frac{1}{p+1}
\end{split}\]
\end{solution}

\begin{ex}
\begin{enumerate}
\item 用求原函数的方法, 计算下列定积分:
\begin{multicols}{2}
\begin{enumerate}
 \item $\Int_{2}^{3}\left(x^{3}+6 x^{2}+4 x+1\right) \dd x$
\item $\Int_{1}^{5} \frac{v^{2}+2}{v^{4}} \dd v$
\item $\Int_{-3}^{-2}\left(x+\frac{2}{x}\right)^{2} \dd x$
\item $\Int_{1}^{10}\left(x+\frac{1}{x}\right) \dd x$
\item $\Int_{a}^{b}|x-1| \dd x$
\end{enumerate}
\end{multicols}
\item 求曲线 $y=2 x-x^{2}$ 与 $x$ 轴之间的面积.
\item 设 $y=x(x-1)(x-2)$,
求 $\Int_{0}^{1} y \dd x,\; \Int_{1}^{2} y\dd x,\; \Int_{0}^{2} y \dd x$,
并说明所得结果的几何意义.

\item 求直线$y=2x$和曲线$y=x^2$之间的面积.
\item 求由曲线$y=\frac{1}{x^2}$和直线$x=1$, $x=4$所围成的曲边梯形的
面积.
\item 求抛物线方程使它具有以下的性质:
\begin{enumerate}
\item 抛物线通过$(0, 0)$, $(1, 2)$两点,
\item 它的对称轴平行$y$轴,且向上凸,
\item 它与$x$轴所围的面积最小.
\end{enumerate}

\item 求正数$a$, 使$\Int^1_0 |x^2-a^2|\dd x$最小.
\item 求下列函数的导数:
\begin{multicols}{2}
    \begin{enumerate}
     \item $F(x)=\Int_{a}^{x}\sin^2 t \dd t$
    \item $F(x)=\Int_{a}^{x^2} \sin^2 t \dd t$
    \end{enumerate}
    \end{multicols}
\end{enumerate}

\end{ex}

\section{不定积分}
由上节的微积分基本定理得知,在$[a,b]$上的两个可
微函数$G_1(x)$和$G_2(x)$均以$f(x)$为其导数,则$G_1(x)$与$G_2(x)$在$[a,b]$上只差一个常数$C$. 由此可见,如果$G'(x)=f(x)$, 且$C$为任意常数,那么$[G(x)+C]'=G'(x)=f(x)$. 这就是说:如果 函数$f(x)$有一个原函数$G(x)$, 那么就有无穷多个原函数,并且所有原函数刚好组成函数族
\[\{G(x)+C\;|\; C\text{是任意常数}\}\]

\begin{blk}
   {定义} 函数$f(x)$在区间$[a,b]$上的原函数全体叫做$f(x)$的不定积分,记作
   \[\int f(x)\dd x\]
   设$G(x)$是$f(x)$的一个原函数,那么由不定积分的定义得到
\[\int f (x) \dd x=G (x) +C\]
   其中$C$是任意常数.
\end{blk}   

关于不定积分,易见有如下性质:

\begin{blk}{性质1}
   积分与微分互为逆运算,即:
\[\begin{cases}
    \dd\Int f(x)\dd x=f(x)\dd x\\
    \Int \dd f(x)=f(x)+C
\end{cases}\]
\end{blk}

事实上,因为$\frac{\dd }{\dd x}\Int^x_a f(t)\dd t=f(x)$,所以
\[\int f(x)\dd x=\int^x_a f(t)\dd t+C\qquad \text{(基本定理1和不定积分定义)}\]
于是
\[\left(\int f(x)\dd x\right)'=\left(\int^x_a f(x)\dd x\right)'=f(x)\]
从而
\[\dd\int f(x)\dd x=f(x)\dd x\]
又
\[\begin{split}
    \int \dd f(x)&=\int f'(x)\dd x=\int^x_a f'(t)\dd t+C_1\\
    &=f(x)-f(a)+C_1\quad \text{(基本定理)}\\
    &=f(x)+C
\end{split}\]
这里$C=-f(a)+C_1$. 

从性质1看出,若不要不定积分等式中的任意常数项,则当符号$\Int$与$\dd$紧接着时,无论哪个在前,哪个在后,都可以消掉.

因为微分,积分是两种互逆的运算,所以它们的运算法则是互相对应的,现将它们相对应的法则叙述如下:

\begin{blk}{性质2}
 \[\begin{cases}
     {\rm d} [f(x)+g(x)]=\dd f(x)+\dd g(x)\\
     \Int [f(x)+g(x)] \dd x=\Int f(x)\dd x+\Int g(x)\dd x+C
 \end{cases}\]
 \end{blk}

 上面不定积分公式的正确性不难证明,只要求出两边的微商,由于所得到的微商恒等,就可以知道它们只差一个常数项,因为,由性质1
\[\begin{split}
    \left(\int[f(x)+g(x)]\dd x\right)'&=f(x)+g(x)\\
    \left(\int f(x)\dd x+\int g(x)\dd x\right)'&=\left(\int f(x)\dd x\right)'+\left(\int g(x)\dd x\right)'=f(x)+g(x)
\end{split}\]
所以$f(x)+g(x)$的两个原函数相差一个常数项,即:
\[\int[f(x)+g(x)]\dd x=\int f(x)\dd x+\int g(x)\dd x+C\]
同理可得:

\begin{blk}{性质3}
\[\begin{cases}
    {\rm d} [a f(x)]=a\dd f(x)\\
    \Int af(x)\dd x=a\Int f(x)\dd x
\end{cases}\qquad (a\ne 0)\]
\end{blk}

\subsection{基本积分表}

由基本微商公式表,倒转顺序,就可以得到下面这个表:
\begin{itemize}
    \item $\frac{\dd}{\dd x}C=0,\qquad \Int 0\dd x=C$
    \item $\frac{\dd }{\dd x}x^{\mu}=\mu x^{\mu-1},\qquad \Int x^{\mu}\dd x=\frac{x^{\mu+1}}{\mu+1}+C\quad \text{($\mu$是不等于$-1$的实数)}$
    \item $\frac{\dd }{\dd x}\ln|x|=\frac{1}{x},\qquad \Int\frac{\dd x}{x}=\ln|x|+C\quad (x\ne 0)$
    \item $\frac{\dd }{\dd x}e^x=e^x,\qquad \Int e^x\dd x=e^x+C$
\end{itemize}
还因为$a^x=e^{x\ln a}$,由复合函数求导法则,得:
\[(a^x)'=e^{x\ln a}(x\ln a)'=\ln a\cdot e^{x\ln a}=\ln a\cdot a^x\]
所以:
\begin{itemize}
    \item $\frac{\dd}{\dd x}a^x=\ln a\cdot a^x,\qquad \Int a^x\dd x=\frac{a^x}{\ln a}+C$
    \item $\frac{\dd}{\dd x} \sin x=\cos x, \qquad \Int \cos x \dd x=\sin x+C$.
    \item $\frac{\dd}{\dd x} \cos x=-\sin x, \qquad \Int \sin x \dd x=-\cos x+C$.
    \item  $\frac{\dd}{\dd x}\tan  x=\sec ^{2} x, \qquad \Int \sec^{2} x \dd x=\tan x+C$
    \item $\frac{\dd}{\dd x}\cot x=-\csc ^{2} x, \qquad \Int \csc^{2} x \dd x=-\cot x+C$
    \item $\frac{\dd}{\dd x} \arcsin x=\frac{1}{\sqrt{1-x^{2}}} ,\qquad \Int \frac{\dd x}{\sqrt{1-x^{2}}}=\arcsin x+C$ 
    
    $\frac{\dd}{\dd x} \arccos x=\frac{-1}{\sqrt{1-x^{2}}},\qquad \Int\frac{-1}{\sqrt{1-x^{2}}}\dd x=-\arccos x+C'$
    \item  $\frac{\dd}{\dd x}\arctan x=\frac{1}{x^{2}+1}, \qquad \Int \frac{\dd x}{x^{2}+1}=\arctan x+C$ 
    
    $\frac{\dd}{\dd x} \arccot x=\frac{-1}{x^{2}+1}, \qquad \Int \frac{-1}{x^{2}+1}\dd x=-\arccot  x+C'$
\end{itemize}

现在来看不定积分的几何意义.

由不定积分的定义知道,求函数$f(x)=3x^2$的原函数全体,就是求不定积分$\Int 3x^2\dd x$, 也就是求满足微分方程
\begin{equation}
    \frac{\dd y}{\dd x}=3x^2
\end{equation}
的所有的原函数.

显然,$y=x^3$就是微分方程(4.5)的一个原函数,它的图象上各点切线的斜率由(4.5)确定.我们称曲线$y=x^3$为
微分方程(4.5)的积分曲线,把它沿$y$轴上下平移任意一段距离,便得到曲线族
\begin{equation}
    y=x^3+C
\end{equation}
并且(4.6)中任意一条曲线与曲线$y=x^3$在横坐标相同的点的切线平行,其斜率都等于$3x^2$, 因此,曲线族(4.6)是满足微分方程(4.5)的所有积分曲线,也就是不定积分$\Int 3x^2\dd x=x^3+C$的几何表示(图4.3).

为要确定(4.5)的某条积分曲线的位置,应当给出这条积分曲线必须经过的一点,例如它经过点$P(1, 3)$. 我们说这条件也是微分方程(4.5)的一个相应的原函数要满足的初值条件
$y\big|_{x=1}=3$
把它们代入(4.6), 得到
\[3=1^3+C\quad \Rightarrow\quad C=2\]
因此,满足微分方程$\frac{\dd y}{\dd x}=3x^2$和初值条$y\big|_{x=1}=3$的原函
数或相应的积分曲线是
\[y=x^3+2\]

通常把满足微分方程
\begin{equation}
\frac{\dd y}{\dd x}=f(x)
\end{equation}
的所有解叫做给定函数$f(x)$的不定积分,即
\[y=\int f(x)\dd x\]
把函数$f(x)$的一个原函数$y=G(x)$的图象叫做微分方程(4.7)的一条\textbf{积分曲线},这样,不定积分
\[y=\int f(x)\dd x=G(x)+C \quad \text{($C$为任意常数)}\]
的几何意义就表示与$y=G(x)$在各点有相同斜率$\frac{\dd y}{\dd x}=f(x)$的积分曲线族(图4.3).
\begin{figure}[htp]
    \centering
\begin{tikzpicture}[>=latex, xscale=3, yscale=.8]
\draw[->](-1,0)--(1.25,0)node[right]{$x$};
\draw[->](0,-4)--(0,4)node[right]{$y$};
\node at (0,0)[below left]{$O$};
\foreach \y in {1,2,3}
{
    \draw[domain=-.8:.8, samples=100, very thick]plot(\x, {\x*\x*\x+\y})node[right]{$y=x^3+$\y};
    \draw[domain=-.8:.8, samples=100, very thick]plot(\x, {\x*\x*\x-\y})node[right]{$y=x^3-$\y};
}
\draw[domain=-.8:.8, samples=100, very thick]plot(\x, {\x*\x*\x})node[right]{$y=x^3$};
\end{tikzpicture}
    \caption{}
\end{figure}

由微积分基本定理2知道,如果$y=G(x)$满足下面两个条件:
\[G'(x)=f(x)\quad \text{且}\quad G(a)=0\]
那么,$y=G(x)$就是$f(x)$的求和函数$G(x)=\Int^x_a f(x)\dd x$.

\begin{example}
求$\Int \sin (kx)\dd x,\quad \Int \cos(kx)\dd x,\quad \Int e^{kx}\dd x,\quad \Int \frac{\dd x}{x^2+a^2}$
\end{example}

\begin{solution}
    因为
\[\begin{split}
    (\cos kx)'&=-k\sin kx\\
    (\sin kx)'&=k\cos kx\\
    (e^{kx})'&=ke^{kx}\\
    \left(\arctan\frac{x}{a}\right)'&=\frac{1}{\left(\frac{x}{a}\right)^2+1}\cdot\left(\frac{x}{a}\right)'=\frac{a}{x^2+a^2}
\end{split}\]
所以
\[\begin{split}
    \int \sin (kx)\dd x&=-\frac{1}{k}\cos kx+C\\
    \Int \cos(kx)\dd x&=\frac{1}{k}\sin kx+C\\
    \Int e^{kx}\dd x&= \frac{1}{k}e^{kx}+C\\
    \Int \frac{\dd x}{x^2+a^2}&=\frac{1}{a}\arctan\frac{x}{a}+C \\
\end{split}  \]
\end{solution}

\begin{example}
    证明:若$\Int f(t)\dd t=F(t)+C$,则
\[\int f(kx)\dd x=\frac{1}{k}F(kx)+C\]
\end{example}

\begin{proof}
只须证明$f(kx)$的一个原函数是$\frac{1}{k}F(kx)$即可.
我们求$\frac{1}{k}F(kx)$对$x$的导数,于是
\[\frac{\dd}{\dd x}\left[\frac{1}{k}F(kx)\right]=\frac{1}{k}F'(kx)\cdot (kx)'=F'(kx)\]
由题设知$F' (t) =f (t)$, 所以
\[\frac{\dd}{\dd x}\left[\frac{1}{k}F(kx)\right]=F'(kx)=f(kx)\]
因此
\[\int f (kx) \dd x=\frac{1}{k}F (kx) +C\]
\end{proof}

\begin{example}
求$\Int \sin kx\sin\ell x\dd x$
\end{example}

\begin{solution}
\[\begin{split}
    \Int \sin kx\sin\ell x\dd x&=\int \frac{1}{2}[\cos(k-\ell)x-\cos(k+\ell)x]\dd x\\
    &=\frac{1}{2}\int \cos (k-\ell)x\dd x-\frac{1}{2}\int \cos (k+\ell)x\dd x\\
&=\frac{\sin(k-\ell)x}{2(k-\ell)}-\frac{\sin(k+\ell)x}{2(k+\ell)}+C
\end{split}\]
    这里$k\ne \pm \ell$.(如果$k=\pm\ell$,则另当考虑)
\end{solution}


\begin{example}
求$\Int \left(\frac{2a}{\sqrt{x}}-\frac{b}{x^2}+2xc^{\tfrac{2}{3}}\right)\dd x$
\end{example}


\begin{solution}
    \[\begin{split}
        \Int \left(\frac{2a}{\sqrt{x}}-\frac{b}{x^2}+2xc^{\tfrac{2}{3}}\right)\dd x&=2a\int x^{-\tfrac{1}{2}}\dd x-b\int x^{-2}\dd x+3c\int x^{\tfrac{2}{3}}\dd x\\
        &=4a\sqrt{x}+\frac{b}{x}+\frac{9}{5}cx^{\tfrac{5}{3}}+C
    \end{split}\]
\end{solution}

\begin{example}
    求$\Int \frac{3x^2}{1+x^2}\dd x$
\end{example}

\begin{solution}
\[\begin{split}
    \Int \frac{3x^2}{1+x^2}\dd x&=3\int \left(1-\frac{1}{x^2+1}\right)\dd x\\
    &=3\int \dd x-3\int \frac{\dd x}{x^2+1}\\
    &=3x-3\arctan x+C
\end{split}\]
\end{solution}

以上各例都是利用不定积分的性质2、3, 同时配合使用三角恒等式或设法将被积函数拆成几项,从而把一个比较复杂的积分化成若干个可以查基本积分表的积分.

\begin{ex}
\begin{enumerate}
    \item 计算下列不定积分:
\begin{multicols}{2}
\begin{enumerate}
\item  $\Int\left(x^{3}+3 x+5\right) \dd x$
\item  $\Int(x+2)^{4} \dd x$
\item  $\Int\left(x^{-2}+x^{-1}+x^{\tfrac{1}{3}}+x^{\tfrac{1}{2}}\right) \dd x$
\item  $\Int\left(a^{\tfrac{2}{3}}-x^{\tfrac{2}{3}}\right)^{3} \dd x$
\item  $\Int\left(\frac{1}{x}+e^{x}\right) \dd x$
\item  $\Int \frac{2 \cdot 3^{x}-5 \cdot 2^{x}}{3^{x}} \dd x$
\item  $\Int \left(\frac{3}{1+x^{2}}-\frac{2}{\sqrt{1-x^{2}}}\right) \dd x$
\item  $\Int \frac{2+x^{2}}{1+x^{2}} \dd x$
\item  $\Int \frac{\dd x}{x^{2}\left(1+x^{2}\right)}$
\item  $\Int \frac{\dd x}{x^{2}\left(a^{2}+x^{2}\right)}$
\item  $\Int 3^{x} e^{x} \dd x$
\end{enumerate}
\end{multicols}

\item  利用三角恒等式, 计算下列不定积分:
    \begin{multicols}{2}
        \begin{enumerate}
    \item  $\Int \tan^{2} x \dd x$
    \item  $\Int \frac{\cos 2 x}{\cos x-\sin x} \dd x$
    \item  $\Int \frac{2+\sin ^{2} x}{\cos ^{2} x} \dd x$
    \item  $\Int \sin ^{2} x \dd x$
    \item  $\Int \cos ^{2} x \dd x$
    \item  $\Int \cos k x \cos \ell x \dd x$
    \item  $\Int \sin 2 x \cos 5 x \dd x$
    \item  $\Int \frac{\dd t}{\sin ^{2} t \cos ^{2} t}$
    \item  $\Int \sin ^{3} x \dd x$
    \item  $\Int \cos ^{3} x \dd x$
\end{enumerate}
\end{multicols}    

\item 若$\Int f(t)\dd t=F(t)+C$, 求证
\[\int f (ax+b) \dd x=\frac{1}{a}F (ax+b) +C\]
\item  若$\frac{\dd y}{\dd x}=4x-6$, 且当$x=0$时,$y=9$, 求$y$的极小值.
\item  若$\frac{\dd y}{\dd x}=ax+b$, 求原函数$y=f(x)$使其满足下列条件:
\begin{enumerate}
    \item 当$x=0$时,$y=5$;
    \item 积分曲线向下凸且曲线$y=f(x)$在$x$轴上截得的弦长等于4;
    \item 曲线$y=f(x)$上纵坐标的最小值等于$-4$.
\end{enumerate}
\item 试求一个次数最低的多项式,使之当$x=1$时,有最大值6,而当$x=3$时,有最小值2.
\item  求下列各式的极限:
\begin{enumerate}
    \item $\Lim_{n\to\infty}\frac{\sqrt[n]{e}+\sqrt[n]{e^2}+\cdots+\sqrt[n]{e^n}}{n}$
    \item $\Lim_{n\to\infty}\frac{\sqrt[n]{e}+\sqrt[n]{e^4}+\cdots+\sqrt[n]{e^{2n}}}{n}$
\end{enumerate}
\end{enumerate}

\end{ex}


\section{不定积分的计算方法}
从上一节看到,虽然利用积分运算法则及基本积分表可以求出不少函数的原函数,但是实际遇到的积分仅凭这一些
方法还不能完全解决,本节将介绍几种典型的方法,利用这些方法,我们就可以计算更多的不定积分,此外,我们还制作了较为详细的积分表,可供使用时查阅.

\subsection{第一换元法}

有一些不定积分,将积分变量进行某种变映后就能由基本积分公式求出,例如求$\Int e^{2x}\dd x$, 在基本积分公式中只有$\Int e^x\dd x$, 没有直接求出它的公式,我们看到所求积分的被积函数$e^{2x}$是$y=e^u$和$u=2x$的复合函数,如果在积分表达式$e^{2x}\dd x$中,把被积函数看成中间变量$u$的函数,即设$u=2x$则$e^{2x}=e^u$, 从而$\dd u=2\dd x$, 由此可见只要再给微分$\dd x$凑上一个常数因子2, 就得到中间变量$u$的微分$\dd u$, 于是
\[\int e^{2x}\dd x=\int e^{2x}\cdot \frac{1}{2}\dd (2x) =\frac{1}{2}\int e^u\dd u=\frac{1}{2}e^u+C\]
然后再代回原来的积分变量,得
\[\int e^{2x}\dd x=\frac{1}{2}e^{2x}+C\]

\begin{example}
    求$\Int \frac{\dd x}{\sqrt{a^2-x^2}}\quad (a>0)$
\end{example}

\begin{solution}
    因为被积函数$\frac{1}{\sqrt{a^2-x^2}}=\frac{1}{a\sqrt{1-\left(\frac{x}{a}\right)^2}}$
是$y=\frac{1}{a\sqrt{1-u^2}}$和$u=\frac{x}{a}$的复合函数,且$\dd u=\frac{1}{a}\dd x$,于是
\[\begin{split}
    \int \frac{\dd x}{\sqrt{a^2-x^2}}=\int \frac{\dd x}{a\sqrt{1-\left(\frac{x}{a}\right)^2}}&=\int \frac{\frac{1}{a}\dd x}{\sqrt{1-\left(\frac{x}{a}\right)^2}}\\
    \text{令}u=\frac{x}{a}\to &= \int \frac{\dd u}{\sqrt{1-u^2}}\\
    \text{查表}\to &=\arcsin u+C\\
    u=\frac{x}{a} \to &=\arcsin \frac{x}{a}+C
\end{split}\]
\end{solution}

\begin{example}
    求$\Int \tan x\dd x$
\end{example}


\begin{solution}
\[\begin{split}
    \Int \tan x\dd x=\Int  \frac{\sin x}{\cos x}\dd x&=-\int \frac{\dd \cos x}{\cos x}\\
    u=\cos x\to &=-\int \frac{\dd u}{u}\\
    &=-\ln|u|+C=-\ln |\cos x|+C=\ln |\sec x|+C
\end{split}\]
\end{solution}

从上述两例看到,在求不定积分时,首先要与已知的基本积分公式相对比,并利用简单的变量代换,把要求的积分凑成公式所具有的形式,求出以后,再把原来的变量代回.以上变量代换中关键的一步是把原来的被积函数$f(x)$的某部分$\varphi(x)$换成中间变量$u$, 那么原来的被积表达式$f(x)\dd x$就拆成了$g[\varphi(x)]=g(u)$和$\varphi'(x)\dd x=\dd u$这两部分的乘积,即
\[f (x) \dd x=g [\varphi (x) ] \varphi' (x) \dd x=g (u) \dd u\]
于是:若$\Int g(x)\dd x=F(x)+C$,那么
\[\int g[\varphi(x)]\varphi'(x)\dd x=\int g(u)\dd u=F(u)+C=F[\varphi(x)]+C\]
上述换元法叫做第一换元法.

\begin{example}
求$\Int \sin^{3}x\dd x$
\end{example}

\begin{solution}
设$u=\cos x$,$\dd u=-\sin x\dd x$,于是
\[\begin{split}
    \Int \sin^{3}x\dd x&=\int (1-\cos^2 x)\cdot \sin x\dd x\\
  u=\cos x\to  &=-\int (1-u^2)\dd u\\
  &=-u+\frac{u^3}{3}+C=\frac{\cos^3 x}{3}-\cos x+C
\end{split}\]
\end{solution}

对换元积分法较熟练以后,所设换元变量$u$可以不写出.

\begin{example}
求$\Int \frac{\dd x}{\sqrt{ax+b}}$
\end{example}

\begin{solution}
    \[\Int \frac{\dd x}{\sqrt{ax+b}}=\frac{1}{a}\int \frac{\dd(ax+b)}{\sqrt{ax+b}}=\frac{2}{a}\sqrt{ax+b}+C\]
\end{solution}

\begin{example}
    求$\Int \frac{\dd x}{a^2-x^2}$
\end{example}

\begin{solution}
\[\begin{split}
    \Int \frac{\dd x}{a^2-x^2}&=\frac{1}{2a}\int\left(\frac{1}{a-x}+\frac{1}{a+x}\right)\dd x\\
    &=\frac{1}{2a}\left[-\int\frac{\dd(a-x)}{a-x}+\int\frac{\dd(a+x)}{a+x}\right]\\
    &=\frac{1}{2a}\left[-\ln|a-x|+\ln|a+x|\right]=\frac{1}{2a}\ln\left|\frac{a+x}{a-x}\right|
\end{split}\]
\end{solution}

\begin{example}
    求$\Int \sec x\dd x$
\end{example}

\begin{solution}
    \[\Int \sec x\dd x=\int \frac{\cos x}{\cos^2 x}\dd x=\int \frac{\dd(\sin x)}{1-\sin^2 x}\]
利用上题结果,得:
\[\begin{split}
    \int \sec x\dd x&=\frac{1}{2}\ln\left|\frac{1+\sin x}{1-\sin x}\right|+C\\
    &=\frac{1}{2}\ln\frac{1+\sin x}{1-\sin x}+C\\
    &=\frac{1}{2}\ln\frac{(1+\sin x)^2}{\cos^2 x}+C\\
    &=\ln|\sec x+\tan x|+C
\end{split}\]

\end{solution}

\begin{example}
    求$\Int \frac{1}{\sin^2x +3\cos^2x}\dd x$
\end{example}

\begin{solution}
    \[\begin{split}
        \Int \frac{1}{\sin^2x +3\cos^2x}\dd x&=\Int \frac{1}{(\tan^2x +3)\cos^2x}\dd x\\
        &=\int \frac{\dd (\tan x)}{\tan^2 x+\left(\sqrt{3}\right)^2}=\frac{1}{\sqrt{3}}\int \frac{\dd\left(\frac{\tan x}{\sqrt{3}}\right)}{\left(\frac{\tan x}{\sqrt{3}}\right)^2+1}\\
        &=\frac{1}{\sqrt{3}}\arctan\left(\frac{\tan x}{\sqrt{3}}\right)+C      
    \end{split}\]
\end{solution}

\begin{ex}
求下列不定积分:
\begin{enumerate}
    \begin{multicols}{2}
\item  $\Int(3 x+5)^{100} \dd x$
\item  $\Int\pi e^{-\tfrac{\pi}{4} x} \dd x$
\item  $\Int \frac{\ln x}{x} \dd x$
\item  $\Int\frac{1}{x^{2}} \cdot e^{\tfrac{1}{x}} \dd x$
\item  $\Int\frac{\cos \sqrt{x}}{\sqrt{x}} \dd x$
\item  $\Int \frac{1}{(\arcsin x)^{2} \sqrt{1-x^{2}}} \dd x$
\item  $\Int\frac{\dd x}{x^{2}+a^{2}}$
\item  $\Int \frac{\dd x}{x^{2}-a^{2}}$
\item  $\Int \frac{1}{\sqrt{x}} e^{\sqrt{x}} \dd x$
\item  $\Int \frac{2 a x+b}{2 \sqrt{a x^{2}+b x+c}} \dd x$
\item  $\Int\frac{1}{1+e^{x}} \dd x$        
    \end{multicols}
    \begin{multicols}{2}
\item  $\Int \frac{e^{x}}{e^{2 x}+1} \dd x$    
\item $\Int \frac{\cos x}{e^{\sin x}} \dd x$
\item $\Int \frac{\dd x}{\cos ^{2} x \sqrt{\tan x}}$
\item $\Int  \frac{\cos a x}{b-\sin a x} \dd x\quad (a \neq 0) $
\item $\Int x \sqrt{a x^{2}+c}\; \dd x$
\item $\Int \frac{\sin ^{2} x \cos x \dd x}{1+\sin x}$
\item $\Int \frac{\sqrt{\arctan x}}{1+x^{2}} \dd x$
\item $\Int \frac{1}{1+\sin x} \dd x$        
    \end{multicols}
    \begin{multicols}{2}
\item $\Int \frac{\sqrt{1+\cos x}}{\sin x} \dd x$
\item $\Int \frac{\dd x}{x^{2}+x-6}$
\item $\Int \frac{x \dd x}{3 x^{2}+x+2}$
\item $\Int  \frac{\dd x}{3 x^{2}+x+2}$
\item $\Int  \cot x \dd x$
\item $\Int  \frac{\dd x}{\sin x}$
\item $\Int \frac{\arcsin x}{\sqrt{1-x^{2}}} \dd x$        
    \end{multicols}

\end{enumerate}
\end{ex}

\subsection{第二换元积分法}

有些积分不能很容易地凑出微分,而是一开始就要作代换,把要求的积分化简,然后再求积分.

\begin{example}
    求$\Int \frac{1}{1+\sqrt{x}}\dd x$.
\end{example}

\begin{solution}
    这个积分的表达式不容易拆成中间变量$u=\varphi(x)$
的函数$g(\varphi(x))=g(u)$和中间变量的微分$\varphi'(x)\dd x=\dd u$的乘积,因此前面的第一换元法对此题用起来不便.这时我们设法作一个代换,以便把被积函数中的根号去掉,化成新变量的有理式.

令$x=u^2$, 即作代换$\sqrt{x}=u$, 于是
\[\begin{split}
    \frac{1}{1+\sqrt{x}}&=\frac{1}{1+u}\\
    \dd x&=2u\dd u
\end{split}\]
从而
\[\begin{split}
    \Int \frac{1}{1+\sqrt{x}}\dd x&=\int \frac{1}{1+u}2u\dd u\\
    &=2\int \frac{(1+u)-1}{1+u}\dd u\\
    &=2\left[\int \dd u-\int \frac{1}{1+u}\dd u\right]\\
    &=2\left[u-\ln |1+u|\right]+C\\
    &=2\left[\sqrt{x}-\ln\left(1+\sqrt{x}\right)\right]+C
\end{split}\]
\end{solution}

这里由于把积分变量$x$换成了$u$的函数$x=u^2$, 因此在最后结果中,用变量$u$表出的函数,还要还原成$x$的函数,这就要求从所作的代换$x=u^2$中解出反函数$u=\sqrt{x}$来.

我们把第二换元积分法用定理形式叙述如下:

\begin{blk}{定理}
设$x=\varphi(u)$是可微函数,且$\varphi'(u)\ne 0$,若
\begin{equation}
    \int f[\varphi(u)]\varphi'(u)\dd u=F(u)+C
\end{equation}
那么
\begin{equation}
    \int f(x)\dd x=F\left[\varphi^{-1}(x)\right]+C
\end{equation}
\end{blk}

\begin{proof}
只须证明$\frac{\dd}{\dd x}F[\varphi^{-1}(x)]=f(x)$即可.

因为$\varphi'(u)\ne 0$,故$x=\varphi(u)$的反函数$u=\varphi^{-1}(x)$存在,于是
\[\begin{split}
    \frac{\dd}{\dd x}F[\varphi^{-1}(x)]&=\frac{\dd}{\dd u}F[\varphi^{-1}(x)]\cdot \frac{\dd }{\dd x}\varphi^{-1}(x)\\
    &=\frac{\dd}{\dd u}F(u)\cdot \frac{1}{\frac{\dd}{\dd u}\varphi(u)}\\
    &=F'(u)\cdot \frac{1}{\varphi'(u)}\\
    &=\left\{f[\varphi(u)]\varphi'(u)\right\}\cdot \frac{1}{\varphi'(u)}\\
    &=f[\varphi(u)]\\
    x=\varphi(u)\to &=f(x)
\end{split}\]
\end{proof}


\begin{example}
    求$\Int\frac{\dd x}{\sqrt{x}+\sqrt[3]{x}}$.
\end{example}

\begin{solution}
    \[\Int\frac{\dd x}{\sqrt{x}+\sqrt[3]{x}}=\int \frac{\dd x}{\left(\sqrt[6]{x}\right)^3+\left(\sqrt[6]{x}\right)^2}\]
设$x=u^6$,$u>0$,则可化去被积函数中的根号,于是
\[\frac{1}{\sqrt{x}+\sqrt[3]{x}}=\frac{1}{u^3+u^2}\quad \Rightarrow\quad \dd x=6u^5\dd u\]
从而
\[\begin{split}
    \Int\frac{\dd x}{\sqrt{x}+\sqrt[3]{x}}&=\int \frac{6u^5}{u^3+u^2}\dd u=\int\frac{6u^3}{u+1}\dd u\\
    &=6\int \left(u^2-u+1-\frac{1}{u+1}\right)\dd u\\
    &=6\left\{\frac{u^3}{3}-\frac{u^2}{2}+u-\ln|u+1|\right\}+C\\
    &=2\sqrt{x}-3\sqrt[3]{x}+6\sqrt[6]{x}-\ln\left(\sqrt[6]{x}+1\right)+C
\end{split}
    \]
\end{solution}

\begin{example}
    求$\Int \frac{x+1}{\sqrt[3]{3x+1}}\dd x$.
\end{example}

\begin{solution}
    设$3x+1=t^3$,即$x=\frac{1}{2}(t^3-1)$,于是:$\dd x=t^2\dd t$
\[\begin{split}
    \Int \frac{x+1}{\sqrt[3]{3x+1}}\dd x&=\int \frac{\frac{1}{3}(t^3-1)+1}{t}t^2\dd t\\
&=\frac{1}{3}\int (t^4+2t)\dd t\\
&=\frac{1}{3}\left(\frac{1}{5}t^5+t^2\right)+C=\frac{1}{15}(3x+1)^{\tfrac{5}{3}}+\frac{1}{3}(3x+1)^{\tfrac{2}{3}}+C\\
&=\frac{1}{5}(x+2)(3x+1)^{\tfrac{2}{3}}+C
\end{split}\]
\end{solution}

上面所介绍的两种变量代换法的基本思想是一致的.第一种是把被积函数中的一个小部分看作一个变量,即“化繁为简”;第二种则把积分变量代以一个新变量的函数,表面上看来是“化简为繁”,但实际上克服了求积分的困难.应当注意,在使用第二种变量代换法时,要保证变换是可微的和一对一的,否则会出现错误.

下面介绍当被积函数含有二次根式$\sqrt{a^2-x^2}$, $\sqrt{a^2+x^2}$, $\sqrt{x^2-a^2}$时,怎样作代换.

\begin{example}
    求$\Int \sqrt{a^2-x^2}\dd x\quad (a>0)$.
\end{example}

\begin{solution}
因为\[\sqrt{a^2-x^2}=\sqrt{a^2\left(1-\frac{x^2}{a^2}\right)}=a\sqrt{1-\left(\frac{x}{a}\right)^2}\]
所以可作正弦代换
\[\frac{x}{a}=\sin t\quad \left(-\frac{\pi}{2}\le t\le \frac{\pi}{2}\right)\]
$\sin t$在$\left[-\frac{\pi}{2},\frac{\pi}{2}\right]$上连续且递增,有反函数$t=\arcsin\frac{x}{a},\; -1\le x\le 1$. 于是
\[\begin{split}
    \dd x&=\dd (a\sin t)=a\cos t\dd t\\
    \sqrt{a^2-x^2}&=a\sqrt{1-\left(\frac{x}{a}\right)^2}=a\sqrt{1-\sin^2 t}=a\cos t
\end{split}\]
从而:
\[\begin{split}
    \Int \sqrt{a^2-x^2}\dd x&=\int a\cos t\cdot a\cos t\dd t=a^2\int \cos^2 t\dd t\\ 
&=a^2\int \frac{1+\cos 2t}{2}\dd t=\frac{a^2}{2}\left[\int \dd t+\frac{1}{2}\int \cos 2t\dd (2t)\right]\\
&=\frac{a^2}{2}\left(t+\frac{1}{2}\sin 2t\right)+C=\frac{a^2}{2}\left(t+\sin t\cos t\right)+C\\
&=\frac{a^2}{2}\left[\arcsin\frac{x}{a}+\frac{x}{a}\sqrt{1-\left(\frac{x}{a}\right)^2}+C\right]\\
&=\frac{a^2}{2}\arcsin\frac{x}{a}+\frac{x}{2}\sqrt{a^2-x^2}+C
\end{split}\]
\end{solution}

\begin{example}
    求$\Int \frac{\dd x}{\sqrt{a^2+x^2}}\quad (a>0)$.
\end{example}

\begin{solution}
因为$\sqrt{a^2+x^2}=a\sqrt{1+\left(\frac{x}{a}\right)^2}$,设$\frac{x}{a}=\tan t\; \left(-\frac{\pi}{2}<t<\frac{\pi}{2}\right)$,则存在反函数
\[t=\arctan\frac{x}{a}\quad \Rightarrow\quad \dd x=a\sec^2 t\dd t\]
又$\sqrt{a^2+x^2}=a\sqrt{1+\tan^2 t}=a\sec t$
,从而:
\[\begin{split}
    \Int \frac{\dd x}{\sqrt{a^2+x^2}}&=\int\frac{a\sec^2 t\dd t}{a\sec t}=\int\sec t\dd t\\
    &=\ln|\sec t+\tan t|+C_1\\
    &=\ln\left|\frac{1}{a}\sqrt{a^2+x^2}+\frac{x}{a}\right|+C_1\\
    &=\ln\left|x+\sqrt{a^2+x^2}\right|-\ln a+C_1\\
    &=\ln\left|x+\sqrt{a^2+x^2}\right|+C\qquad (C=C_1-\ln a)
\end{split}\]
\end{solution}

\begin{example}
    求$\Int \frac{\dd x}{\sqrt{x^2-a^2}}\quad (a>0)$
\end{example}

\begin{solution}

    \begin{center}
\begin{tikzpicture}[scale=.7]
\draw[very thick](0,0)--node[below]{$a$}(3,0)--node[right]{$\sqrt{x^2-a^2}$}(3,4)--node[left]{$x$}(0,0);
\draw(.75,0) arc (0:53.13:.75)node[right]{$t$};
\end{tikzpicture}
    \end{center}

    作正割代换.设$x=a\sec t\; \left(0<t<\frac{\pi}{2}\right)$,此函数在$t=\frac{\pi}{2}$点间断,在区间$\left[0,\frac{\pi}{2}\right)$上由$a$递增到$+\infty$,而在区间$\left(\frac{\pi}{2},\pi\right]$上由$-\infty$递增到$-a$.因此,$x=a\sec t$在$\left(0,\frac{\pi}{2}\right]$上的反函数是 
\[t=\arcsec\frac{x}{a}\quad \left(1<\frac{x}{a}<\infty,\; \text{即}x>a\right)\]
于是,当$x>a$时,有
\[\begin{split}
    \sqrt{x^2-a^2}&=\sqrt{a^2\sec^2 t-a^2}=\sqrt{a^2(\sec^2 t-1)}\\
    &=a\sqrt{\tan^2 t}=a|\tan t|\\
    &=a\tan t \quad \left(0\le t<\frac{\pi}{2}\right)
\end{split}\]
\[\dd x=\dd(a\sec t)=a\frac{\sin t}{\cos^2 t}\dd t\]
从而:
\[\begin{split}
    \Int \frac{1}{\sqrt{x^2-a^2}}\dd x&=\int\frac{1}{a\tan t}\cdot \frac{a\sin t}{\cos^2 t}\dd t\\
    &=\int \frac{1}{\cos t}\dd t=\ln|\sec t+\tan t|+C_1\\
    &=\ln\left|\frac{x}{a}+\frac{\sqrt{x^2-a^2}}{a}\right|+C_1\\
    &=\ln\left|x+\sqrt{x^2-a^2}\right|+C_1-\ln a=\ln\left|x+\sqrt{x^2-a^2}\right|+C
\end{split}\]
其中$C=C_1-\ln a$.

又函数$x=a\sec t$在$\left(\frac{\pi}{2},\pi\right]$上的反函数是
\[t=\arcsec\frac{x}{a}\quad \left(-\infty<\frac{x}{a}<-1,\; \text{即}x<-a\right)\]
于是,当$x<-a$时,有
\[\sqrt{x^2-a^2}=a\sqrt{\tan^2 t}=a|\tan t|=-a\tan t\quad \left(\frac{\pi}{2}<t\le \pi\right)\]
\[\dd x=\dd(a\sec t)=a\frac{\sin t}{\cos^2 t}\cdot \dd t\]
\[\begin{split}
    \Int \frac{1}{\sqrt{x^2-a^2}}\dd x&=\int-\frac{1}{a\tan t}\cdot \frac{a\sin t}{\cos^2 t}\dd t\\
    &=-\int \frac{1}{\cos t}\dd t=-\ln|\sec t+\tan t|+C_2\\
    &=-\ln\left|\frac{x}{a}-\frac{\sqrt{x^2-a^2}}{a}\right|+C_2\\
    &=-\ln\left|x-\sqrt{x^2-a^2}\right|+C_2+\ln a\\
    &=\ln\left|\frac{1}{x-\sqrt{x^2-a^2}}\right|+C'\qquad (C'=C+\ln a)\\
    &=\ln\left|\frac{x+\sqrt{x^2-a^2}}{a^2}\right|+C'=\ln\left|x+\sqrt{x^2-a^2}\right|+C'-\ln a^2\\
    &=\ln\left|x+\sqrt{x^2-a^2}\right|+C''\qquad (C''=C'-\ln a^2)
\end{split}\]
因此,不论$x>a$或$x<-a$,都有
\[\int\frac{\dd x}{\sqrt{x^2-a^2}}=\ln\left|x+\sqrt{x^2-a^2}\right|+C\]
\end{solution}

到现在为止,我们又得到四个重要的不定积分公式:

\begin{blk}{}
\begin{align}
\Int \frac{\dd x}{a^2+x^2}&=\frac{1}{a}\arctan\frac{x}{a}+C\\
\Int\frac{\dd x}{a^2-x^2}&=\frac{1}{2a}\ln\left|\frac{a+x}{a-x}\right|+C\\
 \Int\frac{\dd x}{\sqrt{a^2-x^2}}&=\arcsin\frac{x}{a}+C\\
\Int \frac{\dd x}{\sqrt{x^2\pm a^2}}&=\ln\left|x+\sqrt{x^2\pm a^2}\right|+C
\end{align}
\end{blk}

利用这些公式可以进一步求出许多不定积分.

\begin{blk}
 {类型A}
\[\int \frac{\dd x}{ax^2+bx+c}\]
\end{blk}

如果被积函数的分母可以分解因式,则把被积函数分解成部分分式,就很容易分项求出不定积分.

如分母不能分解因式,那么将它配方得
\[ax^2+bx+c=a\left(x+\frac{b}{2a}\right)^2+\frac{4ac-b^2}{4a}\]
于是
\begin{itemize}
    \item 当$4ac>b^2$时,上面不定积分归结到(4.10);
    \item 当$4ac<b^2$时,上面不定积分归结到(4.11).
\end{itemize}

\begin{example}
    求$\Int \frac{\dd x}{3x^2+x-1}$.
\end{example}

\begin{solution}
\[\begin{split}
    \Int \frac{\dd x}{3x^2+x-1}&=\frac{1}{3}\int \frac{\dd x}{\left(x+\frac{1}{6}\right)^2-\frac{13}{36}}\\
&=\frac{1}{3}\frac{1}{2\sqrt{\frac{13}{36}}}\ln\left|\frac{x+\frac{1}{6}-\frac{\sqrt{13}}{6}}{x+\frac{1}{6}+\frac{\sqrt{13}}{6}}\right|+C\\
&=\frac{1}{\sqrt{13}}\ln\left|\frac{6x+1-\sqrt{13}}{6x+1+\sqrt{13}}\right|+C
\end{split}\]    
\end{solution}

\begin{blk}
    {类型B}
   \[\int \frac{px+q}{ax^2+bx+c}\dd x\]
   \end{blk}

注意到$\frac{\dd}{\dd x}(ax^2+bx+c)=2ax+b$,被积函数的分子可以写成
\[px+q=\frac{p}{2a}(2ax+b)+q-\frac{pb}{2a}\]
于是
\[\int\frac{px+q}{ax^2+bx+c}\dd x=\frac{p}{2a}\int\frac{2ax+b}{ax^2+bx+c}\dd x+\left(q-\frac{bp}{2a}\right)\int\frac{\dd x}{ax^2+bx+c}\]
等式右端第一项是$\frac{p}{2a}\ln(ax^2+bx+c)$,第二项可归结到类型A.

\begin{example}
    求$\Int \frac{5x-1}{3x^2+x+2}\dd x$.
\end{example}

\begin{solution}
\[\begin{split}
    \Int \frac{5x-1}{3x^2+x+2}\dd x&=\frac{5}{6}\int \frac{6x+1}{3x^2+x+2}\dd x-\frac{11}{6}\int\frac{\dd x}{3x^2+x+2}\\
&=\frac{5}{6}\ln(3x^2+x+2)-\frac{11}{6}\cdot \frac{1}{3}\cdot \int \frac{\dd x}{\left(x+\frac{1}{6}\right)^2+\frac{23}{36}}\\
&=\frac{5}{6}\ln(3x^2+x+2)-\frac{11}{6}\cdot\frac{1}{3}\cdot\frac{6}{\sqrt{23}}\arctan\frac{6x+1}{\sqrt{23}}+C\\
&=\frac{5}{6}\ln(3x^2+x+2)-\frac{11}{3\sqrt{23}}\arctan\frac{6x+1}{\sqrt{23}}+C
\end{split}\]    
\end{solution}

\begin{blk}{类型C}
\[\int \frac{\dd x}{\sqrt{ax^2+bx+c}}\]
\end{blk}

\begin{itemize}
    \item 若$a>0$, 则被积函数的根底式经配方后,所求积分可归结到积分公式(4.13);
    \item  若$a<0$, 则此积分可归结到积分公式(4.12).
\end{itemize}

\begin{example}
    求$\Int \frac{\dd x}{\sqrt{2+x-3x^2}}$
\end{example}

\begin{solution}
    \[\begin{split}
        \Int \frac{\dd x}{\sqrt{2+x-3x^2}}&=\frac{1}{\sqrt{3}}\int \frac{\dd x}{\sqrt{\frac{2}{3}+\frac{1}{3}x-x^2}}\\
        &=\frac{1}{\sqrt{3}}\int \frac{\dd x}{\sqrt{\frac{25}{36}-\left(x-\frac{1}{6}\right)^2}}\\
        &=\frac{1}{\sqrt{3}}\arcsin\frac{x-\frac{1}{6}}{\frac{5}{6}}+C\\
        &=\frac{1}{\sqrt{3}}\arcsin\frac{6x-1}{5}+C       
    \end{split}\]
\end{solution}

\begin{blk}{类型D}
\[\int \frac{px+q}{\sqrt{ax^2+bx+c}}\dd x\]
\end{blk}

被积函数的分子可写成
\[px+q=\frac{p}{2a}(2ax+b)+q-\frac{pb}{2a}\]
所以
\[
    \int \frac{px+q}{\sqrt{ax^2+bx+c}}\dd x=\frac{p}{2a}\int \frac{2ax+b}{\sqrt{ax^2+bx+c}}\dd x+\left(q-\frac{pb}{2a}\right)\int\frac{\dd x}{\sqrt{ax^2+bx+c}}
\]
等式右端第一个积分等于被积函数的分母的2倍,即$2\sqrt{ax^2+bx+c}$, 第二个是类型C的积分.

\begin{example}
    求$\Int \frac{5x-1}{\sqrt{3x^2+x+2}}\dd x$.
\end{example}

\begin{solution}
    \[\begin{split}
        \Int \frac{5x-1}{\sqrt{3x^2+x+2}}\dd x&=\frac{5}{6}\int \frac{6x+1}{\sqrt{3x^2+x+2}}\dd x-\frac{11}{6}\int \frac{\dd x}{\sqrt{3x^2+x+2}}\\
&=\frac{5}{6}\cdot 2\sqrt{3x^2+x+2}\\
&\qquad -\frac{11}{6}\cdot\frac{1}{\sqrt{3}}\ln\left[x+\frac{1}{6}+\sqrt{\left(x+\frac{1}{6}\right)^2+\frac{23}{36}}\right]+C\\
&=\frac{5}{3}\sqrt{3x^2+x+2}-\frac{11}{6\sqrt{3}}\ln\left[x+\frac{1}{6}+\sqrt{\left(x+\frac{1}{6}\right)^2+\frac{23}{36}}\right]+C        
    \end{split}\]
\end{solution}

\begin{blk}{类型E}
    \[\int \frac{\dd x}{x\sqrt{ax^2+bx+c}}\]
\end{blk}

作变量代换$x=\frac{1}{y}$,所求积分可化为类型C.因为$\dd x=-\frac{1}{y^2}\dd y$,并且
\[\int \frac{\dd x}{x\sqrt{ax^2+bx+c}}=\int\frac{-\frac{1}{y^2}\dd y}{\frac{1}{y}\sqrt{\frac{a}{y^2}+\frac{b}{y}+c}}=-\int\frac{\dd y}{\sqrt{a+by+cy^2}}\]

\begin{example}
    求$\Int\frac{\dd x}{x\sqrt{x^2+x+1}}$
\end{example}

\begin{solution}
\[\begin{split}
    \Int\frac{\dd x}{x\sqrt{x^2+x+1}}&=-\int \frac{-\frac{1}{y^2}\dd y}{\frac{1}{y}\sqrt{\frac{1}{y^2}+\frac{1}{y}+1}}=-\int \frac{\dd y}{\sqrt{y^2+y+1}}\\
    &=-\int \frac{\dd y }{\sqrt{\left(y+\frac{1}{2}\right)^2+\frac{3}{4}}}  \\
    &=-\ln \left[y+\frac{1}{2}+\sqrt{\left(y+\frac{1}{2}\right)^2+\frac{3}{4}}\right]+C \\
    &=-\ln \left[y+\frac{1}{2}+\sqrt{y^2+y+1}\right]+C\\
    &=-\ln\left[\frac{1}{x}+\frac{1}{2}+\sqrt{\frac{1}{x^2}+\frac{1}{x}+1}\right]+C\\
    &=-\ln\frac{2+x+2\sqrt{x^2+x+1}}{2x}+C
\end{split}\]
\end{solution}
    
\begin{ex}
\begin{enumerate}
    \item 求下列不定积分:
\begin{multicols}{2}
\begin{enumerate}
    \item $\Int \frac{1}{1+\sqrt[3]{x}}\dd x$
    \item $\Int \frac{1}{(x+2) \sqrt{x+1}}\dd x$
    \item $\Int  \frac{x}{\sqrt[3]{1-x}}\dd x $
    \item $\Int \frac{\sqrt{x}}{\sqrt[3]{x^{2}}-\sqrt[4] {x}}\dd x$
    \item $\Int  \frac{\sqrt{x^{2}-a^{2}}}{x}\dd x $
    \item $\Int  \frac{\sqrt{x^{2}-a^{2}}}{x^{2}}\dd x$
    \item $\Int  \frac{\sqrt{a^{2}-x^{2}}}{x}\dd x$
    \item $\Int \frac{\sqrt{a^{2}-x^{2}}}{x^{2}}\dd x$
    \item $\Int \frac{x^{2}}{\sqrt{1-x^{2}}}\dd x$
    \item $\Int \frac{1}{x^{4} \sqrt{1+x^{2}}}\dd x$
\end{enumerate}
\end{multicols}

    \item 求下列不定积分:
\begin{multicols}{2}
\begin{enumerate}
\item $\Int \frac{1}{\sqrt{4 x^{2}+6 x}}\dd x $
\item $\Int \frac{x}{\sqrt{4 x^{2}+6 x}}\dd x$
\item $\Int  \frac{1}{\sqrt{1-x-4 x^{2}}}\dd x$
\item $\Int \frac{x+1}{\sqrt{1-x-4 x^{2}}}\dd x$
\item $\Int \frac{1}{x \sqrt{x^{2}+1}}\dd x$
\item $\Int \frac{1}{x \sqrt{1-x^{2}}}\dd x$
\item $\Int \frac{1}{x \sqrt{x^{2}-4 x}}\dd x$
\item $\Int \sqrt{\frac{x-1}{x+1}}\dd x$  
\end{enumerate}
\end{multicols}

\end{enumerate}    
\end{ex}

\subsection{有理函数的积分}
如果$P(x)$, $Q(x)$是两个多项式,则$\Int \frac{P(x)}{Q(x)}\dd x$可以化
归为下列类型的积分:
\begin{center}
\begin{tabular}{p{.3\textwidth}p{.3\textwidth}}
        $\Int \frac{A \dd x}{x-a}$ & $\Int \frac{A \dd x}{(x-a)^{n}}$ \\
        $\Int \frac{p x+q}{x^{2}+b x+c} \dd x$ & $\Int \frac{p x+q}{\left(x^{2}+b x+c\right)^{n}} \dd x$
\end{tabular}
\end{center}

对这些积分,我们是有办法的.下面我们通过举例来说明求这些积分的大概的方法.

如果有理分式的分子次数高于分母的次数时,则只要作一次除法就可以把它写成一个多项式与一个真分式的和,而前者是容易求积的.所以我们只须注意真分式的积分如何求.

\begin{example}
    求$\Int \frac{x^3+x+1}{x^2+1}\dd x$
\end{example}

\begin{solution}
    \[\begin{split}
        \Int \frac{x^3+x+1}{x^2+1}\dd x&=\int\left(x+\frac{1}{x^2+1}\right)\dd x\\
&=\int x\dd x+\int\frac{\dd x}{1+x^2}\\
&=\frac{x^2}{2}+\arctan x+C        
    \end{split}\]
\end{solution}

\begin{example}
    求$\Int \frac{\dd x}{x^4-1}$
\end{example}

\begin{solution}
$\because\quad x^4-1=(x-1)(x+1)(x^2+1)$

$\therefore\quad \frac{1}{x^4-1}=\frac{A}{x-1}+\frac{B}{x+1}+\frac{Cx+D}{x^2+1}$

把等式右端通分,再比较两端系数,于是等式
\[A (x^2+1) (x+1)+B(x^2+1)(x-1)+(Cx+D) (x^2-1)=1\]
的左边多项式的$x^3,x^2,x$的系数应为零,常数项应为1, 从而得到
\begin{numcases}{}
    A+B+C=0\\
    A-B+D=0\\
    A+B-C=0\\
    A-B-D=1
\end{numcases}
由$(4.14)+(4.16)$和$(4.15)+(4.17)$得到:
\[A+B=0,\qquad A-B=\frac{1}{2}\]
解得:
\[A=\frac{1}{4},\quad B=-\frac{1}{4},\quad C=0,\quad D=-\frac{1}{2}\]
即:
\[\frac{1}{x^4-1}=\frac{1}{4(x-1)}-\frac{1}{4(x+1)}-\frac{1}{2(x^2+1)}\]
从而
\[\begin{split}
    \int\frac{\dd x}{x^4-1}&=\frac{1}{4}\int \frac{\dd x}{x-1}-\frac{1}{4}\int\frac{\dd x}{x+1}-\frac{1}{2}\int\frac{\dd x}{1+x^2}\\
    &=\frac{1}{4}\ln\left|\frac{x-1}{x+1}\right|-\frac{1}{2}\arctan x+C
\end{split}\]
\end{solution}

\begin{example}
求$\Int \frac{3x^2+x-2}{(x-2)^2(1-2x)}\dd x$
\end{example}

\begin{solution}
设$\frac{3x^2+x-2}{(x-2)^2(1-2x)}=\frac{A}{1-2x}+\frac{B}{x-2}+\frac{C}{(x-2)^2}$,则
\[3x^2+x-2=A(x-2)^2+B(1-2x)(x-2)+C(1-2x)\]

令$1-2x=0$,则$A=-\frac{1}{3}$;令$x-2=0$,则$C=-4$. 比较等式两端$x^2$的系数,得
\[3=A-2B\quad\Rightarrow\quad B=-\frac{5}{3}\]
所以
\[\frac{3x^2+x-2}{(x-2)^2(1-2x)}=-\frac{1}{3(1-2x)}-\frac{5}{3(x-2)}-\frac{4}{(x-2)^2}\]
从而
\[\begin{split}
\int \frac{3x^2+x-2}{(x-2)^2(1-2x)}\dd x &=\frac{1}{6}\int \frac{\dd x}{x-\frac{1}{2}}-\frac{5}{3}\int \frac{\dd x}{x-2}-4\int \frac{\dd x}{(x-2)^2}\\
&=\frac{1}{6}\ln\left|x-\frac{1}{2}\right|-\frac{5}{3}\ln|x-2|+\frac{4}{x-2}+C
\end{split}\]
\end{solution}

三角函数有理式的积分
\[\int R (\sin x,\cos x) \dd x\]
恒可用“半角变换”法,即设$t=\tan\frac{x}{2}$或$x=2\arctan t$化为$t$的有理函数的积分.

事实上,由$t=\tan\frac{x}{2}$, 知
\[\begin{split}
\sin x&=\frac{2 \sin \frac{x}{2} \cos \frac{x}{2}}{\sin ^{2} \frac{x}{2}+\cos ^{2} \frac{x}{2}}=\frac{2 \tan \frac{x}{2}}{1+\tan^{2} \frac{x}{2}}=\frac{2 t}{1+t^{2}}\\
\cos x &=\frac{\cos ^{2}-\frac{x}{2}-\sin ^{2} \frac{x}{2}}{\sin ^{2} \frac{x}{2}+\cos ^{2} \frac{x}{2}}=\frac{1-\tan^{2} \frac{x}{2}}{1+\tan^{2} \frac{x}{2}} 
=\frac{1-t^{2}}{1+t^{2}}
\end{split}\]
又由 $x=2 \arctan t$, 知
\[\dd x=\frac{2 \dd t}{1+t^{2}}\]
于是
\[
\int R(\sin x,  \cos x) \dd x=\int R\left(\frac{2 t}{1+t^{2}}, \frac{1-t^{2}}{1+t^{2}}\right) \cdot \frac{2}{1+t^{2}} \dd x 
\]
上式右端是 $t$ 的有理函数.

\begin{example}
    求$\Int \frac{\dd\theta}{\sin\theta}$.
\end{example}

\begin{solution}
令$t=\tan\frac{\theta}{2}$,则:$\theta=2\arctan t,\quad \dd\theta=\frac{2\dd t}{1+t^2}$,从而:
\[\begin{split}
    \int\frac{\dd\theta}{\sin\theta}=\int\frac{1+t^2}{2t}\cdot \frac{2\dd t}{1+t^2}&=\int\frac{\dd t}{t}\\
    &=\ln|t|+C=\ln\left|\tan\frac{\theta}{2}\right|+C
\end{split}\]
又$\tan\frac{\theta}{2}=\frac{\sin\frac{\theta}{2}}{\cos\frac{\theta}{2}}=\frac{2\sin^2 \frac{\theta}{2}}{2\sin\frac{\theta}{2}\cos \frac{\theta}{2}}=\frac{1-\cos\theta}{\sin\theta}=\csc\theta-\cot\theta$,
所以
\[\int\frac{\dd\theta}{\sin\theta}=\ln\left|\tan\frac{\theta}{2}\right|+C=\ln|\csc\theta-\cot\theta|+C\]    
\end{solution}

\begin{example}
    求$\Int \frac{\dd\theta}{3+2\cos\theta}$
\end{example}

\begin{solution}
    令$t=\tan\frac{\theta}{2}$,则:$\dd\theta=\frac{2\dd t}{1+t^2}$,于是
\[\begin{split}
    \Int \frac{\dd\theta}{3+2\cos\theta}=\int\frac{\frac{2\dd t}{1+t^2}}{3+2\frac{1-t^2}{1+t^2}}&=\int \frac{2\dd t}{5+t^2}\\
    &=\frac{2}{\sqrt{5}} \arctan\frac{t}{\sqrt{5}}+C\\
    &=\frac{2}{\sqrt{5}}\arctan\left(\frac{1}{\sqrt{5}}\tan\frac{\theta}{2}\right)+C
\end{split}\]
\end{solution}

\begin{ex}
\begin{enumerate}
\item 求下列不定积分:
\begin{multicols}{2}
\begin{enumerate}
\item  $\Int \frac{1}{x^{2}-3 x+2}\dd x$
\item  $\Int  \frac{x}{x^{2}-3 x+2}\dd x$
\item  $\Int  \frac{x^{2}}{x^{2}-3 x+2}\dd x$
\item  $\Int \frac{x^{2}+1}{x+3}\dd x$
\item  $\Int \frac{1}{x^{2}-3 x+1}\dd x$
\item  $\Int \frac{x}{x^{2}-3 x+1}\dd x$
\item  $\Int \frac{x^{2}}{x^{2}-3 x+1}\dd x$
\end{enumerate}
\end{multicols}

\item 求下列不定积分:
\begin{multicols}{2}
    \begin{enumerate}
        \item  $\Int  \frac{\cos \theta \dd \theta}{1+\sin ^{2} \theta}$
        \item  $\Int \frac{\sin x+\cos x}{\cos ^{2} x}\dd x$
        \item  $\Int  \frac{\dd x}{4 \cos x+3 \sin x+5}$
        \item  $\Int \frac{\sin x}{\cos x+\sin x}\dd x$
    \end{enumerate}
\end{multicols}
\end{enumerate}
\end{ex}

\subsection{分部积分法}

分部积分法是从两个函数乘积的微分公式得到的求积分方法.

设$u=u(x)$, $v=v(x)$是两个可微函数,乘积$u\cdot v$的微分法则是$$\dd (u\cdot v) =u\dd v+v\dd u$$
移项,得$u\dd v=\dd (w\cdot v) -v\dd u$,
两边积分得
\begin{equation}
    \int u\dd v=u\cdot v-\int v\dd u
\end{equation}
公式(4.18)表示如果积分$\Int u\dd v=\Int u(x)v'(x)\dd x$不可能直接求出,就改为求积分$\Int v\dd u=\Int v(x)u'(x)\dd x$, 设$F(x)$是$v(x)\cdot u'(x)$的一个原函数,那么$u(x)\cdot v(x)-F(x)$是$u(x)v'(x)$的一个原函数,用公式(4.18)求不定积分的方法叫做\textbf{分部积分法},公式(4.18)叫做\textbf{分部积分公式}.

\begin{example}
    求$\Int x\sin 3x\dd x$.
\end{example}

\begin{solution}
被积函数$x\sin3x$不能由前面的基本积分表求出它的原函数,我们把$x\sin3x$分成两部分,设$u(x)=x$, $v'(x)=\sin3x$, 于是
\[\int v' (x) \dd x= \int \sin3x\dd x=-\frac{\cos3x}{3}+C\]
即$v(x)=-\frac{\cos3x}{3}$是$v'(x)=\sin3x$的一个原函数,根据公
式(4.18),有
\[\begin{split}
    \int x\sin 3x\dd x&=\int x\dd\left(-\frac{\cos 3x}{3}\right)\\
    &=x\left(-\frac{\cos 3x}{3}\right)-\int\left(-\frac{\cos3x}{3}\right)\dd x \\
&=-\frac{x\cos 3x}{3}+\frac{1}{3}\int \cos 3x\dd x\\
&=-\frac{x\cos 3x}{3}+\frac{\sin 3x}{9}+C
\end{split}\]
\end{solution}

运用公式(4.18)时,应首先将被积表达式分成函数$u(x)$和微分$\dd v=v'(x)\dd x$两部分,选取$u(x)$和$v'(x)\dd x$的原则是:
\begin{itemize}
    \item 充当微分部分的$v'(x)\dd x$, 其$v'(x)$的原函数能够比较容易地求出;
\item 
充当$u(x)$的部分经求微商后要化简,以便使积分$\Int v\dd u$较$\Int u\dd v$简单,容易求出结果.
\end{itemize}

在应用分部积分法求不定积分时,往往要连续使用这个
方法多次才能求出最后结果.

我们把能够应用分部积分法求积分的例题分为以下几类:

\begin{blk}
 {类型A} 被积函数是$x^ne^{ax}$, $x^n\sin ax$, $x^n\cos ax$, 其中$x$的方幂因子经求微商后降次,而非$x$的方幂因子的原函数是容易求得的.   
\end{blk}

\begin{example}
求$\Int x^2 e^{2x}\dd x$.
\end{example}

\begin{solution}
\[\begin{split}
    \Int x^2 e^{2x}\dd x=\int x^2 \dd\left(\frac{e^{2x}}{2}\right)&=\frac{x^2e^{2x}}{2}-\int \frac{e^{2x}}{2}\dd(x^2)\\
&=\frac{x^2e^{2x}}{2}-\int xe^{2x}\dd x=\frac{x^2e^{2x}}{2}-\left(\frac{xe^{2x}}{2}-\int \frac{e^{2x}}{2}\dd x\right)\\
&=\frac{x^2}{2}e^{2x}-\frac{x}{2}e^{2x}+\frac{e^{2x}}{4}+C\\    
&=e^{2x}\left(\frac{x^2}{2}-\frac{x}{2}+\frac{1}{4}\right)+C
\end{split}\]
\end{solution}

\begin{blk}
    {类型B} 被积函数是$x^n\sin^{-1}x$, $x^n\tan^{-1}x$, $x^n\ln x$, 其中乘积中的因子$\tan^{-1} x$, $\ln x$的导数是$x$的有理函数,$\sin^{-1}x$的
导数是$\frac{1}{\sqrt{1-x^2}}$,
它们的积分可求,而$x^n$的原函数是$\frac{x^{n+1}}{n+1}$.
\end{blk}


\begin{example}
求$\Int x\arctan x\dd x$.    
\end{example}

\begin{solution}
  \[\begin{split}
    \Int x\arctan x\dd x=\int \arctan x \dd\left(\frac{x^2}{2}\right)&=\frac{x^2\arctan x}{2}-\int \frac{x^2}{2}\dd(\arctan x)\\
    &=\frac{x^2\arctan x}{2}-\frac{1}{2}\int \frac{x^2}{1+x^2}\dd x\\
    &=\frac{x^2\arctan x}{2}-\frac{1}{2}\int \left(1-\frac{1}{1+x^2}\right)\dd x\\
    &=\frac{x^2\arctan x}{2}-\frac{1}{2}x+\frac{1}{2}\arctan x+C
  \end{split}\]  
\end{solution}


\begin{example}
    求$\Int x^3\ln x\dd x$.
\end{example}

\begin{solution}
\[\begin{split}
    \Int x^3\ln x\dd x=\int \ln x\dd\left(\frac{x^4}{4}\right)&=\frac{x^4}{4}\ln x-\int \frac{x^4}{4}\dd(\ln x)\\
    &= \frac{x^4}{4}\ln x-\int  \frac{x^4}{4}\cdot \frac{1}{x}\dd x\\
    &= \frac{x^4}{4}\ln x- \frac{x^4}{16}+C
\end{split}\]
\end{solution}

\begin{blk}
    {类型C} 被积函数是$\sin^{-1}x$, $\tan^{-1}x$, $\ln x$, 这里没有乘积的形式,不妨设$v'(x)=1$.
\end{blk}

\begin{example}
求$\Int \arcsin x\dd x$.
\end{example}

\begin{solution}
\[\begin{split}
    \Int \arcsin x\dd x=x\arcsin x-\int x\dd(\arcsin x)&=x\arcsin x-\int\frac{x}{\sqrt{1-x^2}}\dd x  \\
    &=x\arcsin x+\frac{1}{2}\int \frac{\dd(1-x^2)}{\sqrt{1-x^2}}\\
    &=x\arcsin x+\frac{1}{2}\cdot 2\sqrt{1-x^2}+C\\
    &=x\arcsin x+\sqrt{1-x^2}+C
\end{split}\]

另解:设$y=\arcsin x,\; -1\le x\le 1$,则$\sin y=x,\; -\frac{\pi}{2}\le y\le \frac{\pi}{2}$
\[\begin{split}
    \Int \arcsin x\dd x=x\arcsin x-\int x\dd(\arcsin x)&=y\sin y-\int \sin y \dd y\\
    &=y\sin y+\cos y+C\\
    &=x\arcsin x+\sqrt{1-\sin^2 y}+C\\
    &=x\arcsin x+\sqrt{1-x^2}+C
\end{split}\]
\end{solution}

\begin{example}
    求$\Int \ln^2 x\dd x$.
\end{example}

\begin{solution}
    \[\begin{split}
        \Int \ln^2 x\dd x=x(\ln x)^2-\int x\dd (\ln^2 x)&=x(\ln x)^2-\int x\cdot 2\ln x\cdot \frac{1}{x}\dd x\\
        &=x(\ln x)^2-2\int \ln x\dd x\\
        &=x(\ln x)^2-2\left[x\ln x-\int x\d(\ln x)\right]\\
        &=x(\ln x)^2-2x\ln x+2\int \dd x\\
        &=x(\ln x)^2-2x\ln x+2x+C   
    \end{split}\]
\end{solution}

\begin{blk}
    {类型D} 被积函数是$e^{ax}\sin bx$, $\sec^3 x$, $\sqrt{a^2+x^2}$. 对于这类例题,两次使用这个方法可以得到含有这个积分作未知元的一次方程.
\end{blk}

\begin{example}
    求$\Int e^{ax}\sin bx\dd x$
\end{example}

\begin{solution}
\[\begin{split}
    I=\Int e^{ax}\sin bx\dd x
    &=\int \sin bx \dd\left(\frac{e^{ax}}{a}\right)\\
&=\sin bx\cdot \frac{e^{ax}}{a}-\int \frac{e^{ax}}{a}\dd(\sin bx)\\
&=\sin bx\cdot \frac{e^{ax}}{a}-\int \frac{e^{ax}}{a}b\cos bx\dd x\\
&=\sin bx\cdot \frac{e^{ax}}{a}-\frac{b}{a}\int \cos bx\dd\left(\frac{e^{ax}}{a}\right)\\
&=\sin bx\cdot \frac{e^{ax}}{a}-\frac{b}{a}\left[\cos bx\cdot \frac{e^{ax}}{a}-\int \frac{e^{ax}}{a}\dd\left(\cos bx\right)\right]\\
&=\sin bx\cdot \frac{e^{ax}}{a}-\frac{b}{a^2}\cos bx\cdot e^{ax}-\frac{b^2}{a^2}\int e^{ax}\sin bx\dd x\\
&=\sin bx\cdot \frac{e^{ax}}{a}-\frac{b}{a^2}\cos bx\cdot e^{ax}-\frac{b^2}{a^2}I
\end{split}\]
所以:
\[\begin{split}
    I\left(1+\frac{b^2}{a^2}\right)&=\frac{1}{a}\sin bx\cdot e^{ax}-\frac{b}{a^2}\cos bx\cdot e^{ax}\\
    I&=\frac{e^{ax}(a\sin bx-b\cos bx)}{a^2+b^2}+C
\end{split}\]    
\end{solution}

\begin{example}
    求$\Int \sqrt{a^2+x^2}\dd x$.
\end{example}

\begin{solution}
\[    \begin{split}
    I =\int \sqrt{a^{2}+x^{2}} \dd x 
    &=x \sqrt{a^{2}+x^{2}}-\int x \dd\left(\sqrt{a^{2}+x^{2}}\right) \\
    &=x \sqrt{a^{2}+x^{2}}-\int \frac{x^{2}}{\sqrt{a^{2}+x^{2}}} \dd x \\
    &=x \sqrt{a^{2}+x^{2}}-\int \frac{x^{2}+a^{2}}{\sqrt{a^{2}+x^{2}}} \dd x+\int \frac{a^{2}}{\sqrt{a^{2}+x^{2}}} \dd x \\
    &=x \sqrt{a^{2}+x^{2}}-I+a^{2} \ln \left(x+\sqrt{a^{2}+x^{2}}\right)
    \end{split}\]
    所以
\[    I=\frac{1}{2} x \sqrt{a^{2}+x^{2}}+\frac{1}{2} a^{2} \ln \left(x+\sqrt{a^{2}+x^{2}}\right)\]
\end{solution}

应用分部积分公式(4.18)还能导出一些求积分的递归公式.所谓求积分的递归公式就是建立一个被积函数含有正整数$n$的积分与被积函数含有非负整数$n-1$或$n-2$的积分的关系式.每次要求$n=k$时的积分,重复利用这个关系式总能回到当$n=0, 1, 2$时的起始积分,于是所求积分总能用$n$的起始值$n=0, 1, 2$中的一个已知积分表示出来.这也就是递归的
意思.
    
\begin{example}
    求$\Int \sin^n x\dd x$.
\end{example}

\begin{solution}
当$n=0$时,$I_0=\Int (\sin x)^0\dd x=\Int \dd x=x+C$

又:
\[
\begin{split}
I_{n} =\int \sin ^{n} x \dd x&=\int \sin ^{n-1} x \dd(-\cos x) \\
&=-\cos x \sin ^{n-1} x+\int \cos x \dd\left(\sin ^{n-1} x\right) \\
&=-\cos x \sin ^{n-1} x+(n-1) \int \cos x \sin ^{n-2} x \cdot \cos x \dd x \\
&=-\cos x \sin ^{n-1} x+(n-1) \int \sin ^{n-2} x\left(1-\sin ^{2} x\right) \dd x
\end{split}
\]
所以
$$
I_{n}=-\cos x \sin ^{n-1} x+(n-1) I_{n-2}-(n-1) I_{n}
$$
即
\begin{equation}
   n I_{n}=-\cos x \sin ^{n-1} x+(n-1) I_{n-2} 
\end{equation}

现在我们用递归关系(4.19)来求 $\Int \sin ^{4} x \dd x$. 
设 $I_{4}=\Int \sin ^{4} x \dd x$, 于是
\begin{align}
4 I_{4}&=-\cos x \sin ^{3} x+3 I_{2} \\
2 I_{2}&=-\cos x \sin x+I_{0} 
\end{align}
将(4.21)代入(4.20)得:
\[\begin{split}
    4I_4&=-\cos x\sin^3 x+3\left(-\frac{\sin x\cos x}{2}+\frac{1}{2}I_0\right)\\
    &=-\cos x\sin^3 x-\frac{3}{2}\cos x\sin x+\frac{3}{2}x+C
\end{split}\]
所以
\[I_4=\int \sin^4 x\dd x=-\frac{1}{4}\cos x\sin^3 x-\frac{3}{8}\cos x\sin x+\frac{3}{8}x+C\]


\begin{ex}
\begin{enumerate}
    \item 求下列不定积分:
\begin{multicols}{2}
\begin{enumerate}
    \item $\Int x \sin x\dd x$
    \item $\Int x \cos 3 x\dd x$
    \item $\Int x \sin ^{2} \frac{x}{2}\dd x$
    \item $\Int x e^{-x}\dd x$
    \item $\Int (x-1) e^{x}\dd x$
    \item $\Int \left(x^{2}+x\right) e^{-x}\dd x$
    \item $\Int  x \sec ^{2} x\dd x$
    \item $\Int  x \arcsin x\dd x$
    \item $\Int  \frac{\arcsin x}{x^{2}}\dd x$
    \item $\Int  \frac{\arctan x}{x^{2}}\dd x$
    \item $\Int  x^{2} e^{3 x}\dd x$
    \item $\Int  e^{a x} \sin ^{2} b x\dd x$
    \item $\Int  \sin x \cdot \ln (\tan x)\dd x$
    \item $\Int  x^{2} \arctan x\dd x$
    \item $\Int  \frac{x^{2} e^{x}\dd x}{(x+2)^{2}}$
    \item $\Int \frac{x e^{x}}{(x+1)^{2}}\dd x$
\end{enumerate}
\end{multicols}
 \item   求下列不定积分:    
   \begin{multicols}{2}
        \begin{enumerate}
    \item $\Int \arctan \sqrt{x}\dd x$
    \item $\Int \ln ^{2} x \dd {x}$
    \item $\Int \ln \left(x+\sqrt{1+x^{2}}\right)\dd x$
    \item $\Int (\arcsin x)^{2}\dd x$
    \item $\Int \sin(\ln x)\dd x$
    \item $\Int \sqrt{4+x^2}\dd x$
\end{enumerate}
\end{multicols}
\end{enumerate}
\end{ex}
\end{solution}
    
\section{定积分的计算与应用}

\subsection{定积分的计算}

由微积分的基本定理得知,定积分$\Int^b_a f(x)\dd x$的计算归结为不定积分$\Int f(x)\dd x$的计算,由后者计算得到$f(x)$的原函数$F(x)$, 则
\[\Int^b_a f(x)\dd x=F (x)\Bigg|^b_a=F (b) -F (a)\] 
于是计算不定积分的各种方法也可移植过来求定积分.

\subsubsection{定积分的换元积分法}

第一换元法:设$\Int f(u)\dd u=F(u)$, $u=\varphi(x)$可微,
那么
\[\begin{split}
    \int^b_a f(\varphi(x))\varphi'(x)\dd x&=\int^{\varphi(b)}_{\varphi(a)}f(u)\dd u\\
    &=F(u)\Big|^{\varphi(b)}_{\varphi(a)}=F[{\varphi(b)}]-F[{\varphi(a)}]
\end{split}\]

\begin{example}
    求$\Int^{\tfrac{\pi}{2}}_0 \sin^3x \cos x\dd x$.
\end{example}

\begin{solution}
    因为
\[\int \sin^3x\cos x\dd x= \int \sin^3x\dd (\sin x) =\frac{1}{4}\sin^4x+C\]
所以
\[\begin{split}
    \Int^{\tfrac{\pi}{2}}_0 \sin^3x \cos x\dd x&=\frac{1}{4}\sin^4 x\Big|^{\tfrac{\pi}{2}}_0 \\
    &=\frac{1}{4}\left[\sin^4\frac{\pi}{4}-0\right]=\frac{1}{4}
\end{split}\]
\end{solution}

\begin{blk}
 {定理1(第二种变量代换法)}设$f(x)$在$[a,b]$上连续,作代换$x=\varphi (t)$, 其中$\varphi (t)$在某一闭区间$[\alpha,\beta]$上有连续导数$\varphi '(t)$, 当$\alpha\le t\le \beta$时,$a\le \varphi (t)\le b$, 且$\varphi (\alpha)=a$, $\varphi (\beta)=b$, 那么
\[\int^b_a f(x) \dd x= \int^{\beta}_{\alpha} f [\varphi  (t) ] \varphi ' (t) \dd t\]   
\end{blk}

\begin{proof}
    设$f(x)$的一个原函数是$G(x)$, 即有
\[\frac{\dd}{\dd x}G (x) =f (x)\]
于是由微积分基本定理得
\begin{equation}
    \int^b_a f (x)\dd x=G(x)\Big|^b_a=G(b) -G (a)
\end{equation}
根据复合函数求导法则,知
\[\begin{split}
    \frac{\dd}{ \dd t}G[\varphi  (t) ] &=\frac{\dd}{ \dd x}G (x)\cdot  \frac{\dd}{ \dd t}\varphi  (t)\\
    &=f(x)\cdot \varphi'(t)=f[\varphi(t)]\varphi'(t)
\end{split}\]
所以$G[\varphi (t)]$是$f[\varphi (t)]\varphi '(t)$的一个原函数,从而
\begin{equation}
    \begin{split}
\int^{\beta}_{\alpha} f [\varphi  (t) ] \varphi ' (t) \dd t&=G [\varphi  (t) ]\Big|^{\beta}_{\alpha}\\
&=       G [\varphi  (\beta) ] -G [\varphi  (\alpha) ]=G (b) -G (a)
    \end{split}
\end{equation}
因为(4.22)(4.23)两个积分相等,所以
\[\int^b_a f(x) \dd x= \int^{\beta}_{\alpha} f [\varphi  (t) ] \varphi ' (t) \dd t\]
\end{proof}

这个公式与不定积分的换元公式很类似,所不同的是,后者最后需要将变量还原,而现在只须把积分限作相应的改变.

\begin{example}
    求$\Int^a_0 \sqrt{a^2-x^2}\dd x$
\end{example}

\begin{solution}
作代换$x=a\sin t$,当$x$从0变到$a$时,相应地$t$自0变到$\frac{\pi}{2}$,于是
\[\begin{split}
    \Int^a_0 \sqrt{a^2-x^2}\dd x=a^2\int^{\tfrac{\pi}{2}}_0 \cos ^2 t\dd t&=a^2\int^{\tfrac{\pi}{2}}_0 \frac{1+\cos 2t}{2}\dd t\\
    &=\frac{a^2}{2}\left(t+\frac{1}{2}\sin 2t\right)\Bigg|^{\tfrac{\pi}{2}}_0=\frac{\pi a^2}{4} 
\end{split}\]
\end{solution}

\begin{example}
    求$\Int^4_0 \frac{1}{1+\sqrt{x}}\dd x$
\end{example}

\begin{solution}
    设$x=u^2$,于是
\[\frac{1}{1+\sqrt{x}}=\frac{1}{1+u},\qquad \dd x=2u\dd u\]
定限:当$x=0$时,从变换$\sqrt{x}=u$中解出$u=0$;当$x=4$时,解出$u=2$. 于是
\[\begin{split}
    \Int^4_0 \frac{1}{1+\sqrt{x}}\dd x=\int^2_0 \frac{1}{1+u}\cdot 2u\dd u&=2\int^2_0 \left(1-\frac{1}{1+u}\right)\dd u\\
    &=2\left[u-\ln(1+u)\right]\Big|^2_0=2(2-\ln 3)
\end{split}\]
\end{solution}

\begin{example}
求证:$\Int^a_0 f(x)\dd x=\Int^a_0 f(a-x)\dd x$
\end{example}

\begin{proof}
    设$x=a-t$,则$\dd x=-\dd t$,且当$x=0$时,$t=a$;当$x=a$时,$t=0$. 于是:
\[\begin{split}
    \Int^a_0 f(x)\dd x=\int^0_a f(a-t)(-\dd t)&=-\int^0_a f(a-t)\dd t\\
    &=\int^a_0 f(a-t)\dd t=\Int^a_0 f(x)\dd x
\end{split}\]
\end{proof}

我们在第三节曾指出,使用第二种变量代换时,要注意条件是否满足,否则会出现矛盾,这里举一反例.

求$\Int^{1}_{-1}\frac{\dd x}{1+x^2}$时,如作变换$x=\frac{1}{t}$,则
\[\Int^{1}_{-1}\frac{\dd x}{1+x^2}=-\Int^{1}_{-1}\frac{\dd t}{1+t^2}\]
于是$\Int^{1}_{-1}\frac{\dd x}{1+x^2}=0$,这当然不对,问题在于,代换$x=\frac{1}{t}$
在$[-1,1]$上并不是连续变化的.


\subsubsection{定积分的分部积分法}
\begin{blk}
    {定理2} 设$u=u(x)$, $v=v(x)$在$[a,b]$上有连续的导数$u'(x)$, $v'(x)$, 则有分部积分公式
\[\int^b_a u\dd v=u\cdot v\Big|^b_a- \int^b_a v\dd u\]
\end{blk}

\begin{proof}
    由牛顿-莱布尼兹公式,显然有
\[\begin{split}
    u (x) v (x)\Big|^b_a &=\int^b_a (u (x) v (x) ) '\dd x\\
&=\int^b_a u' (x) v (x) \dd x+\int^b_a u (x) v' (x) \dd x
\end{split}\]
移项,得
\[\int^b_a u (x) \dd [v (x) ] =u (b) v (b) -u (a) v (a) -\int^b_a v (x) \dd [u (x) ] \]
\end{proof}

这个公式与不定积分的分部积分公式很相似,但是,这里每一项都带着积分限.

\begin{example}
求$\Int^{\sqrt{3}}_{-\sqrt{3}}|\arctan x|\dd x$
\end{example}

\begin{solution}
因为$|\arctan x|$是偶函数,所以
\[I=\Int^{\sqrt{3}}_{-\sqrt{3}}|\arctan x|\dd x=2\Int^{\sqrt{3}}_{0}|\arctan x|\dd x=2\Int^{\sqrt{3}}_{0}\arctan x\dd x\]
设$u(x)=\arctan x$, $v'(x)=1$, 则$v(x)=x$,利用分部积分公式,得到:
\[\begin{split}
    I=2\Int^{\sqrt{3}}_{0}\arctan x\dd x &=2\left[x\arctan x\Big|^{\sqrt{3}}_0-\int^{\sqrt{3}}_0 x\dd (\arctan x)\right]\\
    &=2\left[\sqrt{3}\arctan\sqrt{3}-\int^{\sqrt{3}}_0 x\cdot \frac{1}{1+x^2}\dd x \right]\\
    &=2\sqrt{3}\cdot \frac{\pi}{3}-\int^{\sqrt{3}}_0 \frac{\dd(1+x^2)}{1+x^2}\\
    &=\frac{2\sqrt{3}}{3}\pi -\ln(1+x^2)\Big|^{\sqrt{3}}_0= \frac{2\sqrt{3}}{3}\pi -\ln 4
\end{split}\]
\end{solution}


\begin{example}
    求$\Int^{\tfrac{\pi}{2}}_0 \sin^n \theta\dd \theta$.
\end{example}

\begin{solution}
    设$I_n=\Int^{\tfrac{\pi}{2}}_0 \sin^n \theta\dd \theta$,利用例4.42的结果,得到:
\[nI_n=\left[-\cos\theta\sin^{n-1}\theta\right]\Big|^{\tfrac{\pi}{2}_0} +(n-1)I_{n-2}\]
因为方括号的式子对于积分上,下限的值等于零,所以上面等式化简为
\[I_n=\frac{n-1}{n}I_{n-2}\]
分别考虑$n$是偶数和奇数两种情况:
\begin{enumerate}
    \item 若$n$是偶数,那么
\[\begin{split}
    I_n=\frac{n-1}{n}I_{n-2}&=\frac{n-1}{n}\cdot \frac{n-3}{n-2}I_{n-4}\\
    &=\cdots=\frac{n-1}{n}\cdot \frac{n-3}{n-2}\cdot \frac{n-5}{n-4}\cdots \frac{1}{2}I_{0}
\end{split}
    \]
此处$I_0=\Int^{\tfrac{\pi}{2}}_0 1\cdot \dd \theta=\frac{\pi}{2}$,所以
\[I_n=\frac{(n-1)(n-3)\cdots 3\cdot 1}{n(n-2)\cdots 4\cdot 2}\cdot \frac{\pi}{2}\]

\item 若$n$是奇数,由于
\[I_1=\Int^{\tfrac{\pi}{2}}_0 \sin\theta\dd\theta =(-\cos\theta)\Big|^{\tfrac{\pi}{2}}_0=1\]
所以
\[I_n=\frac{(n-1)(n-3)\cdots 4\cdot 2}{n(n-2)\cdots 5\cdot 3}\]
\end{enumerate}
\end{solution}

\begin{ex}
\begin{enumerate}
    \item 求下列定积分:
\begin{multicols}{2}
\begin{enumerate}
    \item $\Int_{0}^{\tfrac{\pi}{4}}(1+\sin \theta)^{2} \dd \theta $ 
    \item $\Int_{0}^{\tfrac{\pi}{4}} \frac{\cos \theta \dd \theta}{1+\sin ^{2} \theta}$
    \item $\Int_{0}^{1} \frac{x^{2}}{\left(1+x^{2}\right)^{2}} \dd x$
    \item $\Int_{0}^{\tfrac{1}{\sqrt{2}}} \frac{x^{2}}{\sqrt{1-x^{2}}} \dd x$
    \item $\Int_{\tfrac{1}{\sqrt{2}}}^{1} \sqrt{1-x^{2}} \dd x$
    \item $\Int_{0}^{x} \frac{\cos ^{2} x \dd x}{a^{2} \cos ^{2} x+b^{2} \sin x}$
    \item $\Int_{0}^{2 n}[\sin \omega t -\sin (\omega t+\varphi)]^{2} \dd t$
    \item $\Int_{2}^{4} \sqrt{(4-x)(x-2)} \dd x$
    \item $\Int_{1}^{5} \frac{1}{\sqrt{2 x-1}} \dd x$ 
    \item $\Int_{0}^{1} \frac{x+1}{\sqrt{1 - x^{2}}} \dd x$
    \item $\Int_{\tfrac{\pi}{6}}^{\tfrac{\pi}{2}} \cot \theta \dd \theta$
    \item $\Int_{1}^{0} \frac{x^{2}}{ 1+x^{3}} \dd x$
    \item $\Int^{\tfrac{\pi}{2}}_0 \frac{\sin x}{\cos^2 x}\dd x$
    \item $\Int^{2a}_0 x^2\sqrt{2ax-x^2}\dd x$
\end{enumerate}
\end{multicols}

\item 证明:$\Int^{\tfrac{\pi}{2}}_0 \frac{\cos x}{\cos x+\sin x}\dd x=\Int^{\tfrac{\pi}{2}}_0 \frac{\sin x}{\cos x+\sin x}\dd x$

(提示:用变量代换$x=\frac{\pi}{2}-y$.)
\item \begin{enumerate}
    \item 求常数$A,B,C,D$使等式$$\frac{1+2x^2}{(1-x)(1+x)^2}=A+\frac{B}{1-x}+\frac{C}{(1+x)^2}+\frac{D}{1+x}$$成立,并计算$$\Int^{\tfrac{1}{2}}_0 \frac{1+2x^3}{(1-x)(1+x)^2}\dd x$$
    \item 求常数$a$,使$\Int^1_0 \frac{x-a}{(x+1)(3x+1)}\dd x=0$
\end{enumerate}
\item 求极限:$\Lim_{n\to\infty}\left(\frac{1}{n+1}+\frac{1}{n+2}+\cdots +\frac{1}{n+n}\right)$
\item 求一三次多项式$f(x)=ax^3+bx^2+cx+d$, 使它满足下
列条件:
\begin{enumerate}
    \item $f (-1) =f (1) =0$
    \item $\Int^1_{-1} f(x)\dd x=6$
\end{enumerate}
\item 
\begin{enumerate}
\item 在同一个坐标系内,作$y=(x-1)(x-2)$和$y=\Int^x_0 (t-1)(t-2)\dd t$的函数图象.
\item 求函数$\Int^x_0 (t-1)(t-2)\dd t$在闭区间$[0, 3]$上的最大值和最小值;
\item 求在曲线$y=(x-1)(x-2)$, $x$轴和纵坐标$x=0$, $x=3$之间的区域的面积.
\end{enumerate}
\item 
\begin{enumerate}
    \item 若$f(2a-x)=-f(x)$, 求证$\Int^{2a}_0 f (x) \dd x=0$;
\item 
若$f(2a-x)=f(x)$, 求证$\Int^{2a}_0 f (x) \dd x=2\int^a_0 f (x) \dd x$. 
\item 试计算$\int^{\pi}_0 \sin^6 x\cos^3x\dd x$和$\Int^{\pi}_0 \sin^3x \cos^2 x\dd x$.
\end{enumerate}

\item 求$\Int^{\tfrac{\pi}{2}}_0 \cos^n \theta\dd \theta$
\item 求$\Int^{\tfrac{\pi}{4}}_0 x^3\cos2x\dd x$
\item 求$\Int^{\tfrac{3\pi}{2}}_0 \sin^3\theta \cos^2 \theta\dd \theta$
\end{enumerate} 
\end{ex}

\subsection{平面图形的面积}
我们已经证明过在曲线$y=f(x)$与$x$轴,$x=a$, $x=b$之间的曲边梯形的有号面积等于
\[\int^b_a y\dd x=\int^b_a f(x)\dd x\]
对于任意的平面图形.我们总可以取适当的直角坐标系,使得它的边界为两条曲线
\[y=f (x) ,\qquad  y=g (x) ,\quad x\in [a,b]\]
所描述,现在我们来求这样的区域$R$的面积(图4.4).

\begin{blk}
   {定理} 若$f$和$g$是在区间$[a,b]$上的连续函数,而且$f(x)\ge g(x)$, 那么在曲线$y=f(x)$, $y=g(x)$和$x=a$, $x=b$之间的区域$R$的面积等于
\[A=\int^b_a [f (x) -g (x) ] \dd x\] 
\end{blk}

\begin{figure}[htp]\centering
	\begin{minipage}[t]{0.48\textwidth}
		\centering
        \begin{tikzpicture}[>=latex, scale=.8]
\draw[->](-2.5,0)--(3,0)node[right]{$x$};
\draw[->](0,-3)--(0,3)node[right]{$y$};
\draw[domain=-2:2, samples=100, very thick]plot(\x, {.15*(-\x*\x-4*\x +5) })node[right]{$y=f(x)$};
\draw[domain=-2:2, samples=100, very thick]plot(\x, {.1*(\x*\x-4*\x -15)})node[right]{$y=g(x)$};
\draw[domain=-2:2, samples=100, pattern=north east lines, dashed]plot(\x, {.15*(-\x*\x-4*\x +5) })--plot(-\x, {.1*(\x*\x+4*\x -15)});
\node at (0,0)[below left]{$O$};
\draw[dashed](-1.75,-2)--(-1.75,2)node[above]{$x=a$};
\draw[dashed](1.75,-2)--(1.75,2)node[above]{$x=b$};



    \end{tikzpicture}
    \caption{}
\end{minipage}
\begin{minipage}[t]{0.48\textwidth}
    \centering
\begin{tikzpicture}[>=latex, scale=.8]
\draw[->](-2.5,0)--(3,0)node[right]{$x$};
\draw[->](0,-3)--(0,3)node[right]{$y$};
\draw[domain=-2:2, samples=100, very thick]plot(\x, {.15*(-\x*\x-4*\x +5) })node[right]{$y=f(x)-k$};
\draw[domain=-2:2, samples=100, very thick]plot(\x, {.1*(\x*\x-4*\x -15)})node[right]{$y=g(x)-k$};
\draw[domain=-2:2, samples=100, pattern=north east lines, dashed]plot(\x, {.15*(-\x*\x-4*\x +5) })--plot(-\x, {.1*(\x*\x+4*\x -15)});
\node at (0,0)[below left]{$O$};

\node at (0,3)[left]{$y'$};
\draw[thick, ->](-2.5,-2.5)--(3,-2.5)node[right]{$x'$};
\draw[dashed](-1.75,-2.5)node[below]{$a$}--(-1.75,2);
\draw[dashed](1.75,-2.5)node[below]{$b$}--(1.75,2);
\node at (0,-2.5)[below right]{$O'$};
\end{tikzpicture}
\caption{}
\end{minipage}
\end{figure}

\begin{proof}
让我们选择一个常数$k$使其小于函数$g$在区间$[a,b]$上的最小值,则对于每一个$x\in [a,b]$, $g(x)
-k\ge 0$.

又因为$f(x)\ge g(x)$, 因此对于每一个$x\in [a,b]$, 
也有$f(x)-k\ge 0$. 将坐标系$x,y$平行于$y$轴平移到新坐标系$x',y'$, 使新原点在原来坐标系中的坐标为$(0,k)$ (图4.5), 这个坐标系的平移由变换
\[\begin{cases}
    x=x'\\
    y=y'+k
\end{cases}\]
实现,于是原来的函数图象在新坐标的方程是
\[f_1 (x) =f (x) -k,\qquad g_1 (x) =g (x) -k\]
函数$f_1$和$g_1$在区间$[a,b]$上仍连续且取非负值,显然在函数$f_1$和$g_1$之间的区域$R=\{a\le x\le b,\; g_1(x)\le y\le f_1(x)\}$就是原来的区域$R=\{a\le x\le b,\; g(x)\le y\le f(x)\}$, 但是,$R'$的区域的面积等于
\[A=\int^b_a f_1(x)\dd x-\int^b_a g_1(x)\dd x\]
根据定积分的性质进一步得到
\[\begin{split}
    A&=\int^b_a [f_1(x)-g_1(x)]\dd x\\
    &=\int^b_a \Big\{\big[f(x)-k\big]-\big[g(x)-k\big]\Big\}\dd x\\
    &=\int^b_a [f(x)-g(x)]\dd x\\
\end{split}\]
\end{proof}

\begin{example}
求直线$y=x-2$和抛物线$y=2x-x^2$所围成的图形的面积$A$.
\end{example}

\begin{solution}
直线与抛物线交于$(2, 0)$, $(-1,-3)$两点,作出它的草图(图4.6). 依本节定理,得到
\[\begin{split}
    A=\int^2_1 [(2x-x^2)-(x-2)]\dd x
    &=\int^2_1 (-x^2+x+2)\dd x\\
    &=\left[-\frac{1}{3}x^3+\frac{1}{2}x^2+2x\right]\Bigg|^2_1=\frac{9}{2}
\end{split}\]
\end{solution}


\begin{figure}[htp]\centering
	\begin{minipage}[t]{0.48\textwidth}
		\centering
        \begin{tikzpicture}[>=latex, scale=.7]
\draw[->](-2,0)--(5,0)node[right]{$x$};
\draw[->](0,-5)--(0,2.5)node[right]{$y$};
\draw[domain=-1.25:3.25, samples=100, very thick]plot(\x, {2*\x-\x*\x});
\draw[domain=-2:4, samples=10, very thick]plot(\x, {\x-2});
\node at (1,1)[above]{$(1,1)$};
\node at (2,0)[below right]{$(2,0)$};
\node at (-1,-3)[left]{$(-1,-3)$};
\node at (0,0)[above left]{$O$};
\draw[domain=-1:2, samples=100, pattern=north west lines]plot(\x, {2*\x-\x*\x})--(-1,-3);

    \end{tikzpicture}
    \caption{}
\end{minipage}
\begin{minipage}[t]{0.48\textwidth}
    \centering
\begin{tikzpicture}[>=latex, scale=.8]
    \draw[->](-2,0)--(3.5,0)node[right]{$x$};
\draw[->](0,-3.5)--(0,3.5)node[right]{$y$};
\node at (0,0)[below left]{$O$};
\draw[domain=-2.5:2.5, samples=100, very thick]plot({2-0.5*\x*\x}, \x);
\draw[domain=-2.75:2.75, samples=100, very thick]plot({1-0.25*\x*\x}, \x);
\foreach \x in {2,-2}
{
    \node at (0,\x) [left]{$\x$};
}
\node at (1,0)[below left]{1};
\node at (2,0)[below right]{2};
\draw[domain=-2:2, samples=100, pattern=north east lines]plot({2-0.5*\x*\x}, \x)--plot({1-0.25*\x*\x}, -\x);


\end{tikzpicture}
\caption{}
\end{minipage}
\end{figure}


\begin{example}
求抛物线$y^2=-4(x-1)$, $y^2=-2(x-2)$所围成的图形的面积.
\end{example}

\begin{solution}
    先作出它们的草图(图4.7), 且解得它的两个交点$(0, 2)$和$(0,-2)$. 于是
\[\begin{split}
    A=\int^2_{-2} \left[\left(2-\frac{1}{2}y^2\right)-\left(1-\frac{1}{4}y^2\right)\right]\dd y
    &=\frac{1}{4}\int^2_{-2} (4-y^2)\dd y\\
    &=\frac{1}{4}\left[4y-\frac{1}{3}y^3\right]\Bigg|^2_{-2}=\frac{8}{3}
\end{split}\]
\end{solution}

\begin{example}
求半径等于$r$, 中心角等于$\alpha,\; \left(0<\alpha\le \frac{\pi}{2}\right)$的扇形的面积.
\end{example}

\begin{figure}[htp]
    \centering
\begin{tikzpicture}[>=latex, scale=1.5]
\draw[->](-1.5,0)--(2,0)node[right]{$x$};
\draw[->](0,-1.5)--(0,1.5)node[right]{$y$};
\draw[very thick] (0,0) circle (1);
\draw[pattern=north west lines](0,0)--(1,0)node[below right]{$B$} arc (0:60:1)node[above right]{$P(r\cos\alpha, r\sin\alpha)$}--(0,0);
\node at (0,0)[below left]{$O$};
\draw(.5,0)node[below]{$C$}--(60:1);
\draw[<->](.2,0) arc (0:60:.2)node[right]{$\alpha$};

\end{tikzpicture}
    \caption{}
\end{figure}

\begin{solution}
    如图4.8, 扇形$OBP$的区域划分为$\triangle OCP$的区域与曲边三角形$CBP$的区域的和,因此
\[A (OBP) =A (OCP) +A (CBP) \]
因为$P$点坐标是$(r\cos\alpha, r\sin\alpha)$,所以
\[A(OCP)=\frac{1}{2}(r\cos\alpha)\cdot (r\sin\alpha)=\frac{r^2}{4}\sin 2\alpha\]
又扇形的圆方程是$x^2+y^2=r^2$,所以
\[A(CBP)=\int^r_{r\cos \alpha}\sqrt{r^2-x^2}\dd x\]
从而
\[A(OBP)=\frac{r^2}{4}\sin2\alpha+\int^r_{r\cos \alpha}\sqrt{r^2-x^2}\dd x\]

令$x=r\sin\theta,\; \theta\in\left[-\frac{\pi}{2},\frac{\pi}{2}\right]$,则
\[\theta=\arcsin\frac{x}{r},\qquad \dd x=r\cos\theta \dd\theta\]
且当$x=r\cos \alpha$时,
\[\theta=\arcsin(\cos\alpha)=\arcsin\left[\sin\left(\frac{\pi}{2}-\alpha\right)\right]=\frac{\pi}{2}-\alpha\]
当$x=r$时,$\theta=\arcsin 1=\frac{\pi}{2}$

所以
\[\begin{split}
    A(OBP)&=\frac{r^2}{4}\sin2\alpha+\int^{\tfrac{\pi}{2}}_{\tfrac{\pi}{2}-\alpha}(r\cos\theta)(r\cos\theta)\dd \theta\\
&=\frac{r^2}{4}\sin2\alpha+r^2\int^{\tfrac{\pi}{2}}_{\tfrac{\pi}{2}-\alpha}\cos^2\theta\dd\theta\\ 
&=\frac{r^2}{4}\sin2\alpha+r^2\left.\left[\frac{\theta}{2}+\frac{\sin 2\theta}{4}\right]\right|^{\tfrac{\pi}{2}}_{\tfrac{\pi}{2}-\alpha}\\
&=\frac{r^2}{2}\alpha
\end{split}\]

当$\alpha=2\pi$时,由上面公式就得到圆面积$\pi r^2$.
\end{solution}

\begin{example}
$P$点是位于曲线$y=x^2$上的动点,从原点$O$向点$A(1, 1)$移动,过二点$O,P$引直线,它与曲线$y=x^2$及直线$x=1$所围图形的面积为图4.9中阴影部分所示,求使这部分面积最小时$P$点的坐标.
\end{example}

\begin{figure}[htp]\centering
        \begin{tikzpicture}[>=latex, scale=3]
\draw[->](-.2,0)--(1.5,0)node[right]{$x$};
\draw[->](0,0)node[below]{$O$}--(0,1.5)node[right]{$y$};
\draw(0,1)--(1,1)node[right]{$A(1,1)$}--(1,0);
\draw[domain=0:1.15, samples=100, very thick]plot(\x, {\x*\x})node[above right]{$y=x^2$};
\draw[domain=0:1, samples=100, thick, pattern=north west lines]plot(\x, {\x*\x})--(1,.8)--(0,0);
\node at (.8,.64)[left]{$P(t,t^2)$};
        \end{tikzpicture}
        \caption{}
    \end{figure}

\begin{solution}
    设$P$点的坐标是$(t,t^2),\; (0\le t\le 1)$. 直线$OP$的方程是$y=tx$,设阴影部分的面积为$S$,则
\[\begin{split}
S&=\int^t_0 (tx-x^2)\dd x+\int^1_t (x^2-tx)\dd x\\
&=\left[\frac{t}{2}x^2-\frac{x^3}{3}\right]\Bigg|^t_0+\left[\frac{x^3}{3}-\frac{t}{2}x^2\right]\Bigg|^1_t\\
&=\frac{t^3}{3}-\frac{t}{2}+\frac{1}{3}
\end{split}\]
\[S'=t^2-\frac{1}{2}=\left(t+\frac{\sqrt{2}}{2}\right)\left(t-\frac{\sqrt{2}}{2}\right)\]
由$S'$符号的变化,当$t=\frac{\sqrt{2}}{2}$时,$S$极小也就是最小,故所求点的坐标是$\left(\frac{\sqrt{2}}{2},\frac{1}{2}\right)$.
\end{solution}

\begin{example}
    求椭圆$\frac{x^2}{a^2}+\frac{y^2}{b^2}=1\; (a>b>0)$的面积$A$(图4.10).
\end{example}

\begin{figure}[htp]
    \centering
\begin{tikzpicture}[>=latex]
    \draw[->](-3,0)--(3,0)node[right]{$x$};
\draw[->](0,-2)--(0,2)node[right]{$y$};
\node at (0,0) [below left]{$O$};
\draw[very thick](0,0) ellipse [x radius=2.2, y radius=1.3];

\end{tikzpicture}
    \caption{}
\end{figure}

\begin{solution}
由于图形的对称性,只要求出椭圆在第一象限内的面积,再乘以4就是该椭圆的面积,所以
\[\frac{A}{4}=\int^a_0 y\dd x\]
这里将椭圆用参数方程
$\begin{cases}
    x=a\cos t\\ y=b\sin t
\end{cases}$
表出,当$x=a$时,$t=\arccos 1=0$; 当$x=0$时,$t=\arccos 0=\frac{\pi}{2}$, 于是
\[\begin{split}
    \frac{A}{4}=\int^0_{\tfrac{\pi}{2}}b\sin t \dd(a\cos t)&=\int^0_{\tfrac{\pi}{2}}(b\sin t )(-a\sin t)\dd t\\
&=ab \int^{\tfrac{\pi}{2}}_0 \sin^2 t\dd t=\frac{1}{4}ab\pi 
\end{split}\]
所以:$A=\pi ab$.
\end{solution}

\begin{ex}
\begin{enumerate}
    \item 求在曲线$y=\sin2x$, $y=0$, $x\in[0, 2\pi]$之间的面积.
    \item 求下列曲线所围图形的面积:
\begin{enumerate}
    \item $y=8-\frac{1}{2}x^2$和$y=3.2$ 
    \item $y=x^2-5x+7$和$y=-2x^2+10x-5$
    \item $y=\sin x$和$y=\frac{2}{\pi} x,\; x\ge 0$
\end{enumerate}
\item 求以曲线$y=\sin x\; \left(0\le x\le \frac{\pi}{4}\right)$;$y=\cos x\; \left(\frac{\pi}{4}\le x\le \frac{\pi}{2}\right)$和$y=0$为边界的区域的面积.

\item \begin{enumerate}
    \item 证明点$x=t^2-1$, $y=t(t^2-1)$在曲线$y^2=x^2(x+1)$上;
    \item 当$t$由$t=-2$变化到$t=2$时,作出曲线$\begin{cases}
        x=t^2-1\\x=t (t^2-1)
    \end{cases}$的图象;
\item 求由曲线的环路所围区域的面积.
\end{enumerate}

\item 在曲线$y=x^2-2x+2$与过点$(2, 3)$的直线所围的图形
中,求图形面积最小时的直线方程.
\item 求抛物线$y=x^2$及过此抛物线上点$(2, 4)$的切线和$x$轴所围图形的面积.
\item 已知曲线$C:\; y=x^2-7x+10\; (x>0)$与点$A(0, 1)$, 
回答下列问题:
\begin{enumerate}
    \item 求过点$A$所引曲线$C$的切线方程以及切点$T$的坐标.
    \item 设$S$是直线$x=1$与曲线$C$的交点.若曲线$C$上的动点$P$从$S$点运动到$T$点时,将线段$AP$所在的区域用阴影表示出来,并求出这个区域的面积.
\end{enumerate}

\item 求通过$(0, 0)$, $(1, 2)$两点的抛物线,要求它具有以
下的性质:
\begin{enumerate}
    \item 它的对称轴平行于$y$轴,且凸向上,
    \item 它与$x$轴所围区域的面积最小.
\end{enumerate}

\item 对称轴平行$y$轴的抛物线与圆$x^2+y^2=4$切于$A(0, 2)$点并通过$B(-2, 0)$点,计算抛物线与圆所围区域的面积.
\end{enumerate}
\end{ex}

\subsection{利用横断面算体积法}
对于空间的一个立体,我们取一条直线作为坐标轴,取直线上一点作为原点(图4.11),假定这立体紧夹于在$x=a$与$x=b$两点所作的$x$轴的垂直平面之间,又如果在离原点$x$处作这轴的一个垂直平面,并且该垂直平面截得立体的截面的面积是$x$的连续函数$A(x),\; a\le x\le b$, 那么立体的体积可由定积分来计算:
\begin{figure}[htp]
    \centering
\begin{tikzpicture}[>=latex]
    \draw[->](-.5,0)--(6,0)node[right]{$x$};
\tkzDefPoints{.8/0/a, 2/0/A, 3.3/0/x_i, 5/0/b, 0/0/O}
\tkzDrawPoints(a,A,x_i,b,O)
\tkzLabelPoints[below](a,x_i,b,O)
\tkzLabelPoint[below](A){$x_{i-1}$}
\node at (0,0){
\pscurve[linewidth=1pt](.8,0)(1.2,.8)(2,1.3)(3.3,1.15)(5,0)(3.3,-1.15)(2,-1.3)(1.2,-.8)(.8,0)
};
\draw[dashed](A) ellipse [x radius=.35, y radius=1.3]; 
\draw[dashed](x_i) ellipse [x radius=.3, y radius=1.15]; 
\draw[thick](2,1.3) arc [start angle =90, end angle =270, x radius=.35, y radius=1.3];
\draw[thick](3.3,1.15) arc [start angle =90, end angle =270, x radius=.3, y radius=1.15];

\end{tikzpicture}
    \caption{}
\end{figure}

\begin{enumerate}
    \item 分$[a,b]$为$n$份,其分点
\[a=x_0<x_1<x_2<\cdots<x_n=b\]
    记$\|P_n\|=\max(x_i-x_{i-1})\quad i=1, 2,\ldots,n$.

    以平面$x=x_i,\; i=0, 1, 2,\ldots,n$截此立体为$n$个小单元,于是整个体积被看作这$n$个小单元的和.
\item  求阶梯函数的总和作为所求立体体积的近似和.考虑这样一个单元,它介于两个截面$x=x_{i-1},x=x_i$之间,于是以$(x_i-x_{i-1})$为高,以$A(x_{i-1})$作为底面积的直棱柱的体积是
  \[  A (x_{i-1}) (x_i-x_{i-1})\]
    于是,我们得到所求的体积$V$的近似表达式
    \[\sum^n_{i=1}A (x_{i-1}) (x_i-x_{i-1})\] 
    \item 对和取极限:当断面无限增加,$\|P_n\|\to 0$时,有
\[V=\lim_{n\to\infty}\sum^n_{i=1}A (x_{i-1}) (x_i-x_{i-1})=\int^b_a A(x)\dd x\]
\end{enumerate}

现在将上面的讨论总结为下面的定理.

\begin{blk}
   {定理} 如果已知一个给定的立体的垂直于一定方向所有的横断面的面积,取这些横断面的垂直方向作为$x$轴的方向,则这立体的体积由公式
\[V=\int^b_a A (x) \dd x\]
表达.其中$A(x)$是横坐标为$x$的横断面的面积,$a$、$b$为这立体的两端断面的横坐标. 
\end{blk}

\begin{example}
    求底边长为$a$, 高为$h$的正四棱锥的体积.
\end{example}

\begin{figure}[htp]
    \centering
\begin{tikzpicture}[>=latex]
    \tkzDefPoints{0/0/A, 3/0/B, 4/1/C}
    \tkzDefPointsBy[translation= from B to C](A){D}
    \tkzInterLL(A,C)(B,D)  \tkzGetPoint{O}
    \tkzDefPoints{0/3.5/E, 5/3.5/F}
    \tkzDefPointBy[projection = onto E--F](O) \tkzGetPoint{OO}
    \draw[|<->|](.7,1.4)--node[fill=white]{$y$}(2.7,1.4);
    \tkzDrawSegments(OO,A OO,B OO,C A,B B,C)
    \tkzDrawSegments[dashed](OO,D OO,O C,D D,A)
    
    \foreach \x in {A,B,C,D,O}
    {
        \tkzDefPointWith[linear, K=.66](OO,\x)  \tkzGetPoint{\x'}
        \tkzDefPointWith[linear, K=.33](OO,\x)  \tkzGetPoint{\x''}
    }
    \tkzDrawSegments(A',B' B',C' A'',B'' B'',C'')
    \tkzDrawSegments[dashed](C',D' D',A' C'',D'' D'',A'')
\draw[->](OO)--(2,4)node[right]{$x$};
\draw(-1,3.5)--(5,3.5);
\draw[dashed](.2,.5)--(O);
\draw(-1,0.5)--(.2,.5);
\draw[<->](-.8,0.5)--node[fill=white]{$h$}(-.8,3.5);
\draw[<->](4.8,0.5)--node[fill=white]{$x$}(4.8,1.5);
\draw[<->](4.8,1.5)--node[fill=white]{$h-x$}(4.8,3.5);
\draw[<->](3.8,1.5)--node[fill=white]{$\Delta x$}(3.8,2.5);
\tkzDefMidPoint(B,C) \tkzGetPoint{G}
\tkzDefMidPoint(B',C') \tkzGetPoint{G'}
\tkzDefMidPoint(B'',C'') \tkzGetPoint{G''}
\tkzDrawSegments[dashed](O,G O',G' O'',G'')
\tkzDefPointWith[linear, K=2](O,G) \tkzGetPoint{H}
\tkzDefPointWith[linear, K=3](O',G') \tkzGetPoint{H'}
\tkzDefPointWith[linear, K=4.5](O'',G'') \tkzGetPoint{H''}
\tkzDrawSegments(G,H G',H' G'',H'')
\tkzLabelPoints[below](O)

\draw[|<->|](0,-0.25)--node[fill=white]{$a$}(3,-.25);


\end{tikzpicture}
    \caption{}
\end{figure}

\begin{solution}
    设$x$为从底面到截面的距离,$y$为截面正方形的边长(图4.12),于是
\[\frac{h-x}{h}=\frac{y}{a}\]
即:$y=\frac{a}{h}(h-x)$.

截面面积
\[A(x)=\frac{a^2}{h^2}(h-x)^2\]
所以
\[\begin{split}
    V&=\int^h_0 \left(\frac{a}{h}\right)^2(h-x)^2 \dd x\\
    &=-\left(\frac{a}{h}\right)^2\frac{(h-x)^3}{3}\Bigg|^h_0=\frac{1}{3}a^2 h
\end{split}\]
\end{solution}

如果立体是一个旋转体,即先给出一条平面曲线
$y=f(x)$, 绕$x$轴旋转所得出的立体,我们的横断面就是以$y$为半径的圆,所以这立体的体积就等于
\[V=\pi \int^b_a y^2\dd x\]
如果$c\le y\le d$,$x=g(y)$绕$y$轴旋转,则
\[V=\pi \int^d_c x^2\dd y\]


\begin{example}
若球的半径是$a$, 球缺的高是$h$, 求球缺的体积.
\end{example}

\begin{figure}[htp]
    \centering
\begin{tikzpicture}[>=latex, scale=1.5]
    \draw[->](-1.5,0)--(2,0)node[right]{$x$};
    \draw[->](0,-1.5)--(0,1.5)node[right]{$y$};
    \draw[very thick](0,0)circle(1);
\draw(.4,1)--(.4,-1.5);
\draw(1,0)--(1,-1.5);
\draw[<->](0,-1.3)--node[below]{$a-h$}(.4,-1.3);
\draw[<->](0.4,-1.1)--node[fill=white]{$h$}(1,-1.1);
\draw[<->](-.2,0)--node[fill=white]{$a$}(-.2,1);

\draw[pattern=north east lines](45:1) arc (45:50:1)--(.64,0)--(1/1.414,0)--(45:1);
\draw[<->](0,1.3)--node[below]{$x$}(.64,1.3);
\draw(.64,1.5)--(.64,0);
\end{tikzpicture}
    \caption{}
\end{figure}


\begin{solution}
球是由圆$x^2+y^2=a^2$绕$x$轴旋转所得到的立体,且球心到球缺底面的距离为$a-h$, 设球心到截面的距离为$x$, 则截面圆的面积$A (x) =\pi y^2=\pi  (a^2-x^2)$.(图4.13)

所以
\[\begin{split}
    V=\pi\int^a_{a-h}(a^2-x^2)\dd x&=\pi\left[a^2 x-\frac{1}{3}x^3\right]\Bigg|^a_{a-h}\\
    &=\frac{\pi h^2}{3}(3a-h)
\end{split} \]
\end{solution}

\begin{example}
    计算底面半径分别为$r$和$R$, 高为$h$的圆台体积$V$.
\end{example}

\begin{solution}
此圆台可看作$xy$平面上的直线段
\[y=\frac{R-r}{h}x+r,\; (0\le x\le h),\quad x=0,\quad x=h,\quad y=0\]
所围成的直角梯形绕$x$轴旋转所得到的立体,从而
\[\begin{split}
    V&=\pi\int^h_0 \left(\frac{R-r}{h}x+r\right)^2 \dd x\\
    &=\pi\left[\left(\frac{R-r}{h}\right)^2\frac{x^3}{3}+2r\cdot \frac{R-r}{h}\cdot \frac{x^2}{2}+r^2 x\right]\Bigg|^h_0\\
    &=\frac{\pi h}{3}(R^2+Rr+r^2)
\end{split}\]
\end{solution}

\begin{example}
在第一象限中,由$y=\frac{x}{1-x}\; (x\ne 1)$, $y=1$, $y=0$和$x=1$所围成的图形绕着直线$x=1$旋转,求所生成的旋转体的体积(图4.14).
\end{example}

\begin{figure}[htp]
    \centering
\begin{tikzpicture}[>=latex, scale=2.5]
    \draw[->](-.25,0)--(1.8,0)node[right]{$x$};
    \draw[->](0,-.25)--(0,1.8)node[right]{$y$};
\draw[domain=0:.6, samples=100, very thick]plot(\x, {\x/(1-\x)});
\draw(1,0)node[below]{1}--(1,1.5)node[above]{$x=1$};
\draw[dashed](0,1)node[left]{1}--(1,1);
\draw[domain=0:.5, samples=100, pattern=north east lines]plot(\x, {\x/(1-\x)})--(1,1)--(1,0)--(0,0);

\draw[<->|](1.1,0)--node[fill=white]{$y$}(1.1,.4);
\draw[domain=0.286:.333, samples=10, pattern=north west lines]plot(\x, {\x/(1-\x)})--(1,.5)--(1,0.4)--(0.286,0.4);
\draw[|<->|](0.286,0.25)--node[fill=white]{$1-x$}(1,.25);

\end{tikzpicture}
    \caption{}
\end{figure}

\begin{solution}
$y$轴垂直于旋转体的横断面,设原点到横断面的距离为$y$, 则该横断面的面积
\[A (y) =\pi  (1-x)^2\]
由于函数$y=\frac{x}{1-x}$是单射的,故反函数存在,由它解出
\[x=\frac{y}{1+y},\qquad 1-x=\frac{1}{1+y}\]
于是
\[A(y)=\frac{\pi}{(1+y)^2}\]
\[\begin{split}
    V=\int^1_0 A(y)\dd y&=\pi\int^1_0 \frac{\dd y}{(1+y)^2}\\
&=-\pi\left(\frac{1}{1+y}\right)\Bigg|^1_0=-\pi\left(\frac{1}{2}-1\right)=\frac{\pi}{2}
\end{split}\]
\end{solution}

\begin{figure}[htp]
    \centering
    \begin{tikzpicture}[>=latex, scale=1.5]
\draw[dashed](0,0) ellipse [x radius=1.5, y radius=.5];
\draw[dashed](0,3) ellipse [x radius=1.5, y radius=.5];
\draw[dashed](-1.5,0)--(-1.5,3);
\draw[dashed](1.5,0)--(1.5,3);
\draw(.5,-.45)node[below right]{$B$}--(-.5,.45)node[above left]{$A$};
\draw[thick] (-.5,.45) .. controls (1.8,2.8) and (2.1,1.6) .. (.5,-.45);
\draw(0,0)--(1.4,1.7);
\draw[dashed](0,0)--(1.4,.2)--(1.4,1.7);
\tkzDefPoints{0/0/O, 1.4/.2/B1, 1.4/1.7/C}
\tkzMarkAngle[mark=none, size=.4, ->](B1,O,C)
\tkzLabelAngle[pos=.6](B1,O,C){$60^{\circ}$}
\tkzLabelPoints[left](O)
\draw(1.5,0) arc [start angle = 0, end angle =-70, x radius=1.5, y radius=.5];
\draw(1.5,0)--(1.5,1.5);
    \end{tikzpicture}    
    \caption{}
\end{figure}

\begin{ex}
\begin{enumerate}
    \item 底面半径为$a$的直圆柱,用通过底面一条直径并与底面成$60^{\circ}$角的平面去
    截,得到图4.15的立体,求其体积.
    \item 由抛物线$y^2=4ax$和直线$x=h$所围成
    的区域绕$x$轴旋转一周,求所生成旋转体的体积.
    \item 求底面半径为$r$, 高为$h$的圆锥体的
    体积.
    \item 设由椭圆$\frac{x^2}{a^2}+\frac{y^2}{b^2}=1$所围成的区域绕$x$轴旋转,求此旋转
    椭圆体的体积.
    \item 由曲线$y=\frac{\ln x}{x}$, $x$轴和通过曲线的最高点的纵坐标所围
    成的区域绕$x$轴旋转,求该旋转体的体积.
    \item 两条抛物线有公共的顶点和公共的对称轴,但位于两个
    互相垂直的平面内,一个运动着的椭圆,其中心在两抛物线的公共轴上,它的平面与公共轴垂直,它的顶点位于两条抛物线上,如果椭圆从公共顶点开始,移动一段距离$h$, 求由此椭圆所生成的体积.
    \item 求$y=4-x^2$与$x$轴之间的区域绕直线$y=-1$一周所生成的体积.
    \item 求圆$x^2+(y-b)^2=a^2\; (0<a<b)$绕$x$轴旋转所生成的旋转体的体积.
\end{enumerate}
\end{ex}

\section{简单初值问题——不定积分的简单应用}

在本章第2节,我们曾说过,一个函数的原函数有无穷多个,它们彼此之间相差一个常数.但是,在许多问题中,我们感兴趣的往往并不是这无穷多个原函数的全体(不定积分),而是某一个特定的原函数,在几何上,就是一条特定的积分曲线.为了求得这条积分曲线,必须知道它所经过的一个点,也就是说,要知道初始条件,在实际问题中,求满足一定初始条件的原函数问题是很多的,下面举一些简单的实例.

\begin{example}
    已知一质点$m$作具有匀加速度$a$的直线运动,设它的初速度为$u$, 求质点在$t$秒后所经过的距离.
\end{example}

\begin{solution}
由于$\frac{\dd v}{\dd t}=a$,所以
\begin{equation}
    v=at+C
\end{equation}
因为,当$t=0$时,$v=u$,代入(4.24)得:$C=u$,所以
\begin{equation}
    v=at+u
\end{equation}
由于$\frac{\dd s}{\dd t}=u+at$,分离变量,把它写成
\[\dd s=(u+at)\dd t\]
两边求不定积分,得
\[\int \dd s=\int (u+at)\dd t\]
\begin{equation}
    s=ut+\frac{1}{2}at^2+C'
\end{equation}
由初始条件:当$t=0$时,$s=0$,代入(4.26),得:$C'=0$,所以
\begin{equation}
    s=ut+\frac{1}{2}at^2
\end{equation}
为所求.(4.25)和(4.27)就是我们常见的运动学公式.
\end{solution}


\begin{example}
假设在原点处斜抛一质点$m$, 它的初速度$v_0$与水平$x$轴成$\alpha$角,又作用在质点$m$上的力只有向下的重力,求质点$m$的轨迹方程.
\end{example}

\begin{solution}
    设质点$m$的坐标是$(x,y)$, 它的质量为$M$克,于是沿着$x$轴和$y$轴的速度分量分别是
    \[v_x=\frac{\dd x}{\dd t},\qquad v_y=\frac{\dd y}{\dd t}\]
    而沿着$x$轴和$y$轴的加速度分量分别是
    \[\frac{\dd v_x}{\dd t}=\frac{\dd^2 x}{\dd t^2},\qquad \frac{\dd v_y}{\dd t}=\frac{\dd^2 y}{\dd t^2}\]
    根据牛顿第二定律,得到
\begin{align}
\begin{cases}
        M\frac{\dd v_x}{\dd t}=0\\
    M\frac{\dd v_y}{\dd t}=-Mg
\end{cases}
\end{align}
即:
\[\begin{cases}
    \dd v_x=0\\ \dd v_y=-g\dd t
\end{cases}\]
所以
\begin{align}
\begin{cases}
    v_x=C_1\\ v_y=-gt+C_2
\end{cases}
\end{align}
常数$C_1$和$C_2$由初始条件决定,所以
\begin{align}
\begin{cases}
v_x=\frac{\dd x}{\dd t}=v_0\cos\alpha\\
v_y=\frac{\dd y}{\dd t}=-gt+v_0\sin\alpha
\end{cases}
\end{align}
即:    
\[\begin{cases}
    \dd x=v_0\cos\alpha \dd t\\ \dd y=(-g t+v_0\sin\alpha)\dd t
\end{cases}\]
再求不定积分得
\[\begin{cases}
    x=(v_0\cos\alpha)t+C_3\\
    y=-\frac{gt^2}{2}+(v_0\sin\alpha)t+C_4
\end{cases}\]
由于,当$t=0$时,$x=y=0$,因此:
\[C_3=C_4=0\]
于是得到质点$m$以时间$t$为参变数的轨迹方程
\begin{align}
\begin{cases}
    x=(v_0\cos\alpha)t\\
    y=-\frac{gt^2}{2}+(v_0\sin\alpha)t
\end{cases}
\end{align}
消去$t$,得:
\[y=\frac{-gx^2}{2v^2_0 \cos^2\alpha}+x\tan\alpha\]
化为标准式,得
\[\left(x-\frac{v^2_0\sin2\alpha}{2g}\right)^2=-\frac{2v^2_0\cos^2\alpha}{g}\left(y-\frac{v^2_0\sin^2\alpha}{2g}\right)\]
这是顶点在$\left(\frac{v^2_0\sin2\alpha}{2g},\frac{v^2_0\sin^2\alpha}{2g}\right)$处的抛物线方程(如图4.16),且从顶点到焦点的方向是$y$轴的负方向,又焦点到准线的距离是$\frac{v^2_0 \cos^2\alpha}{g}$,因此,从顶点到它的上方的准线距离等于$\frac{v^2_0 \cos^2\alpha}{2g}$,而准线到$x$轴的距离就等于
\[\frac{v^2_0 \cos^2\alpha}{2g}+\frac{v^2_0 \sin^2\alpha}{2g}=\frac{v^2_0}{2g}\]
故准线的位置仅依赖于初速度的大小,而与抛射角的大小无关.

\begin{figure}[htp]
    \centering
\begin{tikzpicture}[>=latex]
\draw[->](-.5,0)--(5,0)node[right]{$x$};
\draw[->](0,-.5)--(0,4)node[right]{$y$};
\draw[domain=0:4.2, samples=100, very thick]plot(\x, {0.5*(-\x*\x+4*\x)});
\draw(-.5,2.75)--(4.5,2.75)node[right]{$y=\frac{v^2_0}{2g}$};
\node at (3.5,1)[right]{$y=\frac{-gx^2}{2v^2_0 \cos^2\alpha}+x\tan\alpha$};
\node at (0,0)[below left]{$O$};
\draw(2,2.75)--(2,0);
\tkzDefPoints{2/2/A, 2/1.25/S}
\tkzDrawPoints(A,S)
\tkzLabelPoints[above right](A,S)
\draw[->, thick](0,0)--(1,2)node[right]{$v_0$};


\end{tikzpicture}
    \caption{}
\end{figure}
\end{solution}

下面我们来讨论人口的增长与放射性元素蜕变的问题.

在很多单纯的自然现象中,一个随着时间而变化的量$y(t)$的变化率常常和$y(t)$当时的值成正比例,用数学式表达,即存在一个随该现象而定的常数$k$(与$t$无关),使得
\[y' (t) =ky (t) \]
或
\begin{equation}
    \frac{\dd }{\dd t}y(t)=ky(t)
\end{equation}
人口的增长与放射性元素的蜕变就是这样的两个例子.

假如人口按某一固定的“增长率”$k$增长,譬如时间单位是“年”,增长率2\%, 则$k=2\%$. 设人口随时间变化的函数是$y(t)$, 本来$y(t)$是整数,但由于数量很大,我们可以把它看作一个平滑函数,$y(t)$满足下面的条件:
\[y(t+\Delta t)-y(t)=ky(t)\cdot \Delta t\]
换言之,人口增加数量与这段时间开始的人口数,以及这段时间间隔成比例,也即
\[\frac{y(t+\Delta t)-y(t)}{\Delta t}=ky(t)\]
令$\Delta t\to 0$, 由上式得到
\[\frac{\dd }{\dd t}y(t)=ky(t)\]
也就是说,人口在时刻$t$的瞬时变率与当时的值$y(x)$成正比例.为求出函数$y(t)$, 我们来解具有初始条件的微分方程
\begin{numcases}{}
    \frac{\dd }{\dd t}y(t)=ky(t)\\
    y(0)=A
\end{numcases}
将(4.33)改写成
\[\frac{\dd y(t)}{y(t)}=k\dd t\]
即:$\ln[y(t)]=kt+C$,由初始条件确定常数$C$的值,得:
\[\ln A=C\]
所以:$\ln[y(t)]=kt+\ln A$,
移项,得:
\[\ln\frac{y(t)}{A}=kt\]
由此得:
\[\frac{y(t)}{A}=e^{kt}\]
所以:$y(t)=Ae^{kt}$为所求.



总结上面的讨论,得到以下命题:

\begin{blk}{命题}
    设$y'(t)=ky(t)$,$y(0)=A$,则:
    \[y(t)=Ae^{kt}\]
\end{blk}




\begin{example}
    若某城市现有人口10万,按照每年3\%的比率增长,问几年后人口是现有人口的2倍?
\end{example}

\begin{solution}
    设$t_1$年后人口加倍,于是依题意有
\[10e^{3\%t_1}=2\x10\]
即:$e^{0.03t_1}=2$, 两边取常用对数,得
\[\begin{split}
    0.03t_1\lg e&=\lg2\\
t_1&= \frac{100\lg2}{3\lg e}\approx \frac{100}{3}\cdot\frac{0.3010}{0. 4343}\\
&\approx \frac{100}{3}\cdot 0.6931\approx 23.
\end{split} \]
答:约23年后,人口会加倍.
\end{solution}

从问题的解法看出,人口的加倍期仅与增长率有关,即$t_1=\frac{\ln2}{k}$, 下面是二者之间的对照表.

\begin{center}
\begin{tabular}{c|ccccc}
\hline
    增长率$k$ & 0.05&0.04&0.03&0.02&0.01\\
\hline
加倍期$t_1$ & 约14年& 约17年& 约23年& 约34.5年& 约39年\\
\hline
        \end{tabular}
\end{center}

同样地可把上述命题应用于放射性元素蜕变问题.

\begin{example}
    放射性元素的数量随时间$t$变化的函数是$N(t)$, $A=N(0)$是起算时的数量,$k$是蜕变率,且知放射性元素经过$\Delta t$这段时间的蜕变,它减少的数量$\Delta N(t)=N(t+\Delta t)-N(t)$与在时刻$t$的数量$N(t)$及这段蜕变时间$\Delta t$成比
例,即
\[\Delta N (t) =-kN (t) \cdot \Delta t\]
因为放射性元素不断减少,所以$\Delta N$是负数,问经过多少年放射性元素剩下了一半?
\end{example}

\begin{solution}
    依题意,有
\[\frac{\dd N}{\dd t}=-kN (t) \]
根据本节命题,得到
\[N (t) =Ae^{-kt}\]

设经过$t_1$年,放射性元素减少了一半,通常称$t_1$为放射性元素的半衰期,于是
\[t_1=\frac{\ln 2}{k}\approx \frac{0. 6931}{k}\]
\end{solution}

下面是一些放射性元素的半衰期的简表.

\begin{center}
    \begin{tabular}{c|ccc}
    \hline
        元素 & 氩Ar$^{39}$ &碳C$^{14}$&氢H$^3$  \\
    半衰期 & 269年& 5730年& 12.3年 \\
    \hline
    元素 & 碘I$^{129}$ &铀U$^{238}$& 铀U$^{235}$ \\
    半衰期 &  $16\x10^6$年& $4.5\x 10^9$年& $0.71\x 10^9$年\\
    \hline
            \end{tabular}
    \end{center}

我们可以把岩石或化石中所含的各种微量的放射性元素,当作天然的计时仪表,因为各种放射性元素的微量都可以测量十分准确,所以我们可以通过对残留微量的测定,
用半衰期去推算岩石或化石的“年龄”,甚至于我们所居住的地球的“年龄”.

\begin{ex}
\begin{enumerate}
    \item 求下列积分曲线:
\begin{multicols}{2}
\begin{enumerate}
    \item $\begin{cases}
        y'=2x\\
        y|_{x=x_0}=y_0
    \end{cases}$
    \item $\begin{cases}
        y'=2\sqrt{y}\\
        y|_{x=0}=0
    \end{cases}$
    \item $\begin{cases}
        y'=\frac{2}{3\sqrt[3]{x}}\\
        y|_{x=0}=4
    \end{cases}$
\end{enumerate}
\end{multicols}
\item 在$20.2^{\circ}{\rm C}$, 细菌受到5\%的消毒溶液消毒,每小时细菌死
    亡的比率为11\%, 假如开始时有1000个细菌,那么在24小时后还剩下多少?
    \item 化学反应速度$v$和温度$t$的增加是按照有机体生长的规律
    进行的,当$t=0$时,$v=30$; 当$t=9.90$时,$v=60$, 求反应率以及当$t=20$时,$v$的值.
\item 若函数$y=f(x)$的变率$\frac{\dd y}{\dd x}$与当时的函数值的平方成
    正比例,且当$x=0$时,$y=f(0)=2$; 当$x=1$时,$y=f(1)=4$, 求函数$y=f(x)$.
\end{enumerate} 
\end{ex}

\subsection*{习题4.5}
\begin{enumerate}
    \item 计算下列不定积分:
\begin{multicols}{2}
\begin{enumerate}
    \item $\displaystyle\int \frac{\dd x}{1-3x^2}$
    \item $\displaystyle\int \frac{1-\sin^{-3}x}{1-\cos^2 x}\dd x$
    \item $\displaystyle\int \frac{\dd x}{\sin x\cos x}$
    \item $\displaystyle\int x\sec^2 x\dd x$
    \item $\displaystyle\int xe^{2x+1}\dd x$
    \item $\displaystyle\int \frac{x^2+2}{x(x-1)}\dd x$
\end{enumerate}
\end{multicols}

\item 求下列定积分:
\begin{multicols}{2}
    \begin{enumerate}
        \item  $\displaystyle\int_{-1}^{3} \frac{\dd x}{x^{2}+7 x+10} $
        \item  $\displaystyle\int_{\sqrt{3}}^{0} \frac{\dd x}{2+x^{2}}$
        \item  $\displaystyle\int_{0}^{1} x \sqrt{\left(2-x^{2}\right)^{3}} \dd x$
        \item  $\displaystyle\int_{0}^{\ln 3} \frac{\dd x}{e^{x}+5+6 e^{-x}}$
        \item  $\displaystyle\int_{1}^{\tfrac{7}{5}} \frac{x^{\tfrac{3}{4}}+x^{\tfrac{1}{4}}}{x^{\tfrac{9}{4}}-x^{\tfrac{5}{4}}+x^{\tfrac{1}{4}}} \dd x$
        \item  $\displaystyle\int_{1}^{2} \frac{2-x+x^{2}}{(1+x)(1-x)^{2}} \dd x$
        \item  $\displaystyle\int_{1}^{2} x \ln x \dd x$
        \item  $\displaystyle\int_{-\pi}^{\pi} \sin m x \sin n x \dd x$
        (此处 $m, n$ 是整数且 $m \pm n\ne 0$)
\item  $\displaystyle\int_{\tfrac{a}{\sqrt{3}}}^{a} \frac{\dd x}{\left(a^{2}+x^{2}\right)^{3 / 2}}$
\end{enumerate}
\end{multicols}
\item  求下列极限:
    \begin{enumerate}
\item  $\displaystyle\lim _{n \to  \infty} \frac{1}{n}\left\{\left(1+
\frac{1}{n}\right)^2+\left(1+\frac{2}{n}\right)^2+\cdots+
\left(1+\frac{n}{n}\right)^{2}\right\}$
\item $\displaystyle\lim _{n \to  \infty} \frac{1^{4}+2^{4}+\cdots+n^{4}}{n^{5}}$
\item $\Lim_{n\to\infty}\frac{1}{n}\left(\sin\frac{\pi}{n}+\sin\frac{2\pi}{n}+\cdots+\sin\frac{n\pi}{n}\right)$
\item $\Lim_{n\to\infty}\left(\frac{1}{\sqrt{n+1}}+\frac{1}{\sqrt{n+2}}+\cdots+\frac{1}{\sqrt{n+n}}\right)\cdot \frac{1}{\sqrt{n}}$
\end{enumerate}

\item 利用求$y=\frac{1}{x}$的定积分,证明
\[\ln(n+1)<1+\frac{1}{2}+\frac{1}{3}+\cdots+\frac{1}{n}<1+\ln n\]
\item 设$0<a\le 1$,求证:
\[\int^a_0\frac{\dd x}{x^3+1}<\int^a_0\frac{\dd x}{x^4+1} \]
\item 设$0\le x\le \frac{\pi}{4}$,利用不等式$0\le \sin x\le x$证明:
\[\frac{\pi}{4}<\int^{\pi/4}_0\frac{\dd x}{\sqrt{1-\sin x}}<2-\sqrt{4-\pi}\]

\item 若$m,n$是正整数,求证:
\[\int^1_0 x^m (1-x)^n\dd x=\frac{n}{m+1}\int^1_0x^{m+1}(1-x)^{n-1}\dd x \]
\item 求证:$\displaystyle\int^\pi_0 xf(\sin x)\dd x=\frac{\pi}{2}\int^\pi_0f(\sin x)\dd x$
\item 若$I_n=\displaystyle\int^{\pi/4}_0 \tan^n \theta \dd \theta\quad (n\ge 2)$. 试求$I_n$和$I_{n-2}$的关系式.
\item 若$I_n=\displaystyle\int \frac{\dd x}{(x^2+a^2)^n}$,求证:
\[2na^2I_{n+1}=\frac{x}{(x^2+a^2)^n}+(2n-1)I_n\]
\item 设$S_n=\frac{1}{\sqrt{1}}+\frac{1}{\sqrt{2}}+\cdots+\frac{1}{\sqrt{n}}$,$n$为正整数.
\[T_n=\frac{1}{\sqrt{1+\frac{1}{2}}}+\frac{1}{\sqrt{2+\frac{1}{2}}}+\cdots+\frac{1}{\sqrt{n+\frac{1}{2}}}\]
求证:
\begin{multicols}{2}
\begin{enumerate}
    \item $S_n\ge \displaystyle\int^{n+1}_1 \frac{\dd x}{\sqrt{x}}$
    \item $\Lim_{n\to\infty}\frac{1}{S_n}=0$
    \item $S_{n+1}-1\le T_n\le S_n$
    \item $\Lim_{n\to\infty}\frac{T_n}{S_n}=1$
\end{enumerate}
\end{multicols}

\item 设$f(x)=(ax+b)e^{x/2}$, 求$a,b$的值,使得等式
\[f (x) =e^{x/2}-1+\frac{1}{2}\int^x_0 f (t) \dd t\]
对于任何$x$的值都成立.
\item 求由曲线$y=\frac{4}{x^2}$, $y=x-1$, $x=1$所围区域的面积.
\item 求在曲线$x^2+y^2=9$和$y=\frac{1}{3}(-x^2+1)$之间位于
第二象限那部分的面积.
\item 求积分$I(a)=\displaystyle\int^1_{-1}|x-a|e^x\dd x$在$|a|\le 1$时的最大值.
\item 设$0<t<\pi$,考察$x$的函数$f(x)=2\sin x\sin(t-x)$
\begin{enumerate}
    \item 画出$f(x)$在区间$0\le x\le \pi$上的图象,再设
    \[F (t) =\int^\pi_0 |f (x)|\dd x\]
根据所作的图象说明$F(t)$所表示的是什么?
\item 用关于$t$的式子写出$F(t)$, 并求出使$F(t)$取最小值时$t$的值.
\end{enumerate}
\item  设$T_1$是由抛物线$y=4x^2$和直线$x=a$, $x=1$, $y=0$所
围成的区域,$T_2$是抛物线$y=4x^2$和二直线$x=a$, $y=0$围成的区域,但是$0<a<1$,试求:
\begin{enumerate}
\item $T_1$绕$x$轴旋转而成的旋转体体积$V_1$;    \item $T_2$绕$y$轴旋转而成的旋转体体积$V_2$;    \item 使$V_1+V_2$为最大时$a$的值.
\end{enumerate}
\item 设$0\le t\le \frac{\pi}{2}$,曲线$y=\sin x$与三条直线$x=t$, $x=2t$, 
$y=0$围成部分绕$x$轴旋转所成的旋转体的体积为$V(t)$.当$t=\alpha$时$V(t)$最大,试求$\cos\alpha$的值.
\item 设曲线$y=f(x)$通过点$P(3, 4)$, 并满足微分方程
$\frac{\dd y}{\dd x}=\frac{y}{x}$,
求此积分曲线.

\item 若$\frac{\dd y}{\dd x}=\cos(x+y)$,求不定积分.(提示:设$z=x+y$)
\end{enumerate}




















\end{document}